\section{Ideenfindung}

Wie eingangs beschrieben stand zu Beginn des Projekts noch ein weites Feld an Themen und Anwendungen zur Verfügung. 
Um in diesem Feld ein konkretes Projektziel zu identifizieren, das innerhalb eines Projektseminars erreicht werden kann, teilte sich das Team in Absprache mit den Projektbetreuern in fünf Gruppen auf, die jeweils einen geeigneten Vorschlag für das Projekt erarbeiten sollten. 
Bei der Abstimmung über die Möglichkeiten kristallisierten sich dabei zwei Favoriten heraus. 
Anstatt uns vollständig für einen dieser Vorschläge zu entscheiden, kombinierten wir diese Alternativen. Damit sieht das Projekt in der finalen Fassung vor, das Meinungsbild auf Twitter zu einem bestimmten Thema aus einer Reihe von Gesichtspunkten darstellen zu können. 
Von der einen Projektidee stammen die Gesichtspunkte des zeitlichen Verlaufs, der räumlichen Verbreitung und der aktuellen Stimmungslage. Das Clustering von Tweets gemäß ihrer Gemeinsamkeiten bildete die zentrale Idee des anderen Vorschlages und wurde als weiter Gesichtspunkt aufgenommen.
Dies wurde um die Idee des ersten Vorschlags erweitert, Module anzubieten, welche die verfügbaren Daten in einem spezifischen Kontext interpretieren und dabei auch weitere externe Daten heranziehen. Als konkretes Modul sollte dabei ein sogenanntes "`Kinomodul"' implementiert werden, welches Informationen aus externen Filmdatenbanken mit den Ergebnissen aus Twitter kombiniert, um Filme vergleichen zu können und Empfehlungen möglich zu machen.