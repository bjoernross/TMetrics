\section{Scrum}
Wie bereits erwähnt, wurde für die Organisation des Projektes auf Hilfsmittel aus der agilen Softwareentwicklung zurückgegriffen. Im Gegensatz zur klassischen Softwareentwicklung z.\,B. nach dem Wasserfallmodell mit teils langen Phasen, sind die Hauptmerkmale der agilen Softwareentwicklung vor allem flexiblere und oftmals vor allem zeitlich weniger aufwändige Methoden. Im Vordergrund stehen neben weiteren Merkmalen Schlagwörter, wie z.\,B. kurze Iterationen, testgetriebene Entwicklung oder häufige Refaktorierungen. %TODO: evtl umformulieren... -> würde das nicht explizit Schlagwörter nennen und vielleicht auch das "weitere Merkmale" weglassen
Die Anzahl der Methoden, die sich unter dem Stichwort agile Softwareentwicklung zusammenfassen lassen, ist relativ groß, sodass sich kaum sämtliche Methoden konsequent innerhalb eines Projektes umsetzen lassen. Es haben sich daher verschiedene Vorgehensmodelle, wie z.\,B. Extreme Programming, Kanban oder Scrum entwickelt. Da wir für die Projektorganisation die Methoden aus Scrum verwendet haben, möchten wir im Folgenden einen Überblick darüber geben, welche Methoden von uns angewandt wurden und auf welche wir aus welchen Gründen verzichten mussten.

\subsection{Grundmerkmale}
Ohne Scrum an dieser Stelle im Detail erklären zu wollen, seien einige wichtige Grundmerkmale genannt, welche sich auch in unserem Projektseminar wiedergefunden haben.

Im Rahmen von Scrum werden die Anforderungen zunächst grob als sogenannte User Stories formuliert und anschließend in möglichst spezifische einzelne Teilaspekte (sogenannte Arbeitspakete) unterteilt. Nachdem die Projektidee ausgereift ist, wird dies innerhalb des Teams im Sinne von Scrum umgesetzt. Anschließend werden die Aufwände der Arbeitspakete innerhalb des Teams geschätzt. Details hierzu folgen in einem der nächsten Abschnitte.

Da das Vorgehen über die komplette Entwicklungszeit nicht aus einzelnen langen und aufeinander folgenden Phasen (wie z.\,B. im Wasserfallmodell) besteht, sondern bei Scrum auf kurze ca. 2-3 Wochen dauernde und wiederholende Iterationen gesetzt wird, folgt nach der Aufwandschätzung das Aufteilen der vorhandenen User Stories auf die jeweils kommende Iteration.
Das von Scrum erklärte und von uns in der Regel erreichte Ziel ist hierbei, am Ende einer jeden Iteration ein funktionsfähiges Stück Software in der Hand zu halten.

Ein weiteres erklärtes Grundprinzip von Scrum ist eine rege Kommunikation, gestützt durch häufige und verschiedene Arten von Meetings. Auch wir konnten im Rahmen des Projektseminars mit den verschiedenen Arten von Meetings unsere Erfahrungen sammeln.

In Kombination mit den kurzen Iterationen und dem Endziel einer jeden Iteration führt dies dazu, dass das Entwicklungsteam schnell auf (sich ggf. ändernde) Kundenwünsche reagieren kann und so mögliche und vor allem langfristig folgenreiche Missverständnisse in großen Teilen verhindert werden können. Auch auf unser Projektteam kamen im Rahmen der Entwicklung einige neue Kundenwünsche zu, welche zumeist erfolgreich und zeitnah umgesetzt werden konnten.

Um den Erfolg von Scrum sicherzustellen, ist auf die Einhaltung dieser Grundprinzipien zu achten. Soweit dies im Rahmen einer Lehrveranstaltung möglich ist, wurde dies vom Projektteam auch gemacht, sodass man festhalten kann, dass Scrum für unser Projekt definitiv ein Erfolg war.

Um die Grundprinzipien und die Umsetzung innerhalb unseres Projektes ein wenig weiter zu beleuchten, werden wir im Folgenden auf einige Rollen und Methoden im Detail eingehen.


\subsection{Rollen}
\subsubsection{Scrum-Team}
Innerhalb des Scrum-Modells gibt es verschiedene Rollen. Die Projektteammitglieder nahmen hierbei natürlich in erster Linie die Rolle der Entwickler ein. Das Entwicklerteam wird bei Scrum auch als Scrum-Team bezeichnet. Ein wichtiges Merkmal ist, dass dieses Team in großen Teilen selbstverwaltet arbeitet und eventuell auftretende Probleme intern zu lösen versucht. Dies hat bei uns sehr gut funktioniert. So wurden die verfügbaren Aufgaben ohne Komplikationen innerhalb des Teams verteilt und auch ein auftretendes Kommunikationsproblem konnte autark innerhalb des Teams gelöst werden.

\subsubsection{Product Owner}
Die Rolle des Product Owners ist am ehesten mit der eines Projektleiters oder Firmeninhabers in einem klassischen Szenario zu vergleichen. Der Product Owner ist derjenige, der das fertige Produkt nach außen verkauft und mit dem Kunden in engem Kontakt steht. Letztendlich nimmt er daher immer auch die Rolle eines Vermittlers zwischen Kunden und Entwicklern ein, die ansonsten nicht unbedingt in direktem Kontakt miteinander stehen müssen.

In unserem Szenario wurde diese Rolle durch einen unserer Projektbetreuer besetzt. Er wohnte in dieser Rolle z.\,B. den sogenannten Sprint-Planning-Meetings bei und sortierte in diesen Meetings die gefundenen User Stories nach einer Priorität im Sinne des Kunden. Generell fanden im Rahmen des Projektseminars regelmäßige Meetings mit dem Product Owner statt, welche je nach aktueller Situation innerhalb der Iteration verschiedene Ausprägungen erhielten und somit nicht nur zum Sprint Planning wurden, sondern z.\,B. auch zum Sprint Review wurden.

\subsubsection{Scrum Master}
Der Scrum Master ist mit dem Entwicklerteam eng verbunden, aber stellt eigentlich kein Mitglied des Entwicklerteams, sondern eine Art Bindeglied zum Product Owner dar. Der Scrum Master hält dem Entwicklerteam den Rücken frei und sorgt z.\,B. dafür, dass die benötigten externen Anforderungen erfüllt sind. Damit hat er vor allem zu Beginn eines Projektes einen etwas erhöhten organisatorischen Aufwand zu bewältigen und nimmt in der restlichen Zeit eine eher moderierende Rolle in den verschiedenen Meetings ein. Normalerweise wechselt im Verlaufe eines Projektes der Scrum Master nicht, sondern wird durchgehend durch die selbe Person repräsentiert. Damit alle Projektmitglieder aber einmal in die Position des Scrum Masters kommen konnten, mussten wir mit diesen Prinzipien leider brechen. Der Scrum Master war also bei uns abweichend von Scrum sowohl durch wechselnde Personen repräsentiert als auch stets ein Mitglied des Entwicklerteams. Das tat dem Moderieren der verschiedenen Meetings aber keinen großen Abbruch, sodass letztendlich alle Projektteammitglieder von den Einblicken in diese Rolle profitiert haben.

\subsubsection{Kunde}
Als weitere Rolle gibt es natürlich den Kunden, welcher in unserem Szenario durch den anderen Projektbetreuer verkörpert wurde. Der Kunde stellt natürlich die anfänglichen grundsätzlichen Anforderungen an das Projekt und beauftragt das gesamte Team mit der Entwicklung. Obwohl die Projektidee in diesem Fall durch das Projektteam selbst entwickelt wurde, hat der Kunde es sicht nicht nehmen lassen im Verlauf des Projektes ändernde Anforderungen einzubringen, wie z.\,B. dem Wunsch nach einem "'Kachelinterface"'. Im eigentlich Scrum-Sinne interagiert der Kunde nur bedingt mit dem gesamtem Entwicklerteam, hiervon wurde im Rahmen des Projektseminars ebenfalls ein wenig abgewichen. Dies hat aber insgesamt zu keiner negativen Beeinflussung im Sinne von Scrum geführt und sich teilweise durch kurze Kommunikationswege eher als Vorteil herausgestellt.


\subsection{Verwendete Methoden}
\subsubsection{User Story}
Die einzelnen Aufgabenkomplexe werden als User Stories bezeichnet, entsprechen also einer eher umgangsprachlichen Formulierung aus der Sicht des Benutzers, was in der Anwendung konkret passieren soll.
Konkretisiert werden diese User Stories durch die Entwickler in dem sie auf kleinere konkrete Aufgaben, sogenannte Tasks unterteilt werden.
User Stories als auch Tasks wurden von uns auf (unterschiedlich großen) Karteikarten festgehalten, um diese später flexibel dem Scrum-Board hinzufügen zu können. Details zum Scrum-Board folgen in einem späteren Abschnitt.

Folgende beispielhafte User Story wurde verfasst, um die Tag-Cloud-Anzeige zu implementierten:
\begin{center}
\begin{tabular}{|lll|} \hline
2.4 & \multicolumn{2}{c|}{Als Benutzer möchte ich Tag-Clouds haben} \\
    &                     &           \\
    & \small{Geschätzter Aufwand} & \small{5 Punkte}  \\
    & \small{Benötigte Stunden} & \small{29}     \\ \hline
\end{tabular}
\end{center}
In der Abbildung \ref{fig:viewTagCloud} ist die Implementierung im Frontend abgebildet.
Für die Implementierung der Tag Cloud wird eine REST-Anfrage benötigt, die die benötigten
Datenbank\-anfragen macht, sowie die 
Visualisierung im Frontend.
Deswegen wurde die User Story in folgende Tasks aufgebrochen:
\begin{center}
\begin{tabular}{|lll|} \hline
2.4.1 & \multicolumn{2}{c|}{REST-Anfrage implementieren} \\
    &                     &           \\
    & \small{Aufwand} & \small{2 Punkte}  \\ \cline{2-3}
    & \multicolumn{1}{|l}{\small{Beteiligte Personen}} & \small{Benötigte Stunden} \\ 
    & \multicolumn{1}{|l}{\small{Person1}}             & \small{2}                    \\
    & \multicolumn{1}{|l}{\small{Person2}}             & \small{4}                    \\
\hline
\end{tabular}
\end{center}
\begin{center}
\begin{tabular}{|lll|} \hline
2.4.2 & \multicolumn{2}{c|}{Visualisierung im Frontend implementieren} \\
    &                     &           \\
    & \small{Aufwand} & \small{3 Punkte}  \\ \cline{2-3}
    & \multicolumn{1}{|l}{\small{Beteiligte Personen}} & \small{Benötigte Stunden} \\ 
    & \multicolumn{1}{|l}{\small{Person2}}             & \small{12}                    \\
    & \multicolumn{1}{|l}{\small{Person3}}             & \small{11}                    \\
\hline
\end{tabular}
\end{center}

\noindent
An der Nummerierung oben links erkennt man, dass die User Story und die Tasks zu der zweiten Iteration gehören, die User Story
hat die Nummer vier, wobei die Nummerierung keine besondere Rolle spielt und lediglich der eindeutigen Identifizierung dient.  Die Tasks haben entsprechend die Nummern eins und zwei. 

Auf den User-Story-Karten wurde protokolliert, wie hoch der Aufwand in Punkten (hier 5) dieser Story ist, welcher sich durch die Summe des Aufwandes der einzelnen Tasks ergibt. In diesem Fall 2 Punkte für den Task 2.4.1 und 3 Punkte für den Task 2.4.2. 
Ebenfalls wurde an den User Stories notiert, wie hoch der benötigte Aufwand in Stunden tatsächlich ist. Dieser setzt sich wie die Punkte aus der Summe der benötigten Stunden für die einzelnen Tasks zusammen, welche von dem jeweiligen Entwickler nach Fertigstellung eines Tasks auf der Task-Karteikarte festgehalten wurde.

Durch diese Art der Buchführung über verbrauchte Punkte und Stunden, ist es relativ einfach möglich, den jeweils aktuellen Fortschritt innerhalb der Iteration zu kontrollieren und sich einen Überblick darüber zu verschaffen, ob am Ende der Iteration alle Tasks und Stories fertig sind oder ob noch etwas offen bleiben wird. Mehr dazu später im Abschnitt über den Burndownchart.

\subsubsection{Planning Poker}
Im Gegensatz beispielsweise zum Wasserfallmodell wird bei Scrum der Aufwand nicht vom Projektleiter oder Geschäftsführer bestimmt und quasi "`von oben"' diktiert, sondern der Aufwand für die Implementierung verschiedener Anforderungen wird aus dem Team heraus bewertet und geschätzt. Hierzu bedienen sich die beteiligten Entwickler einem Schätzverfahren, das unter dem Namen Planning Poker bekannt ist. Als Referenz-Aufwand wird eine im Team bekannte Aufgabe heran gezogen, anhand derer der Aufwand einer neuen Anforderung geschätzt werden soll. Der Hauptvorteil des Planning Pokers liegt darin, dass die beteiligten Personen sich zunächst nicht gegenseitig beeinflussen und daher trotz der allgemein bekannten Referenz-Aufgabe sehr abweichende Schätzungen entstehen können. 
Diese abweichenden Schätzungen eröffnen dann Diskussionen über den tatsächlichen Umfang und Aufwand der Aufgaben. Dies hatte auch bei uns im Projektteam die Folge, dass bei nicht-trivialen Aufgaben der Sachverstand aller Mitglieder einging und damit für jeden eine genauere Kenntnis der Anforderungen erreicht wurde.

Wie wir anhand der User Stories und einzelnen Tasks bereits gesehen haben, wird dieser Aufwand in Punkten geschätzt, wobei Punkte bei Scrum ausdrücklich nicht den Arbeitsstunden entsprechen, sondern als eine Art Einschätzung der Komplexität der Aufgabe zu verstehen sind. Dies ist von Vorteil, wenn sich Entwickler mit unterschiedlichen Erfahrungswerten an die selbe Aufgabe machen und sicherlich unterschiedliche Zeit benötigen würden. Da der Punktestand gemessen werden kann, bietet dies einen besseren Überblick über den Fortschritt der Iteration.

Die Summe der geschätzten Teilaufgaben und User Stories landen gesammelt in dem Product Backlog, welches somit das Pendant zu dem sonst bekannten Pflichten- oder Lastenheft darstellt. Das Pendant zu dem Product Backlog war in unserem Fall die Sammlung der noch nicht geplanten User Stories, welche auf Karteikarten notiert wurden, um diese inklusive der zugehörigen Tasks, an unserem Scrum-Board anbringen zu können.


\subsubsection{Scrum-Board}
Das Scrum-Board ist eine Art Tafel oder White-Board, welches an einer für alle Beteiligten gut sichtbaren Stelle aufgehängt wird. Es ist ein sehr essenzieller Bestandteil von Scrum und dient dazu, einen Überblick über den Fortschritt der einzelnen Aufgaben zu geben, sowie den Entwicklern die Möglichkeit zu geben sich flexibel noch nicht bearbeitete Aufgaben heraus zu nehmen.
Aus der durch den Product Owner priorisierten Liste von User Stories aus dem Sprint Backlog, kann nun eine Story heraus gegriffen und an dem Scrum-Board angebracht werden. Dies signalisiert jedem, dass aktuell daran gearbeitet wird. Die einzelnen Arbeitspakete können aber dennoch von unterschiedlichen Entwicklern bearbeitet werden. Eine Idee von Scrum ist, dass sich ein Entwickler hier durchaus auch mal eine Aufgabe aus einem Gebiet nehmen kann, in dem er nicht unbedingt vollständig eingearbeitet ist, um so seinen eigenen Horizont zu erweitern und auch durch einen anderen Blickwinkel der Entwicklung dieser User Story evtl neue Impulse hinzuzufügen.
Aufgrund steigender Komplexität der einzelnen Komponenten hat dieser Fachgebiets-Wechsel-Effekt im Laufe des Projektes nach und nach abgenommen, was jedoch durchaus zu erwarten war.
Generell war der Überblick der offenen, geschlossenen und in Bearbeitung befindlichen Stories und Tasks jedoch sehr hilfreich.

\subsubsection{Burndownchart}
Neben dem Scrum-Board ist das Burndownchart ein eben so wichtiger Bestandteil von Scrum. Immer wenn am Scrum-Board eine Aufgabe aus dem Bereich der die in Bearbeitung befindlichen Stories und Tasks kennzeichnet, heraus genommen wird, wird der in Punkten auf der Story notierte Aufwand von der Summe der Aufwände der Iteration abgezogen und ein Restwert des Aufwandes wird ermittelt. Dieser Ist-Wert des Rest-Aufwands wird anschaulich an einer Kurve dargestellt, welche als Burndownchart bezeichnet wird.
Idealer Weise verläuft diese Kurve möglichst nah an einer gedachten (und je nach Ausprägung auch eingezeichneten) Ideallinie, welche diagonal von 100\% des Aufwandes bis zu 0\% über den zeitlichen Verlauf der Iteration verläuft. In der Praxis liegt diese Linie doch oftmals oberhalb dieser Ideallinie, da Stories erst nach Beendigung vollständig von dem Rest-Wert abgezogen werden und da durch Probleme die im Verlauf der Iteration auftreten, die Summe des Gesamtaufwandes oft noch ansteigen kann. Um dies besser zu visualisieren haben wir eine weitere Linie in unserem Burndownchart eingeführt, welche den 100\% Wert des Aufwandes darstellt und an entsprechenden Stellen dann einen Knick nach oben bekommt.
Somit konnte man sich dennoch einen guten Überblick darüber verschaffen, wie gut man mit den geplanten Aufgaben dieser Iteration im Zeitplan ist.


\subsubsection{Definition of Done und Testgetriebene Entwicklung}
Wie im vorherigen Abschnitt beschrieben, konnte man unserem Scrum-Board stets entnehmen, welche Aufgaben bereits abgeschlossen sind. Damit sich zwischen den Mitgliedern keine unterschiedlichen Auffassungen etablieren, wann eine Aufgabe als abgeschlossen bezeichnet werden kann, sieht Scrum eine sogenannte \textit{definition of done} (DoD) vor. Eine solche DoD wurde vom Team innerhalb der ersten Wochen erarbeitet und stellte seitdem ein verbindliches Regelwerk dar. Vor allem war ein fester Bestandteil der DoD, dass die jeweilige Komponente getestet worden sein soll, bevor die Aufgabe als abgeschlossen bezeichnet werden kann. Dies wurde durch automatisierte Tests erreicht, worüber in einem kommenden Abschnitt noch mehr zu erfahren sein wird.

\subsection{Scrum, But}
Da bislang die Scrum-Methoden betrachtet wurden, welche vollständig oder zumindest in großen Teilen nach Scrum-Vorgabe durch uns genutzt und umgesetzt wurden, möchten wir noch auf einige Aspekte eingehen, welche in unseren Augen nur teilweise bzw. mit einigen Abweichungen von uns genutzt wurden.
\subsubsection{Daily Scrums}
Diese Form von Meeting sollte wie der Name schon andeutet täglich stattfinden. Da sich aber kaum ein täglicher fixer Termin finden ließ, an dem stets alle oder zumindest die meisten Teammitglieder hätten anwesend sein können, wurde innerhalb unseres Team hieraus ein D-Daily-Scrum gemacht, % d-daily? - A: sic! :P aber von mir aus, schmeisst es raus oder formuliert es um. Mir gefällts so!
 sodass wir uns lediglich Dienstags und Donnerstag zu dieser Art von Meeting zusammen gefunden haben.
Inhaltlich geht es beim Daily Scrum eigentlich darum, dass jedes Mitglied kurz darlegt, was seit dem letzten Daily-Scrum geschehen ist, also letztendlich was entwickelt wurde und was bis zum nächsten Daily Scrum geschehen soll. Ebenfalls sollte kurz erwähnt werden, ob mögliche Komplikationen zu erkennen sind, welche das Ziel bis zum nächsten Daily Scrum gefährden könnten. Dies würde der Scrum Master dann zum Anlass nehmen, aktiv zu werden um die möglichen Hindernisse aus dem Weg zu räumen.

Um den Charakter eines Kurz-Meetings zu stärken und nicht in langwierige Diskussionen abzudriften, wird empfohlen, diese Art von Meetings im Stehen abzuhalten, möglichst auch außerhalb des regulären Arbeitsbereiches. Dies wurde vom Team auch umgesetzt, indem man sich im Erdgeschoss des Institutes traf und besagte Punkte ansprach. Vermutlich aufgrund der nicht täglich stattfindenden Daily-Scrum-Meetings neigten die Meetings teilweise zu längeren Diskussionen und übertrafen dann definitiv die reine Bestandsaufnahme.

\subsubsection{Retrospektive und Sprint Review}
Aufgrund der eingeschränkten Anzahl an gemeinsamen Terminen pro Woche ließen sich die Retrospektiven und Sprint Reviews leider nicht immer vollständig so umsetzen, wie es in Scrum eigentlich vorgesehen ist. Es fanden zwar vergleichbare Termine statt, aber es war selten der Fall, dass eine Retrospektive im Scrum-Sinne zur Selbstreflektion des Entwicklerteams genutzt wurde und einen eigenständigen Termin zu dem Sprint Review darstellte. Meist war das Meeting zum Ende eines Sprints eine Mischform dieser beiden Scrum-Bestandteile.
Dem grundsätzlichen Scrum-Ablauf hat dies in unserem Szenario jedoch keinen Abbruch getan, sodass Scrum für uns dennoch sehr gut funktioniert hat.

\subsubsection{User-Story}
Die Aufgaben wurden als User Stories verfasst (aus Sicht des Benutzers) und 
anschließend in Tasks (kleinere Aufgaben) aufgebrochen. 
Diese wurden auf Karteikarten geschrieben und dem Scrum-Board hinzugefügt. 
