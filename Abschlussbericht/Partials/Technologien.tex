\section{Technologien}
\label{sec:Tech}
Die bei unserem Projektseminar eingesetzten Technologien wurden zum größten Teil auf Grund von persönlicher Erfahrungen eines oder mehrerer Teammitglieder ausgewählt. Die Entscheidung, unser Programm hauptsächlich mit Java zu schreiben, beruht allein auf der Tatsache, dass alle Teammitglieder ausreichend viel Erfahrung mit dieser Programmiersprache hatten. Die Entscheidung für Java als Programmiersprache führte bei den Recherchen bezüglich der Kommunikation mit der Twitter-API zu der Java-Bibliothek Twitter4J \cite{Twitter4J}. Diese kapselt die Kommunikation mit Twitter komplett und liefert die Ergebnisse einer Anfrage an Twitter als einfaches Java-Objekt.

Die Entscheidung für eine bestimmte Datenbank war ebenfalls von Vorwissen im Team geprägt. MySQL ist die Datenbank, mit der alle Mitglieder bereits erste Erfahrungen gemacht hatten. Eine tatsächliche Evaluierung der zur Verfügung stehenden Datenbanken wurde nicht durchgeführt. Eine Auswahl einer Datenbank, die auf unsere Anforderungen besser spezialisiert ist, hätte allerdings unter Umständen aufgetretene Performanceprobleme besser lösbar gemacht.

Die Entscheidung, die Ausgabe unserer Analysen in einer Webseite darzustellen, begründet sich zum einen auf dem expliziten Wunsch des Teams, dies in unser Projekt mit aufzunehmen. Zum anderen erschien dem Team eine Benutzeroberfläche für eine Java-Applikation nicht zeitgemäß. Daher wurde entschieden, für das Frontend JavaScript in Verbindung mit jQuery \cite{jQuery} zu verwenden. Die Graphen in unserer Anzeige wurden mit dem Framework Highcharts erstellt \cite{Highcharts}. Die Entscheidung für Highcharts beruhte in erster Linie wieder auf vorher vorhandene Erfahrungen einiger Teammitglieder. Dieses Framework ermöglicht, es einfache Graphen schnell und unkompliziert zu erstellen. Für die komplexeren Anzeigen, wie zum Beispiel die Tag-Cloud, wurde nach einer Recherche das Framework Data-Driven Documents (D3) verwendet \cite{d3}. Diese umfangreiche Visualisierungsbibliothek löst das Problem einer Visualisierung von Daten allerdings mit einer anderen Herangehensweise als Highcharts. Hierbei werden dem Entwickler alle Werkzeuge an die Hand gegeben, die benötigt werden, um eine beliebige Vektorgrafik zu erstellen und zu animieren.
