
\begin{frame}{Module - Daemon - Aufgaben}
	\begin{itemize}
		\item Sammeln von Tweets zu Begriffen
		\item Speichern von Tweets in der Datenbank
		\item Verwendung der Twitter API
	\end{itemize}
	
	\pause
	
	\begin{center}
		\begin{tabular}{ccccc}
		\includegraphics[width=0.1\textwidth]{../img/daemon/Database.png} &  $\leftrightarrow$ & \includegraphics[width=0.1\textwidth]{../img/daemon/java.png} & $\leftrightarrow$ &  \includegraphics[width=0.1\textwidth]{../img/daemon/Twitter_logo_blue.pdf} \\
		Datenbank & & Daemon & & TwitterAPI \\
		\end{tabular}

	\end{center}
	
	\pause
	
	\begin{itemize}
		\item Sentiment berechnen
	\end{itemize}
	
\end{frame}

\begin{frame}{Module - Daemon - Twitter-API}
	\begin{itemize}
		\item Twitter hat eine offizielle API
		\item Search-API: REST-Anfragen liefern JSON-Objekte
		\item Twitter API unterliegt Restriktionen % (Rate Limit pro Profil, Art, wie Tweets abgefragt werden)
		\item Kommunikation mit Twitter über Twitter4J
	\end{itemize}
\end{frame}

\begin{frame}{Module - Daemon - Suchstrategie}
	\begin{itemize}
		\item Suche zu Suchbegriff immer rückwärts
		\item Suche: ältester Tweet ist Beschränkung für nächste Anfrage
		\item Tweets sind zeitlich sortiert
		\pause
		\item Keine älteren Tweets mehr: Startzeitpunkt auf jetzt setzen
		\item Ab diesem Zeitpunkt wird erneut rückwärts gesucht
	\end{itemize}
\end{frame}

\begin{frame}[t]{Module - Daemon - Suchstrategie}
	\begin{center}
	\includegraphics<1>[width=0.9\textwidth]{../img/daemon/SearchStrategy1.pdf}
	\includegraphics<2>[width=0.9\textwidth]{../img/daemon/SearchStrategy1_5.pdf}
	\includegraphics<3>[width=0.9\textwidth]{../img/daemon/SearchStrategy2.pdf}
	\includegraphics<4>[width=0.9\textwidth]{../img/daemon/SearchStrategy3.pdf}
	\includegraphics<5>[width=0.9\textwidth]{../img/daemon/SearchStrategy4.pdf}
	\end{center}
\end{frame}

\begin{frame}{Module - Daemon - Parallele Suche}
	\begin{itemize}
		\item Parallele Suche: mehr Tweets in kürzer Zeit finden
		\item Idee: mehrere Profile nutzen (Multi-Threading)
		\item Daemon als Master-Worker-Architektur realisieren
	\end{itemize}
\end{frame}

\begin{frame}{Module - Daemon - Scheduling der Suchbegriffe}
	\begin{itemize}
		\item Wie teilt man die Suchbegriffe auf?
		\item Short-Terms, kaum neue Tweets
		\item Long-Terms, viele neuen Tweets
		\item Worker erhält sowohl Short- als auch Long-Terms
	\end{itemize}
\end{frame}

%\begin{frame}[t]{Architektur}
%	\vspace{-1cm}
%	\begin{center}
%	\includegraphics<1>[height=\textheight]{../img/daemon/DataFlow1.pdf}
%	\includegraphics<2>[height=\textheight]{../img/daemon/DataFlow2.pdf}
%	\includegraphics<3>[height=\textheight]{../img/daemon/DataFlow3.pdf}
%	\includegraphics<4>[height=\textheight]{../img/daemon/DataFlow4.pdf}
%%	\includegraphics<5>[height=\textheight]{../img/daemon/DataFlow1.pdf}
%	\end{center}
%\end{frame}


\begin{frame}{Module - Daemon - Erfahrungen}
	\begin{itemize}
		\item Multi-Threading ist komplex
		\item Probleme mit dem Speicherverbrauch der JVM
		\item Konsistenz der verschiedenen Teile komplex
	\end{itemize}
\end{frame}
