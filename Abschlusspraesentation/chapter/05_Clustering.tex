% \begin{frame}{Module - Clustering - Grundidee}
% %Grundidee:
% \begin{itemize}
%  \item Tweets visuell als Punkte darstellen,
%  \item Tweets nach Themen gruppieren,
% % 	damit man einen Eindruck bekommt 
% %        welche unterschiedliche ``Meinungen'' es zu einem Begriff gibt oder welche 
% %        Hashtags zusammen verwendet werden,
%  \item die Distanzen zeigen die Unterschiede.
% %  zwischen den dargestellten Punkten sollen suggerieren, wie 
% %        ähnlich bzw. unterschiedlich sich die betroffenen Tweets sind (zumindest grob).
% \end{itemize}
% \end{frame}


\begin{frame}{Module - Clustering - Grundidee}
%Beispiel:\\
\begin{tikzpicture}[y=\textwidth/100,x=\textwidth/44, background rectangle/.style={draw=black, thick, fill=yellow!10,},show background rectangle]
\def\marRad{0.5mm}
\definecolor{color0}{rgb}{0.73,0.10,0.41}
\definecolor{color1}{rgb}{0.41,0.21,0.04}
\definecolor{color2}{rgb}{0.33,0.66,0.97}
\definecolor{color3}{rgb}{0.71,0.00,0.15}
\definecolor{color4}{rgb}{0.96,0.16,0.94}
\definecolor{color5}{rgb}{0.55,0.95,0.91}
\definecolor{color6}{rgb}{0.94,0.49,0.40}

\path[fill=color0,draw=color0,mark size=\marRad, mark=*] plot coordinates {(11.03, 13.77)};
\path[fill=color0,draw=color0,mark size=\marRad, mark=*] plot coordinates {(10.76, 8.68)};
\path[fill=color0,draw=color0,mark size=\marRad, mark=*] plot coordinates {(10.76, 7.31)};
\path[fill=color0,draw=color0,mark size=\marRad, mark=*] plot coordinates {(14.14, 10.17)};
\path[fill=color0,draw=color0,mark size=\marRad, mark=*] plot coordinates {(10.55, 10.86)};
\path[fill=color0,draw=color0,mark size=\marRad, mark=*] plot coordinates {(10.40, 10.60)};
\path[fill=color0,draw=color0,mark size=\marRad, mark=*] plot coordinates {(10.28, 10.79)};
\path[fill=color0,draw=color0,mark size=\marRad, mark=*] plot coordinates {(9.16, 12.33)};
\path[fill=color0,draw=color0,mark size=\marRad, mark=*] plot coordinates {(10.02, 10.47)};
\path[fill=color0,draw=color0,mark size=\marRad, mark=*] plot coordinates {(10.01, 10.02)};
\path[fill=color0,draw=color0,mark size=\marRad, mark=*] plot coordinates {(12.57, 11.35)};
\path[fill=color0,draw=color0,mark size=\marRad, mark=*] plot coordinates {(11.40, 14.80)};
\path[fill=color0,draw=color0,mark size=\marRad, mark=*] plot coordinates {(11.14, 10.08)};
\path[fill=color0,draw=color0,mark size=\marRad, mark=*] plot coordinates {(8.46, 7.54)};
\path[fill=color0,draw=color0,mark size=\marRad, mark=*] plot coordinates {(6.78, 10.53)};
\path[fill=color0,draw=color0,mark size=\marRad, mark=*] plot coordinates {(10.01, 10.41)};
\path[fill=color0,draw=color0,mark size=\marRad, mark=*] plot coordinates {(11.13, 9.48)};
\path[fill=color0,draw=color0,mark size=\marRad, mark=*] plot coordinates {(9.65, 6.69)};
\path[fill=color0,draw=color0,mark size=\marRad, mark=*] plot coordinates {(11.88, 11.76)};
\path[fill=color0,draw=color0,mark size=\marRad, mark=*] plot coordinates {(7.51, 6.45)};
\path[fill=color0,draw=color0,mark size=\marRad, mark=*] plot coordinates {(12.79, 5.19)};
\path[fill=color0,draw=color0,mark size=\marRad, mark=*] plot coordinates {(11.52, 9.43)};
\path[fill=color0,draw=color0,mark size=\marRad, mark=*] plot coordinates {(10.00, 10.00)};
\path[fill=color0,draw=color0,mark size=\marRad, mark=*] plot coordinates {(10.19, 9.88)};
\path[fill=color0,draw=color0,mark size=\marRad, mark=*] plot coordinates {(5.49, 12.16)};
\path[fill=color0,draw=color0,mark size=2*\marRad, mark=*] plot coordinates {(12.58, 3.24)};
\path[fill=color0,draw=color0,mark size=\marRad, mark=*] plot coordinates {(11.15, 10.37)};
\path[fill=color0,draw=color0,mark size=\marRad, mark=*] plot coordinates {(11.28, 10.47)};
\path[fill=color0,draw=color0,mark size=\marRad, mark=*] plot coordinates {(9.68, 9.02)};
\path[fill=color0,draw=color0,mark size=\marRad, mark=*] plot coordinates {(10.95, 13.26)};
\path[fill=color0,draw=color0,mark size=\marRad, mark=*] plot coordinates {(9.72, 8.21)};
\path[fill=color0,draw=color0,mark size=\marRad, mark=*] plot coordinates {(12.04, 9.78)};
\path[fill=color0,draw=color0,mark size=\marRad, mark=*] plot coordinates {(10.55, 10.90)};
\path[fill=color0,draw=color0,mark size=\marRad, mark=*] plot coordinates {(9.24, 9.11)};
\path[fill=color0,draw=color0,mark size=\marRad, mark=*] plot coordinates {(10.17, 10.60)};
\path[fill=color0,draw=color0,mark size=\marRad, mark=*] plot coordinates {(9.86, 9.20)};
\path[fill=color0,draw=color0,mark size=\marRad, mark=*] plot coordinates {(10.01, 10.01)};
\path[fill=color0,draw=color0,mark size=\marRad, mark=*] plot coordinates {(9.80, 9.22)};
\path[fill=color0,draw=color0,mark size=\marRad, mark=*] plot coordinates {(9.80, 9.82)};
\path[fill=color0,draw=color0,mark size=\marRad, mark=*] plot coordinates {(10.31, 9.96)};
\path[fill=color0,draw=color0,mark size=\marRad, mark=*] plot coordinates {(7.99, 9.64)};
\path[fill=color0,draw=color0,mark size=\marRad, mark=*] plot coordinates {(9.50, 9.67)};
\path[fill=color0,draw=color0,mark size=\marRad, mark=*] plot coordinates {(8.11, 8.55)};
\path[fill=color0,draw=color0,mark size=\marRad, mark=*] plot coordinates {(8.30, 12.36)};
\path[fill=color0,draw=color0,mark size=\marRad, mark=*] plot coordinates {(11.11, 5.92)};
\path[fill=color0,draw=color0,mark size=\marRad, mark=*] plot coordinates {(12.04, 8.00)};
\path[fill=color0,draw=color0,mark size=\marRad, mark=*] plot coordinates {(7.23, 12.75)};
\path[fill=color0,draw=color0,mark size=\marRad, mark=*] plot coordinates {(10.16, 8.78)};
\path[fill=color0,draw=color0,mark size=\marRad, mark=*] plot coordinates {(10.80, 10.23)};
\path[fill=color0,draw=color0,mark size=\marRad, mark=*] plot coordinates {(8.65, 10.37)};
\path[fill=color0,draw=color0,mark size=\marRad, mark=*] plot coordinates {(11.50, 10.68)};
\path[fill=color0,draw=color0,mark size=\marRad, mark=*] plot coordinates {(9.83, 9.72)};
\path[fill=color0,draw=color0,mark size=\marRad, mark=*] plot coordinates {(10.01, 10.86)};
\path[fill=color0,draw=color0,mark size=\marRad, mark=*] plot coordinates {(10.14, 10.52)};
\path[fill=color0,draw=color0,mark size=\marRad, mark=*] plot coordinates {(11.62, 6.29)};
\path[fill=color0,draw=color0,mark size=\marRad, mark=*] plot coordinates {(12.65, 10.41)};
\path[fill=color0,draw=color0,mark size=\marRad, mark=*] plot coordinates {(11.93, 9.55)};
\path[fill=color0,draw=color0,mark size=\marRad, mark=*] plot coordinates {(7.67, 5.26)};
\path[fill=color0,draw=color0,mark size=\marRad, mark=*] plot coordinates {(9.90, 7.58)};
\path[fill=color0,draw=color0,mark size=\marRad, mark=*] plot coordinates {(12.67, 8.72)};
\path[fill=color0,draw=color0,mark size=\marRad, mark=*] plot coordinates {(12.11, 11.76)};
\path[fill=color0,draw=color0,mark size=\marRad, mark=*] plot coordinates {(10.68, 5.64)};
\path[fill=color0,draw=color0,mark size=\marRad, mark=*] plot coordinates {(10.61, 10.66)};
\path[fill=color0,draw=color0,mark size=\marRad, mark=*] plot coordinates {(11.60, 9.23)};
\path[fill=color0,draw=color0,mark size=\marRad, mark=*] plot coordinates {(8.99, 8.45)};
\path[fill=color0,draw=color0,mark size=\marRad, mark=*] plot coordinates {(11.50, 10.41)};
\path[fill=color0,draw=color0,mark size=\marRad, mark=*] plot coordinates {(10.31, 11.31)};
\path[fill=color0,draw=color0,mark size=\marRad, mark=*] plot coordinates {(12.44, 12.02)};
\path[fill=color0,draw=color0,mark size=\marRad, mark=*] plot coordinates {(12.77, 8.62)};
\path[fill=color0,draw=color0,mark size=\marRad, mark=*] plot coordinates {(7.71, 11.65)};
\path[fill=color0,draw=color0,mark size=\marRad, mark=*] plot coordinates {(10.29, 6.66)};
\path[fill=color0,draw=color0,mark size=\marRad, mark=*] plot coordinates {(10.82, 8.34)};
\path[fill=color0,draw=color0,mark size=\marRad, mark=*] plot coordinates {(13.01, 13.26)};
\path[fill=color0,draw=color0,mark size=\marRad, mark=*] plot coordinates {(2.71, 8.52)};
\path[fill=color0,draw=color0,mark size=\marRad, mark=*] plot coordinates {(10.26, 11.31)};
\path[fill=color0,draw=color0,mark size=\marRad, mark=*] plot coordinates {(10.80, 5.47)};
\path[fill=color0,draw=color0,mark size=\marRad, mark=*] plot coordinates {(8.26, 10.37)};
\path[fill=color0,draw=color0,mark size=\marRad, mark=*] plot coordinates {(12.36, 9.95)};
\path[fill=color0,draw=color0,mark size=\marRad, mark=*] plot coordinates {(7.35, 7.17)};
\path[fill=color0,draw=color0,mark size=\marRad, mark=*] plot coordinates {(9.92, 9.19)};
\path[fill=color0,draw=color0,mark size=\marRad, mark=*] plot coordinates {(5.60, 8.39)};
\path[fill=color0,draw=color0,mark size=\marRad, mark=*] plot coordinates {(10.00, 12.17)};
\path[fill=color0,draw=color0,mark size=\marRad, mark=*] plot coordinates {(10.16, 9.04)};
\path[fill=color0,draw=color0,mark size=\marRad, mark=*] plot coordinates {(10.03, 10.52)};
\path[fill=color0,draw=color0,mark size=\marRad, mark=*] plot coordinates {(8.47, 12.80)};
\path[fill=color0,draw=color0,mark size=\marRad, mark=*] plot coordinates {(10.44, 10.32)};
\path[fill=color0,draw=color0,mark size=\marRad, mark=*] plot coordinates {(12.25, 10.98)};
\path[fill=color0,draw=color0,mark size=\marRad, mark=*] plot coordinates {(6.76, 11.33)};
\path[fill=color0,draw=color0,mark size=\marRad, mark=*] plot coordinates {(10.63, 8.22)};
\path[fill=color0,draw=color0,mark size=\marRad, mark=*] plot coordinates {(11.52, 8.42)};
\path[fill=color0,draw=color0,mark size=\marRad, mark=*] plot coordinates {(9.29, 10.60)};
\path[fill=color0,draw=color0,mark size=\marRad, mark=*] plot coordinates {(9.92, 10.79)};
\path[fill=color0,draw=color0,mark size=\marRad, mark=*] plot coordinates {(9.70, 9.85)};
\path[fill=color0,draw=color0,mark size=\marRad, mark=*] plot coordinates {(9.91, 9.92)};
\path[fill=color0,draw=color0,mark size=\marRad, mark=*] plot coordinates {(11.80, 11.60)};
\path[fill=color0,draw=color0,mark size=\marRad, mark=*] plot coordinates {(10.93, 12.40)};
\path[fill=color0,draw=color0,mark size=\marRad, mark=*] plot coordinates {(9.17, 8.55)};
\path[fill=color0,draw=color0,mark size=\marRad, mark=*] plot coordinates {(12.50, 11.69)};
\path[fill=color0,draw=color0,mark size=\marRad, mark=*] plot coordinates {(7.99, 12.80)};
\path[fill=color0,draw=color0,mark size=\marRad, mark=*] plot coordinates {(10.16, 10.05)};
\path[fill=color0,draw=color0,mark size=\marRad, mark=*] plot coordinates {(7.97, 8.82)};
\path[fill=color0,draw=color0,mark size=\marRad, mark=*] plot coordinates {(7.26, 12.93)};
\path[fill=color0,draw=color0,mark size=\marRad, mark=*] plot coordinates {(10.13, 10.84)};
\path[fill=color0,draw=color0,mark size=\marRad, mark=*] plot coordinates {(9.67, 10.12)};
\path[fill=color0,draw=color0,mark size=\marRad, mark=*] plot coordinates {(11.11, 7.46)};
\path[fill=color0,draw=color0,mark size=\marRad, mark=*] plot coordinates {(8.20, 7.34)};
\path[fill=color0,draw=color0,mark size=\marRad, mark=*] plot coordinates {(11.18, 10.38)};
\path[fill=color0,draw=color0,mark size=\marRad, mark=*] plot coordinates {(14.76, 10.67)};
\path[fill=color0,draw=color0,mark size=\marRad, mark=*] plot coordinates {(7.77, 8.49)};
\path[fill=color0,draw=color0,mark size=\marRad, mark=*] plot coordinates {(12.77, 9.27)};
\path[fill=color0,draw=color0,mark size=\marRad, mark=*] plot coordinates {(9.69, 10.06)};
\path[fill=color0,draw=color0,mark size=\marRad, mark=*] plot coordinates {(8.05, 12.68)};
\path[fill=color0,draw=color0,mark size=\marRad, mark=*] plot coordinates {(10.11, 9.76)};
\path[fill=color0,draw=color0,mark size=\marRad, mark=*] plot coordinates {(10.00, 10.35)};
\path[fill=color0,draw=color0,mark size=\marRad, mark=*] plot coordinates {(9.93, 10.65)};
\path[fill=color0,draw=color0,mark size=\marRad, mark=*] plot coordinates {(9.46, 7.94)};
\path[fill=color0,draw=color0,mark size=\marRad, mark=*] plot coordinates {(13.07, 8.75)};
\path[fill=color0,draw=color0,mark size=\marRad, mark=*] plot coordinates {(11.52, 11.02)};
\path[fill=color0,draw=color0,mark size=\marRad, mark=*] plot coordinates {(10.04, 8.83)};
\path[fill=color0,draw=color0,mark size=\marRad, mark=*] plot coordinates {(9.13, 8.94)};
\path[fill=color0,draw=color0,mark size=\marRad, mark=*] plot coordinates {(8.46, 10.38)};
\path[fill=color0,draw=color0,mark size=\marRad, mark=*] plot coordinates {(11.62, 11.51)};
\path[fill=color0,draw=color0,mark size=\marRad, mark=*] plot coordinates {(10.17, 10.06)};
\path[fill=color0,draw=color0,mark size=\marRad, mark=*] plot coordinates {(6.06, 8.44)};
\path[fill=color0,draw=color0,mark size=\marRad, mark=*] plot coordinates {(9.45, 10.51)};
\path[fill=color0,draw=color0,mark size=\marRad, mark=*] plot coordinates {(12.02, 10.81)};
\path[fill=color0,draw=color0,mark size=\marRad, mark=*] plot coordinates {(10.81, 7.08)};
\path[fill=color0,draw=color0,mark size=\marRad, mark=*] plot coordinates {(11.19, 8.95)};
\path[fill=color0,draw=color0,mark size=\marRad, mark=*] plot coordinates {(10.41, 10.92)};
\path[fill=color0,draw=color0,mark size=\marRad, mark=*] plot coordinates {(7.36, 13.76)};
\path[fill=color0,draw=color0,mark size=\marRad, mark=*] plot coordinates {(7.73, 10.49)};
\path[fill=color0,draw=color0,mark size=\marRad, mark=*] plot coordinates {(10.54, 8.46)};
\path[fill=color0,draw=color0,mark size=\marRad, mark=*] plot coordinates {(7.95, 10.51)};
\path[fill=color0,draw=color0,mark size=\marRad, mark=*] plot coordinates {(9.99, 9.90)};
\path[fill=color0,draw=color0,mark size=\marRad, mark=*] plot coordinates {(3.57, 7.18)};
\path[fill=color0,draw=color0,mark size=\marRad, mark=*] plot coordinates {(7.80, 8.21)};
\path[fill=color0,draw=color0,mark size=\marRad, mark=*] plot coordinates {(11.94, 12.06)};
\path[fill=color0,draw=color0,mark size=\marRad, mark=*] plot coordinates {(8.89, 10.89)};
\path[fill=color0,draw=color0,mark size=\marRad, mark=*] plot coordinates {(10.23, 10.42)};
\path[fill=color0,draw=color0,mark size=\marRad, mark=*] plot coordinates {(12.61, 10.29)};
\path[fill=color0,draw=color0,mark size=\marRad, mark=*] plot coordinates {(9.80, 10.18)};
\path[fill=color0,draw=color0,mark size=\marRad, mark=*] plot coordinates {(10.97, 11.83)};
\path[fill=color0,draw=color0,mark size=\marRad, mark=*] plot coordinates {(8.04, 11.96)};
\path[fill=color0,draw=color0,mark size=\marRad, mark=*] plot coordinates {(10.14, 9.91)};
\path[fill=color0,draw=color0,mark size=\marRad, mark=*] plot coordinates {(10.65, 11.04)};
\path[fill=color0,draw=color0,mark size=\marRad, mark=*] plot coordinates {(10.12, 15.00)};
\path[fill=color0,draw=color0,mark size=\marRad, mark=*] plot coordinates {(11.77, 10.99)};
\path[fill=color0,draw=color0,mark size=\marRad, mark=*] plot coordinates {(9.49, 13.77)};
\path[fill=color0,draw=color0,mark size=\marRad, mark=*] plot coordinates {(12.78, 6.93)};
\path[fill=color0,draw=color0,mark size=\marRad, mark=*] plot coordinates {(7.95, 7.76)};
\path[fill=color0,draw=color0,mark size=\marRad, mark=*] plot coordinates {(8.76, 8.44)};
\path[fill=color0,draw=color0,mark size=\marRad, mark=*] plot coordinates {(10.74, 10.34)};
\path[fill=color0,draw=color0,mark size=\marRad, mark=*] plot coordinates {(5.09, 9.05)};
\path[fill=color0,draw=color0,mark size=\marRad, mark=*] plot coordinates {(12.34, 8.57)};
\path[fill=color0,draw=color0,mark size=\marRad, mark=*] plot coordinates {(9.20, 10.42)};
\path[fill=color0,draw=color0,mark size=\marRad, mark=*] plot coordinates {(9.07, 10.02)};
\path[fill=color0,draw=color0,mark size=\marRad, mark=*] plot coordinates {(8.76, 9.30)};
\path[fill=color0,draw=color0,mark size=\marRad, mark=*] plot coordinates {(8.18, 10.62)};
\path[fill=color0,draw=color0,mark size=\marRad, mark=*] plot coordinates {(12.94, 15.26)};
\path[fill=color0,draw=color0,mark size=\marRad, mark=*] plot coordinates {(10.91, 7.79)};
\path[fill=color0,draw=color0,mark size=\marRad, mark=*] plot coordinates {(9.74, 8.57)};
\path[fill=color0,draw=color0,mark size=\marRad, mark=*] plot coordinates {(9.18, 9.89)};
\path[fill=color0,draw=color0,mark size=\marRad, mark=*] plot coordinates {(8.88, 9.76)};
\path[fill=color0,draw=color0,mark size=\marRad, mark=*] plot coordinates {(9.82, 11.65)};
\path[fill=color0,draw=color0,mark size=\marRad, mark=*] plot coordinates {(12.46, 11.08)};
\path[fill=color0,draw=color0,mark size=\marRad, mark=*] plot coordinates {(7.90, 9.75)};
\path[fill=color0,draw=color0,mark size=\marRad, mark=*] plot coordinates {(9.39, 6.64)};
\path[fill=color0,draw=color0,mark size=\marRad, mark=*] plot coordinates {(9.68, 9.94)};
\path[fill=color0,draw=color0,mark size=\marRad, mark=*] plot coordinates {(8.64, 11.70)};
\path[fill=color0,draw=color0,mark size=\marRad, mark=*] plot coordinates {(13.24, 8.50)};
\path[fill=color0,draw=color0,mark size=\marRad, mark=*] plot coordinates {(6.40, 10.78)};
\path[fill=color0,draw=color0,mark size=\marRad, mark=*] plot coordinates {(10.68, 8.50)};
\path[fill=color0,draw=color0,mark size=\marRad, mark=*] plot coordinates {(10.01, 10.02)};
\path[fill=color0,draw=color0,mark size=\marRad, mark=*] plot coordinates {(10.15, 8.76)};
\path[fill=color0,draw=color0,mark size=\marRad, mark=*] plot coordinates {(8.43, 11.54)};
\path[fill=color0,draw=color0,mark size=\marRad, mark=*] plot coordinates {(9.93, 10.43)};
\path[fill=color0,draw=color0,mark size=\marRad, mark=*] plot coordinates {(10.01, 10.00)};
\path[fill=color0,draw=color0,mark size=\marRad, mark=*] plot coordinates {(9.59, 4.88)};
\path[fill=color0,draw=color0,mark size=\marRad, mark=*] plot coordinates {(10.31, 9.16)};
\path[fill=color0,draw=color0,mark size=\marRad, mark=*] plot coordinates {(11.83, 11.48)};
\path[fill=color0,draw=color0,mark size=\marRad, mark=*] plot coordinates {(6.51, 8.06)};
\path[fill=color0,draw=color0,mark size=\marRad, mark=*] plot coordinates {(10.41, 10.10)};
\path[fill=color0,draw=color0,mark size=\marRad, mark=*] plot coordinates {(10.75, 10.50)};
\path[fill=color0,draw=color0,mark size=\marRad, mark=*] plot coordinates {(9.05, 9.59)};
\path[fill=color0,draw=color0,mark size=\marRad, mark=*] plot coordinates {(6.64, 14.04)};
\path[fill=color0,draw=color0,mark size=\marRad, mark=*] plot coordinates {(10.66, 10.08)};
\path[fill=color0,draw=color0,mark size=\marRad, mark=*] plot coordinates {(9.38, 8.28)};
\path[fill=color0,draw=color0,mark size=\marRad, mark=*] plot coordinates {(8.08, 8.74)};
\path[fill=color0,draw=color0,mark size=\marRad, mark=*] plot coordinates {(12.77, 9.77)};
\path[fill=color0,draw=color0,mark size=\marRad, mark=*] plot coordinates {(10.80, 9.54)};
\path[fill=color0,draw=color0,mark size=\marRad, mark=*] plot coordinates {(9.82, 10.37)};
\path[fill=color0,draw=color0,mark size=\marRad, mark=*] plot coordinates {(6.11, 9.61)};
\path[fill=color0,draw=color0,mark size=\marRad, mark=*] plot coordinates {(7.15, 9.67)};
\path[fill=color0,draw=color0,mark size=\marRad, mark=*] plot coordinates {(9.56, 10.12)};
\path[fill=color0,draw=color0,mark size=\marRad, mark=*] plot coordinates {(7.95, 10.11)};
\path[fill=color0,draw=color0,mark size=\marRad, mark=*] plot coordinates {(10.36, 9.52)};
\path[fill=color0,draw=color0,mark size=\marRad, mark=*] plot coordinates {(13.15, 11.54)};
\path[fill=color0,draw=color0,mark size=\marRad, mark=*] plot coordinates {(10.06, 9.72)};
\path[fill=color0,draw=color0,mark size=\marRad, mark=*] plot coordinates {(9.53, 10.75)};
\path[fill=color0,draw=color0,mark size=\marRad, mark=*] plot coordinates {(6.52, 9.99)};
\path[fill=color1,draw=color1,mark size=\marRad, mark=square*] plot coordinates {(31.03, 13.77)};
\path[fill=color1,draw=color1,mark size=\marRad, mark=square*] plot coordinates {(30.76, 8.68)};
\path[fill=color1,draw=color1,mark size=\marRad, mark=square*] plot coordinates {(30.76, 7.31)};
\path[fill=color1,draw=color1,mark size=\marRad, mark=square*] plot coordinates {(34.14, 10.17)};
\path[fill=color1,draw=color1,mark size=\marRad, mark=square*] plot coordinates {(30.55, 10.86)};
\path[fill=color1,draw=color1,mark size=\marRad, mark=square*] plot coordinates {(30.40, 10.60)};
\path[fill=color1,draw=color1,mark size=\marRad, mark=square*] plot coordinates {(30.28, 10.79)};
\path[fill=color1,draw=color1,mark size=\marRad, mark=square*] plot coordinates {(29.16, 12.33)};
\path[fill=color1,draw=color1,mark size=\marRad, mark=square*] plot coordinates {(30.02, 10.47)};
\path[fill=color1,draw=color1,mark size=\marRad, mark=square*] plot coordinates {(30.01, 10.02)};
\path[fill=color1,draw=color1,mark size=\marRad, mark=square*] plot coordinates {(32.57, 11.35)};
\path[fill=color1,draw=color1,mark size=\marRad, mark=square*] plot coordinates {(31.40, 14.80)};
\path[fill=color1,draw=color1,mark size=\marRad, mark=square*] plot coordinates {(31.14, 10.08)};
\path[fill=color1,draw=color1,mark size=\marRad, mark=square*] plot coordinates {(28.46, 7.54)};
\path[fill=color1,draw=color1,mark size=\marRad, mark=square*] plot coordinates {(26.78, 10.53)};
\path[fill=color1,draw=color1,mark size=\marRad, mark=square*] plot coordinates {(30.01, 10.41)};
\path[fill=color1,draw=color1,mark size=\marRad, mark=square*] plot coordinates {(31.13, 9.48)};
\path[fill=color1,draw=color1,mark size=\marRad, mark=square*] plot coordinates {(29.65, 6.69)};
\path[fill=color1,draw=color1,mark size=\marRad, mark=square*] plot coordinates {(31.88, 11.76)};
\path[fill=color1,draw=color1,mark size=\marRad, mark=square*] plot coordinates {(27.51, 6.45)};
\path[fill=color1,draw=color1,mark size=\marRad, mark=square*] plot coordinates {(32.79, 5.19)};
\path[fill=color1,draw=color1,mark size=\marRad, mark=square*] plot coordinates {(31.52, 9.43)};
\path[fill=color1,draw=color1,mark size=\marRad, mark=square*] plot coordinates {(30.00, 10.00)};
\path[fill=color1,draw=color1,mark size=\marRad, mark=square*] plot coordinates {(30.19, 9.88)};
\path[fill=color1,draw=color1,mark size=\marRad, mark=square*] plot coordinates {(25.49, 12.16)};
\path[fill=color1,draw=color1,mark size=\marRad, mark=square*] plot coordinates {(32.58, 3.24)};
\path[fill=color1,draw=color1,mark size=\marRad, mark=square*] plot coordinates {(31.15, 10.37)};
\path[fill=color1,draw=color1,mark size=\marRad, mark=square*] plot coordinates {(31.28, 10.47)};
\path[fill=color1,draw=color1,mark size=\marRad, mark=square*] plot coordinates {(29.68, 9.02)};
\path[fill=color1,draw=color1,mark size=\marRad, mark=square*] plot coordinates {(30.95, 13.26)};
\path[fill=color1,draw=color1,mark size=\marRad, mark=square*] plot coordinates {(29.72, 8.21)};
\path[fill=color1,draw=color1,mark size=\marRad, mark=square*] plot coordinates {(32.04, 9.78)};
\path[fill=color1,draw=color1,mark size=\marRad, mark=square*] plot coordinates {(30.55, 10.90)};
\path[fill=color1,draw=color1,mark size=\marRad, mark=square*] plot coordinates {(29.24, 9.11)};
\path[fill=color1,draw=color1,mark size=\marRad, mark=square*] plot coordinates {(30.17, 10.60)};
\path[fill=color1,draw=color1,mark size=\marRad, mark=square*] plot coordinates {(29.86, 9.20)};
\path[fill=color1,draw=color1,mark size=\marRad, mark=square*] plot coordinates {(30.01, 10.01)};
\path[fill=color1,draw=color1,mark size=\marRad, mark=square*] plot coordinates {(29.80, 9.22)};
\path[fill=color1,draw=color1,mark size=\marRad, mark=square*] plot coordinates {(29.80, 9.82)};
\path[fill=color1,draw=color1,mark size=\marRad, mark=square*] plot coordinates {(30.31, 9.96)};
\path[fill=color1,draw=color1,mark size=\marRad, mark=square*] plot coordinates {(27.99, 9.64)};
\path[fill=color1,draw=color1,mark size=\marRad, mark=square*] plot coordinates {(29.50, 9.67)};
\path[fill=color1,draw=color1,mark size=\marRad, mark=square*] plot coordinates {(28.11, 8.55)};
\path[fill=color1,draw=color1,mark size=\marRad, mark=square*] plot coordinates {(28.30, 12.36)};
\path[fill=color1,draw=color1,mark size=\marRad, mark=square*] plot coordinates {(31.11, 5.92)};
\path[fill=color1,draw=color1,mark size=\marRad, mark=square*] plot coordinates {(32.04, 8.00)};
\path[fill=color1,draw=color1,mark size=\marRad, mark=square*] plot coordinates {(27.23, 12.75)};
\path[fill=color1,draw=color1,mark size=\marRad, mark=square*] plot coordinates {(30.16, 8.78)};
\path[fill=color1,draw=color1,mark size=\marRad, mark=square*] plot coordinates {(30.80, 10.23)};
\path[fill=color1,draw=color1,mark size=\marRad, mark=square*] plot coordinates {(28.65, 10.37)};
\path[fill=color1,draw=color1,mark size=\marRad, mark=square*] plot coordinates {(31.50, 10.68)};
\path[fill=color1,draw=color1,mark size=\marRad, mark=square*] plot coordinates {(29.83, 9.72)};
\path[fill=color1,draw=color1,mark size=\marRad, mark=square*] plot coordinates {(30.01, 10.86)};
\path[fill=color1,draw=color1,mark size=\marRad, mark=square*] plot coordinates {(30.14, 10.52)};
\path[fill=color1,draw=color1,mark size=\marRad, mark=square*] plot coordinates {(31.62, 6.29)};
\path[fill=color1,draw=color1,mark size=\marRad, mark=square*] plot coordinates {(32.65, 10.41)};
\path[fill=color1,draw=color1,mark size=\marRad, mark=square*] plot coordinates {(31.93, 9.55)};
\path[fill=color1,draw=color1,mark size=\marRad, mark=square*] plot coordinates {(27.67, 5.26)};
\path[fill=color1,draw=color1,mark size=\marRad, mark=square*] plot coordinates {(29.90, 7.58)};
\path[fill=color1,draw=color1,mark size=\marRad, mark=square*] plot coordinates {(32.67, 8.72)};
\path[fill=color1,draw=color1,mark size=\marRad, mark=square*] plot coordinates {(32.11, 11.76)};
\path[fill=color1,draw=color1,mark size=\marRad, mark=square*] plot coordinates {(30.68, 5.64)};
\path[fill=color1,draw=color1,mark size=\marRad, mark=square*] plot coordinates {(30.61, 10.66)};
\path[fill=color1,draw=color1,mark size=\marRad, mark=square*] plot coordinates {(31.60, 9.23)};
\path[fill=color1,draw=color1,mark size=\marRad, mark=square*] plot coordinates {(28.99, 8.45)};
\path[fill=color1,draw=color1,mark size=\marRad, mark=square*] plot coordinates {(31.50, 10.41)};
\path[fill=color1,draw=color1,mark size=\marRad, mark=square*] plot coordinates {(30.31, 11.31)};
\path[fill=color1,draw=color1,mark size=\marRad, mark=square*] plot coordinates {(32.44, 12.02)};
\path[fill=color1,draw=color1,mark size=\marRad, mark=square*] plot coordinates {(32.77, 8.62)};
\path[fill=color1,draw=color1,mark size=\marRad, mark=square*] plot coordinates {(27.71, 11.65)};
\path[fill=color1,draw=color1,mark size=\marRad, mark=square*] plot coordinates {(30.29, 6.66)};
\path[fill=color1,draw=color1,mark size=\marRad, mark=square*] plot coordinates {(30.82, 8.34)};
\path[fill=color1,draw=color1,mark size=\marRad, mark=square*] plot coordinates {(33.01, 13.26)};
\path[fill=color1,draw=color1,mark size=\marRad, mark=square*] plot coordinates {(22.71, 8.52)};
\path[fill=color1,draw=color1,mark size=\marRad, mark=square*] plot coordinates {(30.26, 11.31)};
\path[fill=color1,draw=color1,mark size=\marRad, mark=square*] plot coordinates {(30.80, 5.47)};
\path[fill=color1,draw=color1,mark size=\marRad, mark=square*] plot coordinates {(28.26, 10.37)};
\path[fill=color1,draw=color1,mark size=\marRad, mark=square*] plot coordinates {(32.36, 9.95)};
\path[fill=color1,draw=color1,mark size=\marRad, mark=square*] plot coordinates {(27.35, 7.17)};
\path[fill=color1,draw=color1,mark size=\marRad, mark=square*] plot coordinates {(29.92, 9.19)};
\path[fill=color1,draw=color1,mark size=\marRad, mark=square*] plot coordinates {(25.60, 8.39)};
\path[fill=color1,draw=color1,mark size=\marRad, mark=square*] plot coordinates {(30.00, 12.17)};
\path[fill=color1,draw=color1,mark size=\marRad, mark=square*] plot coordinates {(30.16, 9.04)};
\path[fill=color1,draw=color1,mark size=\marRad, mark=square*] plot coordinates {(30.03, 10.52)};
\path[fill=color1,draw=color1,mark size=\marRad, mark=square*] plot coordinates {(28.47, 12.80)};
\path[fill=color1,draw=color1,mark size=\marRad, mark=square*] plot coordinates {(30.44, 10.32)};
\path[fill=color1,draw=color1,mark size=\marRad, mark=square*] plot coordinates {(32.25, 10.98)};
\path[fill=color1,draw=color1,mark size=\marRad, mark=square*] plot coordinates {(26.76, 11.33)};
\path[fill=color1,draw=color1,mark size=\marRad, mark=square*] plot coordinates {(30.63, 8.22)};
\path[fill=color1,draw=color1,mark size=\marRad, mark=square*] plot coordinates {(31.52, 8.42)};
\path[fill=color1,draw=color1,mark size=\marRad, mark=square*] plot coordinates {(29.29, 10.60)};
\path[fill=color1,draw=color1,mark size=\marRad, mark=square*] plot coordinates {(29.92, 10.79)};
\path[fill=color1,draw=color1,mark size=\marRad, mark=square*] plot coordinates {(29.70, 9.85)};
\path[fill=color1,draw=color1,mark size=\marRad, mark=square*] plot coordinates {(29.91, 9.92)};
\path[fill=color1,draw=color1,mark size=\marRad, mark=square*] plot coordinates {(31.80, 11.60)};
\path[fill=color1,draw=color1,mark size=\marRad, mark=square*] plot coordinates {(30.93, 12.40)};
\path[fill=color1,draw=color1,mark size=\marRad, mark=square*] plot coordinates {(29.17, 8.55)};
\path[fill=color1,draw=color1,mark size=\marRad, mark=square*] plot coordinates {(32.50, 11.69)};
\path[fill=color1,draw=color1,mark size=\marRad, mark=square*] plot coordinates {(27.99, 12.80)};
\path[fill=color1,draw=color1,mark size=\marRad, mark=square*] plot coordinates {(30.16, 10.05)};
\path[fill=color1,draw=color1,mark size=\marRad, mark=square*] plot coordinates {(27.97, 8.82)};
\path[fill=color1,draw=color1,mark size=\marRad, mark=square*] plot coordinates {(27.26, 12.93)};
\path[fill=color1,draw=color1,mark size=\marRad, mark=square*] plot coordinates {(30.13, 10.84)};
\path[fill=color1,draw=color1,mark size=\marRad, mark=square*] plot coordinates {(29.67, 10.12)};
\path[fill=color1,draw=color1,mark size=\marRad, mark=square*] plot coordinates {(31.11, 7.46)};
\path[fill=color1,draw=color1,mark size=\marRad, mark=square*] plot coordinates {(28.20, 7.34)};
\path[fill=color1,draw=color1,mark size=\marRad, mark=square*] plot coordinates {(31.18, 10.38)};
\path[fill=color1,draw=color1,mark size=\marRad, mark=square*] plot coordinates {(34.76, 10.67)};
\path[fill=color1,draw=color1,mark size=\marRad, mark=square*] plot coordinates {(27.77, 8.49)};
\path[fill=color1,draw=color1,mark size=\marRad, mark=square*] plot coordinates {(32.77, 9.27)};
\path[fill=color1,draw=color1,mark size=\marRad, mark=square*] plot coordinates {(29.69, 10.06)};
\path[fill=color1,draw=color1,mark size=\marRad, mark=square*] plot coordinates {(28.05, 12.68)};
\path[fill=color1,draw=color1,mark size=\marRad, mark=square*] plot coordinates {(30.11, 9.76)};
\path[fill=color1,draw=color1,mark size=\marRad, mark=square*] plot coordinates {(30.00, 10.35)};
\path[fill=color1,draw=color1,mark size=\marRad, mark=square*] plot coordinates {(29.93, 10.65)};
\path[fill=color1,draw=color1,mark size=\marRad, mark=square*] plot coordinates {(29.46, 7.94)};
\path[fill=color1,draw=color1,mark size=\marRad, mark=square*] plot coordinates {(33.07, 8.75)};
\path[fill=color1,draw=color1,mark size=\marRad, mark=square*] plot coordinates {(31.52, 11.02)};
\path[fill=color1,draw=color1,mark size=\marRad, mark=square*] plot coordinates {(30.04, 8.83)};
\path[fill=color1,draw=color1,mark size=\marRad, mark=square*] plot coordinates {(29.13, 8.94)};
\path[fill=color1,draw=color1,mark size=\marRad, mark=square*] plot coordinates {(28.46, 10.38)};
\path[fill=color1,draw=color1,mark size=\marRad, mark=square*] plot coordinates {(31.62, 11.51)};
\path[fill=color1,draw=color1,mark size=\marRad, mark=square*] plot coordinates {(30.17, 10.06)};
\path[fill=color1,draw=color1,mark size=\marRad, mark=square*] plot coordinates {(26.06, 8.44)};
\path[fill=color1,draw=color1,mark size=\marRad, mark=square*] plot coordinates {(29.45, 10.51)};
\path[fill=color1,draw=color1,mark size=\marRad, mark=square*] plot coordinates {(32.02, 10.81)};
\path[fill=color1,draw=color1,mark size=\marRad, mark=square*] plot coordinates {(30.81, 7.08)};
\path[fill=color1,draw=color1,mark size=\marRad, mark=square*] plot coordinates {(31.19, 8.95)};
\path[fill=color1,draw=color1,mark size=\marRad, mark=square*] plot coordinates {(30.41, 10.92)};
\path[fill=color1,draw=color1,mark size=\marRad, mark=square*] plot coordinates {(27.36, 13.76)};
\path[fill=color1,draw=color1,mark size=\marRad, mark=square*] plot coordinates {(27.73, 10.49)};
\path[fill=color1,draw=color1,mark size=\marRad, mark=square*] plot coordinates {(30.54, 8.46)};
\path[fill=color1,draw=color1,mark size=\marRad, mark=square*] plot coordinates {(27.95, 10.51)};
\path[fill=color1,draw=color1,mark size=\marRad, mark=square*] plot coordinates {(29.99, 9.90)};
\path[fill=color1,draw=color1,mark size=\marRad, mark=square*] plot coordinates {(23.57, 7.18)};
\path[fill=color1,draw=color1,mark size=\marRad, mark=square*] plot coordinates {(27.80, 8.21)};
\path[fill=color1,draw=color1,mark size=\marRad, mark=square*] plot coordinates {(31.94, 12.06)};
\path[fill=color1,draw=color1,mark size=\marRad, mark=square*] plot coordinates {(28.89, 10.89)};
\path[fill=color1,draw=color1,mark size=\marRad, mark=square*] plot coordinates {(30.23, 10.42)};
\path[fill=color1,draw=color1,mark size=\marRad, mark=square*] plot coordinates {(32.61, 10.29)};
\path[fill=color1,draw=color1,mark size=\marRad, mark=square*] plot coordinates {(29.80, 10.18)};
\path[fill=color1,draw=color1,mark size=\marRad, mark=square*] plot coordinates {(30.97, 11.83)};
\path[fill=color1,draw=color1,mark size=\marRad, mark=square*] plot coordinates {(28.04, 11.96)};
\path[fill=color1,draw=color1,mark size=\marRad, mark=square*] plot coordinates {(30.14, 9.91)};
\path[fill=color1,draw=color1,mark size=\marRad, mark=square*] plot coordinates {(30.65, 11.04)};
\path[fill=color1,draw=color1,mark size=\marRad, mark=square*] plot coordinates {(30.12, 15.00)};
\path[fill=color1,draw=color1,mark size=\marRad, mark=square*] plot coordinates {(31.77, 10.99)};
\path[fill=color1,draw=color1,mark size=\marRad, mark=square*] plot coordinates {(29.49, 13.77)};
\path[fill=color1,draw=color1,mark size=\marRad, mark=square*] plot coordinates {(32.78, 6.93)};
\path[fill=color1,draw=color1,mark size=\marRad, mark=square*] plot coordinates {(27.95, 7.76)};
\path[fill=color1,draw=color1,mark size=\marRad, mark=square*] plot coordinates {(28.76, 8.44)};
\path[fill=color1,draw=color1,mark size=\marRad, mark=square*] plot coordinates {(30.74, 10.34)};
\path[fill=color1,draw=color1,mark size=\marRad, mark=square*] plot coordinates {(25.09, 9.05)};
\path[fill=color1,draw=color1,mark size=\marRad, mark=square*] plot coordinates {(32.34, 8.57)};
\path[fill=color1,draw=color1,mark size=\marRad, mark=square*] plot coordinates {(29.20, 10.42)};
\path[fill=color1,draw=color1,mark size=\marRad, mark=square*] plot coordinates {(29.07, 10.02)};
\path[fill=color1,draw=color1,mark size=\marRad, mark=square*] plot coordinates {(28.76, 9.30)};
\path[fill=color1,draw=color1,mark size=\marRad, mark=square*] plot coordinates {(28.18, 10.62)};
\path[fill=color1,draw=color1,mark size=\marRad, mark=square*] plot coordinates {(32.94, 15.26)};
\path[fill=color1,draw=color1,mark size=\marRad, mark=square*] plot coordinates {(30.91, 7.79)};
\path[fill=color1,draw=color1,mark size=\marRad, mark=square*] plot coordinates {(29.74, 8.57)};
\path[fill=color1,draw=color1,mark size=\marRad, mark=square*] plot coordinates {(29.18, 9.89)};
\path[fill=color1,draw=color1,mark size=\marRad, mark=square*] plot coordinates {(28.88, 9.76)};
\path[fill=color1,draw=color1,mark size=\marRad, mark=square*] plot coordinates {(29.82, 11.65)};
\path[fill=color1,draw=color1,mark size=\marRad, mark=square*] plot coordinates {(32.46, 11.08)};
\path[fill=color1,draw=color1,mark size=\marRad, mark=square*] plot coordinates {(27.90, 9.75)};
\path[fill=color1,draw=color1,mark size=\marRad, mark=square*] plot coordinates {(29.39, 6.64)};
\path[fill=color1,draw=color1,mark size=\marRad, mark=square*] plot coordinates {(29.68, 9.94)};
\path[fill=color1,draw=color1,mark size=\marRad, mark=square*] plot coordinates {(28.64, 11.70)};
\path[fill=color1,draw=color1,mark size=\marRad, mark=square*] plot coordinates {(33.24, 8.50)};
\path[fill=color1,draw=color1,mark size=\marRad, mark=square*] plot coordinates {(26.40, 10.78)};
\path[fill=color1,draw=color1,mark size=\marRad, mark=square*] plot coordinates {(30.68, 8.50)};
\path[fill=color1,draw=color1,mark size=\marRad, mark=square*] plot coordinates {(30.01, 10.02)};
\path[fill=color1,draw=color1,mark size=\marRad, mark=square*] plot coordinates {(30.15, 8.76)};
\path[fill=color1,draw=color1,mark size=\marRad, mark=square*] plot coordinates {(28.43, 11.54)};
\path[fill=color1,draw=color1,mark size=\marRad, mark=square*] plot coordinates {(29.93, 10.43)};
\path[fill=color1,draw=color1,mark size=\marRad, mark=square*] plot coordinates {(30.01, 10.00)};
\path[fill=color1,draw=color1,mark size=\marRad, mark=square*] plot coordinates {(29.59, 4.88)};
\path[fill=color1,draw=color1,mark size=\marRad, mark=square*] plot coordinates {(30.31, 9.16)};
\path[fill=color1,draw=color1,mark size=\marRad, mark=square*] plot coordinates {(31.83, 11.48)};
\path[fill=color1,draw=color1,mark size=\marRad, mark=square*] plot coordinates {(26.51, 8.06)};
\path[fill=color1,draw=color1,mark size=\marRad, mark=square*] plot coordinates {(30.41, 10.10)};
\path[fill=color1,draw=color1,mark size=\marRad, mark=square*] plot coordinates {(30.75, 10.50)};
\path[fill=color1,draw=color1,mark size=\marRad, mark=square*] plot coordinates {(29.05, 9.59)};
\path[fill=color1,draw=color1,mark size=\marRad, mark=square*] plot coordinates {(26.64, 14.04)};
\path[fill=color1,draw=color1,mark size=\marRad, mark=square*] plot coordinates {(30.66, 10.08)};
\path[fill=color1,draw=color1,mark size=\marRad, mark=square*] plot coordinates {(29.38, 8.28)};
\path[fill=color1,draw=color1,mark size=\marRad, mark=square*] plot coordinates {(28.08, 8.74)};
\path[fill=color1,draw=color1,mark size=\marRad, mark=square*] plot coordinates {(32.77, 9.77)};
\path[fill=color1,draw=color1,mark size=\marRad, mark=square*] plot coordinates {(30.80, 9.54)};
\path[fill=color1,draw=color1,mark size=\marRad, mark=square*] plot coordinates {(29.82, 10.37)};
\path[fill=color1,draw=color1,mark size=\marRad, mark=square*] plot coordinates {(26.11, 9.61)};
\path[fill=color1,draw=color1,mark size=\marRad, mark=square*] plot coordinates {(27.15, 9.67)};
\path[fill=color1,draw=color1,mark size=\marRad, mark=square*] plot coordinates {(29.56, 10.12)};
\path[fill=color1,draw=color1,mark size=\marRad, mark=square*] plot coordinates {(27.95, 10.11)};
\path[fill=color1,draw=color1,mark size=\marRad, mark=square*] plot coordinates {(30.36, 9.52)};
\path[fill=color1,draw=color1,mark size=\marRad, mark=square*] plot coordinates {(33.15, 11.54)};
\path[fill=color1,draw=color1,mark size=\marRad, mark=square*] plot coordinates {(30.06, 9.72)};
\path[fill=color1,draw=color1,mark size=\marRad, mark=square*] plot coordinates {(29.53, 10.75)};
\path[fill=color1,draw=color1,mark size=\marRad, mark=square*] plot coordinates {(26.52, 9.99)};
\path[fill=color2,draw=color2,mark size=\marRad, mark=triangle*] plot coordinates {(21.03, 23.77)};
\path[fill=color2,draw=color2,mark size=\marRad, mark=triangle*] plot coordinates {(20.76, 18.68)};
\path[fill=color2,draw=color2,mark size=\marRad, mark=triangle*] plot coordinates {(20.76, 17.31)};
\path[fill=color2,draw=color2,mark size=\marRad, mark=triangle*] plot coordinates {(24.14, 20.17)};
\path[fill=color2,draw=color2,mark size=\marRad, mark=triangle*] plot coordinates {(20.55, 20.86)};
\path[fill=color2,draw=color2,mark size=\marRad, mark=triangle*] plot coordinates {(20.40, 20.60)};
\path[fill=color2,draw=color2,mark size=\marRad, mark=triangle*] plot coordinates {(20.28, 20.79)};
\path[fill=color2,draw=color2,mark size=\marRad, mark=triangle*] plot coordinates {(19.16, 22.33)};
\path[fill=color2,draw=color2,mark size=\marRad, mark=triangle*] plot coordinates {(20.02, 20.47)};
\path[fill=color2,draw=color2,mark size=\marRad, mark=triangle*] plot coordinates {(20.01, 20.02)};
\path[fill=color2,draw=color2,mark size=\marRad, mark=triangle*] plot coordinates {(22.57, 21.35)};
\path[fill=color2,draw=color2,mark size=\marRad, mark=triangle*] plot coordinates {(21.40, 24.80)};
\path[fill=color2,draw=color2,mark size=\marRad, mark=triangle*] plot coordinates {(21.14, 20.08)};
\path[fill=color2,draw=color2,mark size=\marRad, mark=triangle*] plot coordinates {(18.46, 17.54)};
\path[fill=color2,draw=color2,mark size=\marRad, mark=triangle*] plot coordinates {(16.78, 20.53)};
\path[fill=color2,draw=color2,mark size=\marRad, mark=triangle*] plot coordinates {(20.01, 20.41)};
\path[fill=color2,draw=color2,mark size=\marRad, mark=triangle*] plot coordinates {(21.13, 19.48)};
\path[fill=color2,draw=color2,mark size=\marRad, mark=triangle*] plot coordinates {(19.65, 16.69)};
\path[fill=color2,draw=color2,mark size=\marRad, mark=triangle*] plot coordinates {(21.88, 21.76)};
\path[fill=color2,draw=color2,mark size=\marRad, mark=triangle*] plot coordinates {(17.51, 16.45)};
\path[fill=color2,draw=color2,mark size=\marRad, mark=triangle*] plot coordinates {(22.79, 15.19)};
\path[fill=color2,draw=color2,mark size=\marRad, mark=triangle*] plot coordinates {(21.52, 19.43)};
\path[fill=color2,draw=color2,mark size=\marRad, mark=triangle*] plot coordinates {(20.00, 20.00)};
\path[fill=color2,draw=color2,mark size=\marRad, mark=triangle*] plot coordinates {(20.19, 19.88)};
\path[fill=color2,draw=color2,mark size=\marRad, mark=triangle*] plot coordinates {(15.49, 22.16)};
\path[fill=color2,draw=color2,mark size=\marRad, mark=triangle*] plot coordinates {(22.58, 13.24)};
\path[fill=color2,draw=color2,mark size=\marRad, mark=triangle*] plot coordinates {(21.15, 20.37)};
\path[fill=color2,draw=color2,mark size=\marRad, mark=triangle*] plot coordinates {(21.28, 20.47)};
\path[fill=color2,draw=color2,mark size=\marRad, mark=triangle*] plot coordinates {(19.68, 19.02)};
\path[fill=color2,draw=color2,mark size=\marRad, mark=triangle*] plot coordinates {(20.95, 23.26)};
\path[fill=color2,draw=color2,mark size=\marRad, mark=triangle*] plot coordinates {(19.72, 18.21)};
\path[fill=color2,draw=color2,mark size=\marRad, mark=triangle*] plot coordinates {(22.04, 19.78)};
\path[fill=color2,draw=color2,mark size=\marRad, mark=triangle*] plot coordinates {(20.55, 20.90)};
\path[fill=color2,draw=color2,mark size=\marRad, mark=triangle*] plot coordinates {(19.24, 19.11)};
\path[fill=color2,draw=color2,mark size=\marRad, mark=triangle*] plot coordinates {(20.17, 20.60)};
\path[fill=color2,draw=color2,mark size=\marRad, mark=triangle*] plot coordinates {(19.86, 19.20)};
\path[fill=color2,draw=color2,mark size=\marRad, mark=triangle*] plot coordinates {(20.01, 20.01)};
\path[fill=color2,draw=color2,mark size=\marRad, mark=triangle*] plot coordinates {(19.80, 19.22)};
\path[fill=color2,draw=color2,mark size=\marRad, mark=triangle*] plot coordinates {(19.80, 19.82)};
\path[fill=color2,draw=color2,mark size=\marRad, mark=triangle*] plot coordinates {(20.31, 19.96)};
\path[fill=color2,draw=color2,mark size=\marRad, mark=triangle*] plot coordinates {(17.99, 19.64)};
\path[fill=color2,draw=color2,mark size=\marRad, mark=triangle*] plot coordinates {(19.50, 19.67)};
\path[fill=color2,draw=color2,mark size=\marRad, mark=triangle*] plot coordinates {(18.11, 18.55)};
\path[fill=color2,draw=color2,mark size=\marRad, mark=triangle*] plot coordinates {(18.30, 22.36)};
\path[fill=color2,draw=color2,mark size=\marRad, mark=triangle*] plot coordinates {(21.11, 15.92)};
\path[fill=color2,draw=color2,mark size=\marRad, mark=triangle*] plot coordinates {(22.04, 18.00)};
\path[fill=color2,draw=color2,mark size=\marRad, mark=triangle*] plot coordinates {(17.23, 22.75)};
\path[fill=color2,draw=color2,mark size=\marRad, mark=triangle*] plot coordinates {(20.16, 18.78)};
\path[fill=color2,draw=color2,mark size=\marRad, mark=triangle*] plot coordinates {(20.80, 20.23)};
\path[fill=color2,draw=color2,mark size=\marRad, mark=triangle*] plot coordinates {(18.65, 20.37)};
\path[fill=color2,draw=color2,mark size=\marRad, mark=triangle*] plot coordinates {(21.50, 20.68)};
\path[fill=color2,draw=color2,mark size=\marRad, mark=triangle*] plot coordinates {(19.83, 19.72)};
\path[fill=color2,draw=color2,mark size=\marRad, mark=triangle*] plot coordinates {(20.01, 20.86)};
\path[fill=color2,draw=color2,mark size=\marRad, mark=triangle*] plot coordinates {(20.14, 20.52)};
\path[fill=color2,draw=color2,mark size=\marRad, mark=triangle*] plot coordinates {(21.62, 16.29)};
\path[fill=color2,draw=color2,mark size=\marRad, mark=triangle*] plot coordinates {(22.65, 20.41)};
\path[fill=color2,draw=color2,mark size=\marRad, mark=triangle*] plot coordinates {(21.93, 19.55)};
\path[fill=color2,draw=color2,mark size=\marRad, mark=triangle*] plot coordinates {(17.67, 15.26)};
\path[fill=color2,draw=color2,mark size=\marRad, mark=triangle*] plot coordinates {(19.90, 17.58)};
\path[fill=color2,draw=color2,mark size=\marRad, mark=triangle*] plot coordinates {(22.67, 18.72)};
\path[fill=color2,draw=color2,mark size=\marRad, mark=triangle*] plot coordinates {(22.11, 21.76)};
\path[fill=color2,draw=color2,mark size=\marRad, mark=triangle*] plot coordinates {(20.68, 15.64)};
\path[fill=color2,draw=color2,mark size=\marRad, mark=triangle*] plot coordinates {(20.61, 20.66)};
\path[fill=color2,draw=color2,mark size=\marRad, mark=triangle*] plot coordinates {(21.60, 19.23)};
\path[fill=color2,draw=color2,mark size=\marRad, mark=triangle*] plot coordinates {(18.99, 18.45)};
\path[fill=color2,draw=color2,mark size=\marRad, mark=triangle*] plot coordinates {(21.50, 20.41)};
\path[fill=color2,draw=color2,mark size=\marRad, mark=triangle*] plot coordinates {(20.31, 21.31)};
\path[fill=color2,draw=color2,mark size=\marRad, mark=triangle*] plot coordinates {(22.44, 22.02)};
\path[fill=color2,draw=color2,mark size=\marRad, mark=triangle*] plot coordinates {(22.77, 18.62)};
\path[fill=color2,draw=color2,mark size=\marRad, mark=triangle*] plot coordinates {(17.71, 21.65)};
\path[fill=color2,draw=color2,mark size=\marRad, mark=triangle*] plot coordinates {(20.29, 16.66)};
\path[fill=color2,draw=color2,mark size=\marRad, mark=triangle*] plot coordinates {(20.82, 18.34)};
\path[fill=color2,draw=color2,mark size=\marRad, mark=triangle*] plot coordinates {(23.01, 23.26)};
\path[fill=color2,draw=color2,mark size=\marRad, mark=triangle*] plot coordinates {(12.71, 18.52)};
\path[fill=color2,draw=color2,mark size=\marRad, mark=triangle*] plot coordinates {(20.26, 21.31)};
\path[fill=color2,draw=color2,mark size=\marRad, mark=triangle*] plot coordinates {(20.80, 15.47)};
\path[fill=color2,draw=color2,mark size=\marRad, mark=triangle*] plot coordinates {(18.26, 20.37)};
\path[fill=color2,draw=color2,mark size=\marRad, mark=triangle*] plot coordinates {(22.36, 19.95)};
\path[fill=color2,draw=color2,mark size=\marRad, mark=triangle*] plot coordinates {(17.35, 17.17)};
\path[fill=color2,draw=color2,mark size=\marRad, mark=triangle*] plot coordinates {(19.92, 19.19)};
\path[fill=color2,draw=color2,mark size=\marRad, mark=triangle*] plot coordinates {(15.60, 18.39)};
\path[fill=color2,draw=color2,mark size=\marRad, mark=triangle*] plot coordinates {(20.00, 22.17)};
\path[fill=color2,draw=color2,mark size=\marRad, mark=triangle*] plot coordinates {(20.16, 19.04)};
\path[fill=color2,draw=color2,mark size=\marRad, mark=triangle*] plot coordinates {(20.03, 20.52)};
\path[fill=color2,draw=color2,mark size=\marRad, mark=triangle*] plot coordinates {(18.47, 22.80)};
\path[fill=color2,draw=color2,mark size=\marRad, mark=triangle*] plot coordinates {(20.44, 20.32)};
\path[fill=color2,draw=color2,mark size=\marRad, mark=triangle*] plot coordinates {(22.25, 20.98)};
\path[fill=color2,draw=color2,mark size=\marRad, mark=triangle*] plot coordinates {(16.76, 21.33)};
\path[fill=color2,draw=color2,mark size=\marRad, mark=triangle*] plot coordinates {(20.63, 18.22)};
\path[fill=color2,draw=color2,mark size=\marRad, mark=triangle*] plot coordinates {(21.52, 18.42)};
\path[fill=color2,draw=color2,mark size=\marRad, mark=triangle*] plot coordinates {(19.29, 20.60)};
\path[fill=color2,draw=color2,mark size=\marRad, mark=triangle*] plot coordinates {(19.92, 20.79)};
\path[fill=color2,draw=color2,mark size=\marRad, mark=triangle*] plot coordinates {(19.70, 19.85)};
\path[fill=color2,draw=color2,mark size=\marRad, mark=triangle*] plot coordinates {(19.91, 19.92)};
\path[fill=color2,draw=color2,mark size=\marRad, mark=triangle*] plot coordinates {(21.80, 21.60)};
\path[fill=color2,draw=color2,mark size=\marRad, mark=triangle*] plot coordinates {(20.93, 22.40)};
\path[fill=color2,draw=color2,mark size=\marRad, mark=triangle*] plot coordinates {(19.17, 18.55)};
\path[fill=color2,draw=color2,mark size=\marRad, mark=triangle*] plot coordinates {(22.50, 21.69)};
\path[fill=color2,draw=color2,mark size=\marRad, mark=triangle*] plot coordinates {(17.99, 22.80)};
\path[fill=color2,draw=color2,mark size=\marRad, mark=triangle*] plot coordinates {(20.16, 20.05)};
\path[fill=color2,draw=color2,mark size=\marRad, mark=triangle*] plot coordinates {(17.97, 18.82)};
\path[fill=color2,draw=color2,mark size=\marRad, mark=triangle*] plot coordinates {(17.26, 22.93)};
\path[fill=color2,draw=color2,mark size=\marRad, mark=triangle*] plot coordinates {(20.13, 20.84)};
\path[fill=color2,draw=color2,mark size=\marRad, mark=triangle*] plot coordinates {(19.67, 20.12)};
\path[fill=color2,draw=color2,mark size=\marRad, mark=triangle*] plot coordinates {(21.11, 17.46)};
\path[fill=color2,draw=color2,mark size=\marRad, mark=triangle*] plot coordinates {(18.20, 17.34)};
\path[fill=color2,draw=color2,mark size=\marRad, mark=triangle*] plot coordinates {(21.18, 20.38)};
\path[fill=color2,draw=color2,mark size=\marRad, mark=triangle*] plot coordinates {(24.76, 20.67)};
\path[fill=color2,draw=color2,mark size=\marRad, mark=triangle*] plot coordinates {(17.77, 18.49)};
\path[fill=color2,draw=color2,mark size=\marRad, mark=triangle*] plot coordinates {(22.77, 19.27)};
\path[fill=color2,draw=color2,mark size=\marRad, mark=triangle*] plot coordinates {(19.69, 20.06)};
\path[fill=color2,draw=color2,mark size=\marRad, mark=triangle*] plot coordinates {(18.05, 22.68)};
\path[fill=color2,draw=color2,mark size=\marRad, mark=triangle*] plot coordinates {(20.11, 19.76)};
\path[fill=color2,draw=color2,mark size=\marRad, mark=triangle*] plot coordinates {(20.00, 20.35)};
\path[fill=color2,draw=color2,mark size=\marRad, mark=triangle*] plot coordinates {(19.93, 20.65)};
\path[fill=color2,draw=color2,mark size=\marRad, mark=triangle*] plot coordinates {(19.46, 17.94)};
\path[fill=color2,draw=color2,mark size=\marRad, mark=triangle*] plot coordinates {(23.07, 18.75)};
\path[fill=color2,draw=color2,mark size=\marRad, mark=triangle*] plot coordinates {(21.52, 21.02)};
\path[fill=color2,draw=color2,mark size=\marRad, mark=triangle*] plot coordinates {(20.04, 18.83)};
\path[fill=color2,draw=color2,mark size=\marRad, mark=triangle*] plot coordinates {(19.13, 18.94)};
\path[fill=color2,draw=color2,mark size=\marRad, mark=triangle*] plot coordinates {(18.46, 20.38)};
\path[fill=color2,draw=color2,mark size=\marRad, mark=triangle*] plot coordinates {(21.62, 21.51)};
\path[fill=color2,draw=color2,mark size=\marRad, mark=triangle*] plot coordinates {(20.17, 20.06)};
\path[fill=color2,draw=color2,mark size=\marRad, mark=triangle*] plot coordinates {(16.06, 18.44)};
\path[fill=color2,draw=color2,mark size=\marRad, mark=triangle*] plot coordinates {(19.45, 20.51)};
\path[fill=color2,draw=color2,mark size=\marRad, mark=triangle*] plot coordinates {(22.02, 20.81)};
\path[fill=color2,draw=color2,mark size=\marRad, mark=triangle*] plot coordinates {(20.81, 17.08)};
\path[fill=color2,draw=color2,mark size=\marRad, mark=triangle*] plot coordinates {(21.19, 18.95)};
\path[fill=color2,draw=color2,mark size=\marRad, mark=triangle*] plot coordinates {(20.41, 20.92)};
\path[fill=color2,draw=color2,mark size=\marRad, mark=triangle*] plot coordinates {(17.36, 23.76)};
\path[fill=color2,draw=color2,mark size=\marRad, mark=triangle*] plot coordinates {(17.73, 20.49)};
\path[fill=color2,draw=color2,mark size=\marRad, mark=triangle*] plot coordinates {(20.54, 18.46)};
\path[fill=color2,draw=color2,mark size=\marRad, mark=triangle*] plot coordinates {(17.95, 20.51)};
\path[fill=color2,draw=color2,mark size=\marRad, mark=triangle*] plot coordinates {(19.99, 19.90)};
\path[fill=color2,draw=color2,mark size=\marRad, mark=triangle*] plot coordinates {(13.57, 17.18)};
\path[fill=color2,draw=color2,mark size=\marRad, mark=triangle*] plot coordinates {(17.80, 18.21)};
\path[fill=color2,draw=color2,mark size=\marRad, mark=triangle*] plot coordinates {(21.94, 22.06)};
\path[fill=color2,draw=color2,mark size=\marRad, mark=triangle*] plot coordinates {(18.89, 20.89)};
\path[fill=color2,draw=color2,mark size=\marRad, mark=triangle*] plot coordinates {(20.23, 20.42)};
\path[fill=color2,draw=color2,mark size=\marRad, mark=triangle*] plot coordinates {(22.61, 20.29)};
\path[fill=color2,draw=color2,mark size=\marRad, mark=triangle*] plot coordinates {(19.80, 20.18)};
\path[fill=color2,draw=color2,mark size=\marRad, mark=triangle*] plot coordinates {(20.97, 21.83)};
\path[fill=color2,draw=color2,mark size=\marRad, mark=triangle*] plot coordinates {(18.04, 21.96)};
\path[fill=color2,draw=color2,mark size=\marRad, mark=triangle*] plot coordinates {(20.14, 19.91)};
\path[fill=color2,draw=color2,mark size=\marRad, mark=triangle*] plot coordinates {(20.65, 21.04)};
\path[fill=color2,draw=color2,mark size=\marRad, mark=triangle*] plot coordinates {(20.12, 25.00)};
\path[fill=color2,draw=color2,mark size=\marRad, mark=triangle*] plot coordinates {(21.77, 20.99)};
\path[fill=color2,draw=color2,mark size=\marRad, mark=triangle*] plot coordinates {(19.49, 23.77)};
\path[fill=color2,draw=color2,mark size=\marRad, mark=triangle*] plot coordinates {(22.78, 16.93)};
\path[fill=color2,draw=color2,mark size=\marRad, mark=triangle*] plot coordinates {(17.95, 17.76)};
\path[fill=color2,draw=color2,mark size=\marRad, mark=triangle*] plot coordinates {(18.76, 18.44)};
\path[fill=color2,draw=color2,mark size=\marRad, mark=triangle*] plot coordinates {(20.74, 20.34)};
\path[fill=color2,draw=color2,mark size=\marRad, mark=triangle*] plot coordinates {(15.09, 19.05)};
\path[fill=color2,draw=color2,mark size=\marRad, mark=triangle*] plot coordinates {(22.34, 18.57)};
\path[fill=color2,draw=color2,mark size=\marRad, mark=triangle*] plot coordinates {(19.20, 20.42)};
\path[fill=color2,draw=color2,mark size=\marRad, mark=triangle*] plot coordinates {(19.07, 20.02)};
\path[fill=color2,draw=color2,mark size=\marRad, mark=triangle*] plot coordinates {(18.76, 19.30)};
\path[fill=color2,draw=color2,mark size=\marRad, mark=triangle*] plot coordinates {(18.18, 20.62)};
\path[fill=color2,draw=color2,mark size=\marRad, mark=triangle*] plot coordinates {(22.94, 25.26)};
\path[fill=color2,draw=color2,mark size=\marRad, mark=triangle*] plot coordinates {(20.91, 17.79)};
\path[fill=color2,draw=color2,mark size=\marRad, mark=triangle*] plot coordinates {(19.74, 18.57)};
\path[fill=color2,draw=color2,mark size=\marRad, mark=triangle*] plot coordinates {(19.18, 19.89)};
\path[fill=color2,draw=color2,mark size=\marRad, mark=triangle*] plot coordinates {(18.88, 19.76)};
\path[fill=color2,draw=color2,mark size=\marRad, mark=triangle*] plot coordinates {(19.82, 21.65)};
\path[fill=color2,draw=color2,mark size=\marRad, mark=triangle*] plot coordinates {(22.46, 21.08)};
\path[fill=color2,draw=color2,mark size=\marRad, mark=triangle*] plot coordinates {(17.90, 19.75)};
\path[fill=color2,draw=color2,mark size=\marRad, mark=triangle*] plot coordinates {(19.39, 16.64)};
\path[fill=color2,draw=color2,mark size=\marRad, mark=triangle*] plot coordinates {(19.68, 19.94)};
\path[fill=color2,draw=color2,mark size=\marRad, mark=triangle*] plot coordinates {(18.64, 21.70)};
\path[fill=color2,draw=color2,mark size=\marRad, mark=triangle*] plot coordinates {(23.24, 18.50)};
\path[fill=color2,draw=color2,mark size=\marRad, mark=triangle*] plot coordinates {(16.40, 20.78)};
\path[fill=color2,draw=color2,mark size=\marRad, mark=triangle*] plot coordinates {(20.68, 18.50)};
\path[fill=color2,draw=color2,mark size=\marRad, mark=triangle*] plot coordinates {(20.01, 20.02)};
\path[fill=color2,draw=color2,mark size=\marRad, mark=triangle*] plot coordinates {(20.15, 18.76)};
\path[fill=color2,draw=color2,mark size=\marRad, mark=triangle*] plot coordinates {(18.43, 21.54)};
\path[fill=color2,draw=color2,mark size=\marRad, mark=triangle*] plot coordinates {(19.93, 20.43)};
\path[fill=color2,draw=color2,mark size=\marRad, mark=triangle*] plot coordinates {(20.01, 20.00)};
\path[fill=color2,draw=color2,mark size=\marRad, mark=triangle*] plot coordinates {(19.59, 14.88)};
\path[fill=color2,draw=color2,mark size=\marRad, mark=triangle*] plot coordinates {(20.31, 19.16)};
\path[fill=color2,draw=color2,mark size=\marRad, mark=triangle*] plot coordinates {(21.83, 21.48)};
\path[fill=color2,draw=color2,mark size=\marRad, mark=triangle*] plot coordinates {(16.51, 18.06)};
\path[fill=color2,draw=color2,mark size=\marRad, mark=triangle*] plot coordinates {(20.41, 20.10)};
\path[fill=color2,draw=color2,mark size=\marRad, mark=triangle*] plot coordinates {(20.75, 20.50)};
\path[fill=color2,draw=color2,mark size=\marRad, mark=triangle*] plot coordinates {(19.05, 19.59)};
\path[fill=color2,draw=color2,mark size=\marRad, mark=triangle*] plot coordinates {(16.64, 24.04)};
\path[fill=color2,draw=color2,mark size=\marRad, mark=triangle*] plot coordinates {(20.66, 20.08)};
\path[fill=color2,draw=color2,mark size=\marRad, mark=triangle*] plot coordinates {(19.38, 18.28)};
\path[fill=color2,draw=color2,mark size=\marRad, mark=triangle*] plot coordinates {(18.08, 18.74)};
\path[fill=color2,draw=color2,mark size=\marRad, mark=triangle*] plot coordinates {(22.77, 19.77)};
\path[fill=color2,draw=color2,mark size=\marRad, mark=triangle*] plot coordinates {(20.80, 19.54)};
\path[fill=color2,draw=color2,mark size=\marRad, mark=triangle*] plot coordinates {(19.82, 20.37)};
\path[fill=color2,draw=color2,mark size=\marRad, mark=triangle*] plot coordinates {(16.11, 19.61)};
\path[fill=color2,draw=color2,mark size=\marRad, mark=triangle*] plot coordinates {(17.15, 19.67)};
\path[fill=color2,draw=color2,mark size=\marRad, mark=triangle*] plot coordinates {(19.56, 20.12)};
\path[fill=color2,draw=color2,mark size=\marRad, mark=triangle*] plot coordinates {(17.95, 20.11)};
\path[fill=color2,draw=color2,mark size=\marRad, mark=triangle*] plot coordinates {(20.36, 19.52)};
\path[fill=color2,draw=color2,mark size=\marRad, mark=triangle*] plot coordinates {(23.15, 21.54)};
\path[fill=color2,draw=color2,mark size=\marRad, mark=triangle*] plot coordinates {(20.06, 19.72)};
\path[fill=color2,draw=color2,mark size=\marRad, mark=triangle*] plot coordinates {(19.53, 20.75)};
\path[fill=color2,draw=color2,mark size=\marRad, mark=triangle*] plot coordinates {(16.52, 19.99)};
\path[fill=color6,draw=color6,mark size=\marRad, mark=heart] plot coordinates {(41.03, 33.77)};
\path[fill=color6,draw=color6,mark size=\marRad, mark=heart] plot coordinates {(40.76, 28.68)};
\path[fill=color6,draw=color6,mark size=\marRad, mark=heart] plot coordinates {(40.76, 27.31)};
\path[fill=color6,draw=color6,mark size=\marRad, mark=heart] plot coordinates {(44.14, 30.17)};
\path[fill=color6,draw=color6,mark size=\marRad, mark=heart] plot coordinates {(40.55, 30.86)};
\path[fill=color6,draw=color6,mark size=\marRad, mark=heart] plot coordinates {(40.40, 30.60)};
\path[fill=color6,draw=color6,mark size=\marRad, mark=heart] plot coordinates {(40.28, 30.79)};
\path[fill=color6,draw=color6,mark size=\marRad, mark=heart] plot coordinates {(39.16, 32.33)};
\path[fill=color6,draw=color6,mark size=\marRad, mark=heart] plot coordinates {(40.02, 30.47)};
\path[fill=color6,draw=color6,mark size=\marRad, mark=heart] plot coordinates {(40.01, 30.02)};
\path[fill=color6,draw=color6,mark size=\marRad, mark=heart] plot coordinates {(42.57, 31.35)};
\path[fill=color6,draw=color6,mark size=\marRad, mark=heart] plot coordinates {(41.40, 34.80)};
\path[fill=color6,draw=color6,mark size=\marRad, mark=heart] plot coordinates {(41.14, 30.08)};
\path[fill=color6,draw=color6,mark size=\marRad, mark=heart] plot coordinates {(38.46, 27.54)};
\path[fill=color6,draw=color6,mark size=\marRad, mark=heart] plot coordinates {(36.78, 30.53)};
\path[fill=color6,draw=color6,mark size=\marRad, mark=heart] plot coordinates {(40.01, 30.41)};
\path[fill=color6,draw=color6,mark size=\marRad, mark=heart] plot coordinates {(41.13, 29.48)};
\path[fill=color6,draw=color6,mark size=\marRad, mark=heart] plot coordinates {(39.65, 26.69)};
\path[fill=color6,draw=color6,mark size=\marRad, mark=heart] plot coordinates {(41.88, 31.76)};
\path[fill=color6,draw=color6,mark size=\marRad, mark=heart] plot coordinates {(37.51, 26.45)};
\path[fill=color6,draw=color6,mark size=\marRad, mark=heart] plot coordinates {(42.79, 25.19)};
\path[fill=color6,draw=color6,mark size=\marRad, mark=heart] plot coordinates {(41.52, 29.43)};
\path[fill=color6,draw=color6,mark size=\marRad, mark=heart] plot coordinates {(40.00, 30.00)};
\path[fill=color6,draw=color6,mark size=\marRad, mark=heart] plot coordinates {(40.19, 29.88)};
\path[fill=color6,draw=color6,mark size=\marRad, mark=heart] plot coordinates {(35.49, 32.16)};
\path[fill=color6,draw=color6,mark size=\marRad, mark=heart] plot coordinates {(42.58, 23.24)};
\path[fill=color6,draw=color6,mark size=\marRad, mark=heart] plot coordinates {(41.15, 30.37)};
\path[fill=color6,draw=color6,mark size=\marRad, mark=heart] plot coordinates {(41.28, 30.47)};
\path[fill=color6,draw=color6,mark size=\marRad, mark=heart] plot coordinates {(39.68, 29.02)};
\path[fill=color6,draw=color6,mark size=\marRad, mark=heart] plot coordinates {(40.95, 33.26)};
\path[fill=color6,draw=color6,mark size=\marRad, mark=heart] plot coordinates {(39.72, 28.21)};
\path[fill=color6,draw=color6,mark size=\marRad, mark=heart] plot coordinates {(42.04, 29.78)};
\path[fill=color6,draw=color6,mark size=\marRad, mark=heart] plot coordinates {(40.55, 30.90)};
\path[fill=color6,draw=color6,mark size=\marRad, mark=heart] plot coordinates {(39.24, 29.11)};
\path[fill=color6,draw=color6,mark size=\marRad, mark=heart] plot coordinates {(40.17, 30.60)};
\path[fill=color6,draw=color6,mark size=\marRad, mark=heart] plot coordinates {(39.86, 29.20)};
\path[fill=color6,draw=color6,mark size=\marRad, mark=heart] plot coordinates {(40.01, 30.01)};
\path[fill=color6,draw=color6,mark size=\marRad, mark=heart] plot coordinates {(39.80, 29.22)};
\path[fill=color6,draw=color6,mark size=\marRad, mark=heart] plot coordinates {(39.80, 29.82)};
\path[fill=color6,draw=color6,mark size=\marRad, mark=heart] plot coordinates {(40.31, 29.96)};
\path[fill=color6,draw=color6,mark size=\marRad, mark=heart] plot coordinates {(37.99, 29.64)};
\path[fill=color6,draw=color6,mark size=\marRad, mark=heart] plot coordinates {(39.50, 29.67)};
\path[fill=color6,draw=color6,mark size=\marRad, mark=heart] plot coordinates {(38.11, 28.55)};
\path[fill=color6,draw=color6,mark size=\marRad, mark=heart] plot coordinates {(38.30, 32.36)};
\path[fill=color6,draw=color6,mark size=\marRad, mark=heart] plot coordinates {(41.11, 25.92)};
\path[fill=color6,draw=color6,mark size=\marRad, mark=heart] plot coordinates {(42.04, 28.00)};
\path[fill=color6,draw=color6,mark size=\marRad, mark=heart] plot coordinates {(37.23, 32.75)};
\path[fill=color6,draw=color6,mark size=\marRad, mark=heart] plot coordinates {(40.16, 28.78)};
\path[fill=color6,draw=color6,mark size=\marRad, mark=heart] plot coordinates {(40.80, 30.23)};
\path[fill=color6,draw=color6,mark size=\marRad, mark=heart] plot coordinates {(38.65, 30.37)};
\path[fill=color6,draw=color6,mark size=\marRad, mark=heart] plot coordinates {(41.50, 30.68)};
\path[fill=color6,draw=color6,mark size=\marRad, mark=heart] plot coordinates {(39.83, 29.72)};
\path[fill=color6,draw=color6,mark size=\marRad, mark=heart] plot coordinates {(40.01, 30.86)};
\path[fill=color6,draw=color6,mark size=\marRad, mark=heart] plot coordinates {(40.14, 30.52)};
\path[fill=color6,draw=color6,mark size=\marRad, mark=heart] plot coordinates {(41.62, 26.29)};
\path[fill=color6,draw=color6,mark size=\marRad, mark=heart] plot coordinates {(42.65, 30.41)};
\path[fill=color6,draw=color6,mark size=\marRad, mark=heart] plot coordinates {(41.93, 29.55)};
\path[fill=color6,draw=color6,mark size=\marRad, mark=heart] plot coordinates {(37.67, 25.26)};
\path[fill=color6,draw=color6,mark size=\marRad, mark=heart] plot coordinates {(39.90, 27.58)};
\path[fill=color6,draw=color6,mark size=\marRad, mark=heart] plot coordinates {(42.67, 28.72)};
\path[fill=color6,draw=color6,mark size=\marRad, mark=heart] plot coordinates {(42.11, 31.76)};
\path[fill=color6,draw=color6,mark size=\marRad, mark=heart] plot coordinates {(40.68, 25.64)};
\path[fill=color6,draw=color6,mark size=\marRad, mark=heart] plot coordinates {(40.61, 30.66)};
\path[fill=color6,draw=color6,mark size=\marRad, mark=heart] plot coordinates {(41.60, 29.23)};
\path[fill=color6,draw=color6,mark size=\marRad, mark=heart] plot coordinates {(38.99, 28.45)};
\path[fill=color6,draw=color6,mark size=\marRad, mark=heart] plot coordinates {(41.50, 30.41)};
\path[fill=color6,draw=color6,mark size=\marRad, mark=heart] plot coordinates {(40.31, 31.31)};
\path[fill=color6,draw=color6,mark size=\marRad, mark=heart] plot coordinates {(42.44, 32.02)};
\path[fill=color6,draw=color6,mark size=\marRad, mark=heart] plot coordinates {(42.77, 28.62)};
\path[fill=color6,draw=color6,mark size=\marRad, mark=heart] plot coordinates {(37.71, 31.65)};
\path[fill=color6,draw=color6,mark size=\marRad, mark=heart] plot coordinates {(40.29, 26.66)};
\path[fill=color6,draw=color6,mark size=\marRad, mark=heart] plot coordinates {(40.82, 28.34)};
\path[fill=color6,draw=color6,mark size=\marRad, mark=heart] plot coordinates {(43.01, 33.26)};
\path[fill=color6,draw=color6,mark size=\marRad, mark=heart] plot coordinates {(32.71, 28.52)};
\path[fill=color6,draw=color6,mark size=\marRad, mark=heart] plot coordinates {(40.26, 31.31)};
\path[fill=color6,draw=color6,mark size=\marRad, mark=heart] plot coordinates {(40.80, 25.47)};
\path[fill=color6,draw=color6,mark size=\marRad, mark=heart] plot coordinates {(38.26, 30.37)};
\path[fill=color6,draw=color6,mark size=\marRad, mark=heart] plot coordinates {(42.36, 29.95)};
\path[fill=color6,draw=color6,mark size=\marRad, mark=heart] plot coordinates {(37.35, 27.17)};
\path[fill=color6,draw=color6,mark size=\marRad, mark=heart] plot coordinates {(39.92, 29.19)};
\path[fill=color6,draw=color6,mark size=\marRad, mark=heart] plot coordinates {(35.60, 28.39)};
\path[fill=color6,draw=color6,mark size=\marRad, mark=heart] plot coordinates {(40.00, 32.17)};
\path[fill=color6,draw=color6,mark size=\marRad, mark=heart] plot coordinates {(40.16, 29.04)};
\path[fill=color6,draw=color6,mark size=\marRad, mark=heart] plot coordinates {(40.03, 30.52)};
\path[fill=color6,draw=color6,mark size=\marRad, mark=heart] plot coordinates {(38.47, 32.80)};
\path[fill=color6,draw=color6,mark size=\marRad, mark=heart] plot coordinates {(40.44, 30.32)};
\path[fill=color6,draw=color6,mark size=\marRad, mark=heart] plot coordinates {(42.25, 30.98)};
\path[fill=color6,draw=color6,mark size=\marRad, mark=heart] plot coordinates {(36.76, 31.33)};
\path[fill=color6,draw=color6,mark size=\marRad, mark=heart] plot coordinates {(40.63, 28.22)};
\path[fill=color6,draw=color6,mark size=\marRad, mark=heart] plot coordinates {(41.52, 28.42)};
\path[fill=color6,draw=color6,mark size=\marRad, mark=heart] plot coordinates {(39.29, 30.60)};
\path[fill=color6,draw=color6,mark size=\marRad, mark=heart] plot coordinates {(39.92, 30.79)};
\path[fill=color6,draw=color6,mark size=\marRad, mark=heart] plot coordinates {(39.70, 29.85)};
\path[fill=color6,draw=color6,mark size=\marRad, mark=heart] plot coordinates {(39.91, 29.92)};
\path[fill=color6,draw=color6,mark size=\marRad, mark=heart] plot coordinates {(41.80, 31.60)};
\path[fill=color6,draw=color6,mark size=\marRad, mark=heart] plot coordinates {(40.93, 32.40)};
\path[fill=color6,draw=color6,mark size=\marRad, mark=heart] plot coordinates {(39.17, 28.55)};
\path[fill=color6,draw=color6,mark size=\marRad, mark=heart] plot coordinates {(42.50, 31.69)};
\path[fill=color6,draw=color6,mark size=\marRad, mark=heart] plot coordinates {(37.99, 32.80)};
\path[fill=color6,draw=color6,mark size=\marRad, mark=heart] plot coordinates {(40.16, 30.05)};
\path[fill=color6,draw=color6,mark size=\marRad, mark=heart] plot coordinates {(37.97, 28.82)};
\path[fill=color6,draw=color6,mark size=\marRad, mark=heart] plot coordinates {(37.26, 32.93)};
\path[fill=color6,draw=color6,mark size=\marRad, mark=heart] plot coordinates {(40.13, 30.84)};
\path[fill=color6,draw=color6,mark size=\marRad, mark=heart] plot coordinates {(39.67, 30.12)};
\path[fill=color6,draw=color6,mark size=\marRad, mark=heart] plot coordinates {(41.11, 27.46)};
\path[fill=color6,draw=color6,mark size=\marRad, mark=heart] plot coordinates {(38.20, 27.34)};
\path[fill=color6,draw=color6,mark size=\marRad, mark=heart] plot coordinates {(41.18, 30.38)};
\path[fill=color6,draw=color6,mark size=\marRad, mark=heart] plot coordinates {(44.76, 30.67)};
\path[fill=color6,draw=color6,mark size=\marRad, mark=heart] plot coordinates {(37.77, 28.49)};
\path[fill=color6,draw=color6,mark size=\marRad, mark=heart] plot coordinates {(42.77, 29.27)};
\path[fill=color6,draw=color6,mark size=\marRad, mark=heart] plot coordinates {(39.69, 30.06)};
\path[fill=color6,draw=color6,mark size=\marRad, mark=heart] plot coordinates {(38.05, 32.68)};
\path[fill=color6,draw=color6,mark size=\marRad, mark=heart] plot coordinates {(40.11, 29.76)};
\path[fill=color6,draw=color6,mark size=\marRad, mark=heart] plot coordinates {(40.00, 30.35)};
\path[fill=color6,draw=color6,mark size=\marRad, mark=heart] plot coordinates {(39.93, 30.65)};
\path[fill=color6,draw=color6,mark size=\marRad, mark=heart] plot coordinates {(39.46, 27.94)};
\path[fill=color6,draw=color6,mark size=\marRad, mark=heart] plot coordinates {(43.07, 28.75)};
\path[fill=color6,draw=color6,mark size=\marRad, mark=heart] plot coordinates {(41.52, 31.02)};
\path[fill=color6,draw=color6,mark size=\marRad, mark=heart] plot coordinates {(40.04, 28.83)};
\path[fill=color6,draw=color6,mark size=\marRad, mark=heart] plot coordinates {(39.13, 28.94)};
\path[fill=color6,draw=color6,mark size=\marRad, mark=heart] plot coordinates {(38.46, 30.38)};
\path[fill=color6,draw=color6,mark size=\marRad, mark=heart] plot coordinates {(41.62, 31.51)};
\path[fill=color6,draw=color6,mark size=\marRad, mark=heart] plot coordinates {(40.17, 30.06)};
\path[fill=color6,draw=color6,mark size=\marRad, mark=heart] plot coordinates {(36.06, 28.44)};
\path[fill=color6,draw=color6,mark size=\marRad, mark=heart] plot coordinates {(39.45, 30.51)};
\path[fill=color6,draw=color6,mark size=\marRad, mark=heart] plot coordinates {(42.02, 30.81)};
\path[fill=color6,draw=color6,mark size=\marRad, mark=heart] plot coordinates {(40.81, 27.08)};
\path[fill=color6,draw=color6,mark size=\marRad, mark=heart] plot coordinates {(41.19, 28.95)};
\path[fill=color6,draw=color6,mark size=\marRad, mark=heart] plot coordinates {(40.41, 30.92)};
\path[fill=color6,draw=color6,mark size=\marRad, mark=heart] plot coordinates {(37.36, 33.76)};
\path[fill=color6,draw=color6,mark size=\marRad, mark=heart] plot coordinates {(37.73, 30.49)};
\path[fill=color6,draw=color6,mark size=\marRad, mark=heart] plot coordinates {(40.54, 28.46)};
\path[fill=color6,draw=color6,mark size=\marRad, mark=heart] plot coordinates {(37.95, 30.51)};
\path[fill=color6,draw=color6,mark size=\marRad, mark=heart] plot coordinates {(39.99, 29.90)};
\path[fill=color6,draw=color6,mark size=\marRad, mark=heart] plot coordinates {(33.57, 27.18)};
\path[fill=color6,draw=color6,mark size=\marRad, mark=heart] plot coordinates {(37.80, 28.21)};
\path[fill=color6,draw=color6,mark size=\marRad, mark=heart] plot coordinates {(41.94, 32.06)};
\path[fill=color6,draw=color6,mark size=\marRad, mark=heart] plot coordinates {(38.89, 30.89)};
\path[fill=color6,draw=color6,mark size=\marRad, mark=heart] plot coordinates {(40.23, 30.42)};
\path[fill=color6,draw=color6,mark size=\marRad, mark=heart] plot coordinates {(42.61, 30.29)};
\path[fill=color6,draw=color6,mark size=\marRad, mark=heart] plot coordinates {(39.80, 30.18)};
\path[fill=color6,draw=color6,mark size=\marRad, mark=heart] plot coordinates {(40.97, 31.83)};
\path[fill=color6,draw=color6,mark size=\marRad, mark=heart] plot coordinates {(38.04, 31.96)};
\path[fill=color6,draw=color6,mark size=\marRad, mark=heart] plot coordinates {(40.14, 29.91)};
\path[fill=color6,draw=color6,mark size=\marRad, mark=heart] plot coordinates {(40.65, 31.04)};
\path[fill=color6,draw=color6,mark size=\marRad, mark=heart] plot coordinates {(40.12, 35.00)};
\path[fill=color6,draw=color6,mark size=\marRad, mark=heart] plot coordinates {(41.77, 30.99)};
\path[fill=color6,draw=color6,mark size=\marRad, mark=heart] plot coordinates {(39.49, 33.77)};
\path[fill=color6,draw=color6,mark size=\marRad, mark=heart] plot coordinates {(42.78, 26.93)};
\path[fill=color6,draw=color6,mark size=\marRad, mark=heart] plot coordinates {(37.95, 27.76)};
\path[fill=color6,draw=color6,mark size=\marRad, mark=heart] plot coordinates {(38.76, 28.44)};
\path[fill=color6,draw=color6,mark size=\marRad, mark=heart] plot coordinates {(40.74, 30.34)};
\path[fill=color6,draw=color6,mark size=\marRad, mark=heart] plot coordinates {(35.09, 29.05)};
\path[fill=color6,draw=color6,mark size=\marRad, mark=heart] plot coordinates {(42.34, 28.57)};
\path[fill=color6,draw=color6,mark size=\marRad, mark=heart] plot coordinates {(39.20, 30.42)};
\path[fill=color6,draw=color6,mark size=\marRad, mark=heart] plot coordinates {(39.07, 30.02)};
\path[fill=color6,draw=color6,mark size=\marRad, mark=heart] plot coordinates {(38.76, 29.30)};
\path[fill=color6,draw=color6,mark size=\marRad, mark=heart] plot coordinates {(38.18, 30.62)};
\path[fill=color6,draw=color6,mark size=\marRad, mark=heart] plot coordinates {(42.94, 35.26)};
\path[fill=color6,draw=color6,mark size=\marRad, mark=heart] plot coordinates {(40.91, 27.79)};
\path[fill=color6,draw=color6,mark size=\marRad, mark=heart] plot coordinates {(39.74, 28.57)};
\path[fill=color6,draw=color6,mark size=\marRad, mark=heart] plot coordinates {(39.18, 29.89)};
\path[fill=color6,draw=color6,mark size=\marRad, mark=heart] plot coordinates {(38.88, 29.76)};
\path[fill=color6,draw=color6,mark size=\marRad, mark=heart] plot coordinates {(39.82, 31.65)};
\path[fill=color6,draw=color6,mark size=\marRad, mark=heart] plot coordinates {(42.46, 31.08)};
\path[fill=color6,draw=color6,mark size=\marRad, mark=heart] plot coordinates {(37.90, 29.75)};
\path[fill=color6,draw=color6,mark size=\marRad, mark=heart] plot coordinates {(39.39, 26.64)};
\path[fill=color6,draw=color6,mark size=\marRad, mark=heart] plot coordinates {(39.68, 29.94)};
\path[fill=color6,draw=color6,mark size=\marRad, mark=heart] plot coordinates {(38.64, 31.70)};
\path[fill=color6,draw=color6,mark size=\marRad, mark=heart] plot coordinates {(43.24, 28.50)};
\path[fill=color6,draw=color6,mark size=\marRad, mark=heart] plot coordinates {(36.40, 30.78)};
\path[fill=color6,draw=color6,mark size=\marRad, mark=heart] plot coordinates {(40.68, 28.50)};
\path[fill=color6,draw=color6,mark size=\marRad, mark=heart] plot coordinates {(40.01, 30.02)};
\path[fill=color6,draw=color6,mark size=\marRad, mark=heart] plot coordinates {(40.15, 28.76)};
\path[fill=color6,draw=color6,mark size=\marRad, mark=heart] plot coordinates {(38.43, 31.54)};
\path[fill=color6,draw=color6,mark size=\marRad, mark=heart] plot coordinates {(39.93, 30.43)};
\path[fill=color6,draw=color6,mark size=\marRad, mark=heart] plot coordinates {(40.01, 30.00)};
\path[fill=color6,draw=color6,mark size=\marRad, mark=heart] plot coordinates {(39.59, 24.88)};
\path[fill=color6,draw=color6,mark size=\marRad, mark=heart] plot coordinates {(40.31, 29.16)};
\path[fill=color6,draw=color6,mark size=\marRad, mark=heart] plot coordinates {(41.83, 31.48)};
\path[fill=color6,draw=color6,mark size=\marRad, mark=heart] plot coordinates {(36.51, 28.06)};
\path[fill=color6,draw=color6,mark size=\marRad, mark=heart] plot coordinates {(40.41, 30.10)};
\path[fill=color6,draw=color6,mark size=\marRad, mark=heart] plot coordinates {(40.75, 30.50)};
\path[fill=color6,draw=color6,mark size=\marRad, mark=heart] plot coordinates {(39.05, 29.59)};
\path[fill=color6,draw=color6,mark size=\marRad, mark=heart] plot coordinates {(36.64, 34.04)};
\path[fill=color6,draw=color6,mark size=\marRad, mark=heart] plot coordinates {(40.66, 30.08)};
\path[fill=color6,draw=color6,mark size=\marRad, mark=heart] plot coordinates {(39.38, 28.28)};
\path[fill=color6,draw=color6,mark size=\marRad, mark=heart] plot coordinates {(38.08, 28.74)};
\path[fill=color6,draw=color6,mark size=\marRad, mark=heart] plot coordinates {(42.77, 29.77)};
\path[fill=color6,draw=color6,mark size=\marRad, mark=heart] plot coordinates {(40.80, 29.54)};
\path[fill=color6,draw=color6,mark size=\marRad, mark=heart] plot coordinates {(39.82, 30.37)};
\path[fill=color6,draw=color6,mark size=\marRad, mark=heart] plot coordinates {(36.11, 29.61)};
\path[fill=color6,draw=color6,mark size=\marRad, mark=heart] plot coordinates {(37.15, 29.67)};
\path[fill=color6,draw=color6,mark size=\marRad, mark=heart] plot coordinates {(39.56, 30.12)};
\path[fill=color6,draw=color6,mark size=\marRad, mark=heart] plot coordinates {(37.95, 30.11)};
\path[fill=color6,draw=color6,mark size=\marRad, mark=heart] plot coordinates {(40.36, 29.52)};
\path[fill=color6,draw=color6,mark size=\marRad, mark=heart] plot coordinates {(43.15, 31.54)};
\path[fill=color6,draw=color6,mark size=\marRad, mark=heart] plot coordinates {(40.06, 29.72)};
\path[fill=color6,draw=color6,mark size=\marRad, mark=heart] plot coordinates {(39.53, 30.75)};
\path[fill=color6,draw=color6,mark size=\marRad, mark=heart] plot coordinates {(36.52, 29.99)};

\draw (10, 40) node[very thick,draw=black!50,top color=white,bottom color=black!20,anchor=west, draw=black] (A) {Cluster zu verschiedenen Themen};
\draw[-{Latex[length=2mm]}] (A.south) -- ($(A.south)!0.7!(20, 20)$);
\draw[-{Latex[length=2mm]}] (A.south) .. controls ++(-10,-7) .. ($($(A.south)+(-10,-7)$)!0.7!(10, 10)$);
\draw[-{Latex[length=2mm]}] (A.south) -- ($(A.south)!0.7!(40, 30)$);
\draw[-{Latex[length=2mm]}] (A.south) .. controls ++(10,-7)   .. ($($(A.south)+(10,-7)$)!0.7!(30, 10)$);

\draw (10, 0) node[very thick,draw=black!50,top color=white,bottom color=black!20,anchor=north,draw=black, rounded corners] (B) {Ein Tweet};
\draw[-{Stealth[length=2mm]}] (B.north) -- (12.58, 3.24);
\end{tikzpicture}

\end{frame}


\begin{frame}{Module - Clustering - Umsetzungsidee}
%Umsetzungsidee:
\begin{itemize}
 \item Features für Tweets/Hashtags berechnen.
%  \item Hashtags aus allen Tweets zu einem Begriff als binäre Attribute der Tweets benutzen,
%  um Ähnlichkeiten zwischen den Tweets zu berechnen:
%  \begin{center}
% \begin{tabular}{r||c|c|c|c}
%          & hashtag1 & hashtag2 & $\cdots$ & hashtagN \\ \hline
% tweet1   & 1        & 0        & $\cdots$ & 1        \\ \hline
% tweet2   & 1        & 1        & $\cdots$ & 0        \\ \hline
% $\vdots$ & $\vdots$ & $\vdots$ & $\ddots$ & $\vdots$ \\ \hline
% tweetM   & 1        & 0        & $\cdots$ & 1     
%  \end{tabular}
%  \end{center} 
\item Ähnlichkeiten zwischen allen Tweet/Hashtag-Paaren berechnen.
% Beispielähnlichkeit (Russel Rao): $$s_{ij} = \dfrac{\#\text{Spalte mit gemeinsamen Einsen}}{\#\text{Spalten}}.$$
\item Tweets clustern, um festzulegen welche Tweets zusammengehören.
\item Tweets in der xy-Ebene Positionen zuordnen.
\end{itemize}
\end{frame}


\begin{frame}{Module - Clustering - Umsetzungsidee}
Ergebnis: 

\def\maxW{3.33}
\begin{tikzpicture}[y=\textwidth/3.0,x=\textwidth/3.0, background rectangle/.style={draw=black, thick, fill=yellow!10,},show background rectangle]
\def\marRad{0.5mm}
\definecolor{color0}{rgb}{0.06,0.64,0.44}
\definecolor{color1}{rgb}{1.00,0.00,0.00}
\definecolor{color2}{rgb}{0.17,0.36,0.70}
\definecolor{color3}{rgb}{0.05,0.96,0.52}
\definecolor{color4}{rgb}{0.38,0.31,0.64}
\definecolor{color5}{rgb}{0.49,0.24,0.58}
\definecolor{color6}{rgb}{0.55,0.58,0.40}
\definecolor{color7}{rgb}{0.48,0.75,0.82}
\definecolor{color8}{rgb}{0.64,0.70,0.11}
\definecolor{color9}{rgb}{0.03,0.40,0.65}
\path[fill=color0,draw=color0,mark size=\marRad, mark=*] plot coordinates {(0.21, -0.60)};
\path[fill=color0,draw=color0,mark size=\marRad, mark=*] plot coordinates {(0.13, -0.71)};
\path[fill=color0,draw=color0,mark size=\marRad, mark=*] plot coordinates {(0.54, -0.48)};
\path[fill=color0,draw=color0,mark size=\marRad, mark=*] plot coordinates {(0.27, -0.57)};
\path[fill=color0,draw=color0,mark size=\marRad, mark=*] plot coordinates {(0.32, -0.54)};
\path[fill=color0,draw=color0,mark size=\marRad, mark=*] plot coordinates {(0.32, -0.54)};
\path[fill=color0,draw=color0,mark size=\marRad, mark=*] plot coordinates {(0.19, -0.60)};
\path[fill=color0,draw=color0,mark size=\marRad, mark=*] plot coordinates {(0.25, -0.59)};
\path[fill=color0,draw=color0,mark size=\marRad, mark=*] plot coordinates {(0.40, -0.50)};
\path[fill=color0,draw=color0,mark size=\marRad, mark=*] plot coordinates {(0.43, -0.47)};
\path[fill=color0,draw=color0,mark size=\marRad, mark=*] plot coordinates {(0.30, -0.56)};
\path[fill=color0,draw=color0,mark size=\marRad, mark=*] plot coordinates {(0.15, -0.62)};
\path[fill=color0,draw=color0,mark size=\marRad, mark=*] plot coordinates {(0.20, -0.67)};
\path[fill=color0,draw=color0,mark size=\marRad, mark=*] plot coordinates {(0.28, -0.56)};
\path[fill=color0,draw=color0,mark size=\marRad, mark=*] plot coordinates {(0.37, -0.52)};
\path[fill=color0,draw=color0,mark size=\marRad, mark=*] plot coordinates {(0.38, -0.51)};
\path[fill=color0,draw=color0,mark size=\marRad, mark=*] plot coordinates {(0.29, -0.56)};
\path[fill=color0,draw=color0,mark size=\marRad, mark=*] plot coordinates {(0.34, -0.54)};
\path[fill=color0,draw=color0,mark size=\marRad, mark=*] plot coordinates {(0.30, -0.56)};
\path[fill=color0,draw=color0,mark size=\marRad, mark=*] plot coordinates {(0.44, -0.56)};
\path[fill=color0,draw=color0,mark size=\marRad, mark=*] plot coordinates {(0.47, -0.54)};
\path[fill=color0,draw=color0,mark size=\marRad, mark=*] plot coordinates {(0.44, -0.47)};
\path[fill=color0,draw=color0,mark size=\marRad, mark=*] plot coordinates {(0.44, -0.47)};
\path[fill=color0,draw=color0,mark size=\marRad, mark=*] plot coordinates {(0.38, -0.62)};
\path[fill=color0,draw=color0,mark size=\marRad, mark=*] plot coordinates {(0.06, -0.63)};
\path[fill=color0,draw=color0,mark size=\marRad, mark=*] plot coordinates {(0.33, -0.55)};
\path[fill=color0,draw=color0,mark size=\marRad, mark=*] plot coordinates {(0.33, -0.55)};
\path[fill=color0,draw=color0,mark size=\marRad, mark=*] plot coordinates {(0.35, -0.52)};
\path[fill=color0,draw=color0,mark size=\marRad, mark=*] plot coordinates {(0.20, -0.68)};
\path[fill=color0,draw=color0,mark size=\marRad, mark=*] plot coordinates {(0.38, -0.51)};
\path[fill=color0,draw=color0,mark size=\marRad, mark=*] plot coordinates {(0.35, -0.54)};
\path[fill=color0,draw=color0,mark size=\marRad, mark=*] plot coordinates {(0.27, -0.66)};
\path[fill=color0,draw=color0,mark size=\marRad, mark=*] plot coordinates {(0.12, -0.70)};
\path[fill=color0,draw=color0,mark size=\marRad, mark=*] plot coordinates {(0.29, -0.57)};
\path[fill=color0,draw=color0,mark size=\marRad, mark=*] plot coordinates {(0.40, -0.61)};
\path[fill=color0,draw=color0,mark size=\marRad, mark=*] plot coordinates {(0.33, -0.65)};
\path[fill=color1,draw=color1,mark size=\marRad, mark=square*] plot coordinates {(0.17, 0.46)};
\path[fill=color1,draw=color1,mark size=\marRad, mark=square*] plot coordinates {(0.01, 0.04)};
\path[fill=color1,draw=color1,mark size=\marRad, mark=square*] plot coordinates {(0.31, -0.01)};
\path[fill=color1,draw=color1,mark size=\marRad, mark=square*] plot coordinates {(-0.28, 0.41)};
\path[fill=color1,draw=color1,mark size=\marRad, mark=square*] plot coordinates {(-0.05, 0.49)};
\path[fill=color1,draw=color1,mark size=\marRad, mark=square*] plot coordinates {(0.39, 0.30)};
\path[fill=color1,draw=color1,mark size=\marRad, mark=square*] plot coordinates {(-0.16, -0.39)};
\path[fill=color1,draw=color1,mark size=\marRad, mark=square*] plot coordinates {(0.01, 0.04)};
\path[fill=color1,draw=color1,mark size=\marRad, mark=square*] plot coordinates {(0.01, 0.04)};
\path[fill=color1,draw=color1,mark size=\marRad, mark=square*] plot coordinates {(0.01, 0.04)};
\path[fill=color1,draw=color1,mark size=\marRad, mark=square*] plot coordinates {(0.01, 0.04)};
\path[fill=color1,draw=color1,mark size=\marRad, mark=square*] plot coordinates {(-0.46, -0.02)};
\path[fill=color1,draw=color1,mark size=\marRad, mark=square*] plot coordinates {(-0.25, -0.09)};
\path[fill=color1,draw=color1,mark size=\marRad, mark=square*] plot coordinates {(0.47, 0.12)};
\path[fill=color1,draw=color1,mark size=\marRad, mark=square*] plot coordinates {(0.01, 0.04)};
\path[fill=color1,draw=color1,mark size=\marRad, mark=square*] plot coordinates {(-0.20, 0.26)};
\path[fill=color1,draw=color1,mark size=\marRad, mark=square*] plot coordinates {(0.01, 0.04)};
\path[fill=color1,draw=color1,mark size=\marRad, mark=square*] plot coordinates {(0.01, 0.04)};
\path[fill=color1,draw=color1,mark size=\marRad, mark=square*] plot coordinates {(0.01, 0.04)};
\path[fill=color1,draw=color1,mark size=\marRad, mark=square*] plot coordinates {(0.47, 0.08)};
\path[fill=color1,draw=color1,mark size=\marRad, mark=square*] plot coordinates {(0.01, 0.04)};
\path[fill=color1,draw=color1,mark size=\marRad, mark=square*] plot coordinates {(-0.27, 0.51)};
\path[fill=color1,draw=color1,mark size=\marRad, mark=square*] plot coordinates {(-0.44, 0.13)};
\path[fill=color1,draw=color1,mark size=\marRad, mark=square*] plot coordinates {(-0.39, -0.20)};
\path[fill=color1,draw=color1,mark size=\marRad, mark=square*] plot coordinates {(-0.27, 0.14)};
\path[fill=color1,draw=color1,mark size=\marRad, mark=square*] plot coordinates {(0.01, 0.04)};
\path[fill=color1,draw=color1,mark size=\marRad, mark=square*] plot coordinates {(0.01, 0.04)};
\path[fill=color1,draw=color1,mark size=\marRad, mark=square*] plot coordinates {(0.01, 0.04)};
\path[fill=color1,draw=color1,mark size=\marRad, mark=square*] plot coordinates {(-0.54, -0.13)};
\path[fill=color1,draw=color1,mark size=\marRad, mark=square*] plot coordinates {(0.01, 0.04)};
\path[fill=color1,draw=color1,mark size=\marRad, mark=square*] plot coordinates {(0.01, 0.04)};
\path[fill=color1,draw=color1,mark size=\marRad, mark=square*] plot coordinates {(-0.04, 0.59)};
\path[fill=color1,draw=color1,mark size=\marRad, mark=square*] plot coordinates {(0.01, 0.04)};
\path[fill=color1,draw=color1,mark size=\marRad, mark=square*] plot coordinates {(0.29, -0.32)};
\path[fill=color1,draw=color1,mark size=\marRad, mark=square*] plot coordinates {(0.28, 0.43)};
\path[fill=color1,draw=color1,mark size=\marRad, mark=square*] plot coordinates {(0.47, 0.10)};
\path[fill=color1,draw=color1,mark size=\marRad, mark=square*] plot coordinates {(0.10, 0.33)};
\path[fill=color1,draw=color1,mark size=\marRad, mark=square*] plot coordinates {(0.01, -0.25)};
\path[fill=color1,draw=color1,mark size=\marRad, mark=square*] plot coordinates {(0.01, 0.04)};
\path[fill=color1,draw=color1,mark size=\marRad, mark=square*] plot coordinates {(0.27, -0.11)};
\path[fill=color1,draw=color1,mark size=\marRad, mark=square*] plot coordinates {(0.01, 0.04)};
\path[fill=color1,draw=color1,mark size=\marRad, mark=square*] plot coordinates {(0.50, 0.34)};
\path[fill=color1,draw=color1,mark size=\marRad, mark=square*] plot coordinates {(0.46, 0.13)};
\path[fill=color1,draw=color1,mark size=\marRad, mark=square*] plot coordinates {(0.27, 0.20)};
\path[fill=color1,draw=color1,mark size=\marRad, mark=square*] plot coordinates {(0.01, 0.04)};
\path[fill=color1,draw=color1,mark size=\marRad, mark=square*] plot coordinates {(0.01, 0.04)};
\path[fill=color1,draw=color1,mark size=\marRad, mark=square*] plot coordinates {(0.01, 0.04)};
\path[fill=color1,draw=color1,mark size=\marRad, mark=square*] plot coordinates {(0.01, 0.04)};
\path[fill=color1,draw=color1,mark size=\marRad, mark=square*] plot coordinates {(0.01, 0.04)};
\path[fill=color1,draw=color1,mark size=\marRad, mark=square*] plot coordinates {(0.01, 0.04)};
\path[fill=color1,draw=color1,mark size=\marRad, mark=square*] plot coordinates {(0.01, 0.04)};
\path[fill=color1,draw=color1,mark size=\marRad, mark=square*] plot coordinates {(-0.36, 0.31)};
\path[fill=color1,draw=color1,mark size=\marRad, mark=square*] plot coordinates {(0.01, 0.04)};
\path[fill=color1,draw=color1,mark size=\marRad, mark=square*] plot coordinates {(-0.44, -0.06)};
\path[fill=color1,draw=color1,mark size=\marRad, mark=square*] plot coordinates {(0.56, -0.13)};
\path[fill=color1,draw=color1,mark size=\marRad, mark=square*] plot coordinates {(0.01, 0.04)};
\path[fill=color1,draw=color1,mark size=\marRad, mark=square*] plot coordinates {(0.01, 0.04)};
\path[fill=color1,draw=color1,mark size=\marRad, mark=square*] plot coordinates {(0.11, 0.50)};
\path[fill=color1,draw=color1,mark size=\marRad, mark=square*] plot coordinates {(-0.41, 0.27)};
\path[fill=color1,draw=color1,mark size=\marRad, mark=square*] plot coordinates {(-0.25, 0.19)};
\path[fill=color1,draw=color1,mark size=\marRad, mark=square*] plot coordinates {(0.01, 0.04)};
\path[fill=color1,draw=color1,mark size=\marRad, mark=square*] plot coordinates {(-0.01, -0.55)};
\path[fill=color1,draw=color1,mark size=\marRad, mark=square*] plot coordinates {(0.01, 0.04)};
\path[fill=color1,draw=color1,mark size=\marRad, mark=square*] plot coordinates {(0.26, 0.22)};
\path[fill=color1,draw=color1,mark size=\marRad, mark=square*] plot coordinates {(-0.09, 0.49)};
\path[fill=color1,draw=color1,mark size=\marRad, mark=square*] plot coordinates {(-0.24, -0.12)};
\path[fill=color1,draw=color1,mark size=\marRad, mark=square*] plot coordinates {(0.01, 0.04)};
\path[fill=color1,draw=color1,mark size=\marRad, mark=square*] plot coordinates {(0.01, 0.04)};
\path[fill=color1,draw=color1,mark size=\marRad, mark=square*] plot coordinates {(-0.19, 0.27)};
\path[fill=color1,draw=color1,mark size=\marRad, mark=square*] plot coordinates {(0.07, -0.42)};
\path[fill=color1,draw=color1,mark size=\marRad, mark=square*] plot coordinates {(0.28, -0.08)};
\path[fill=color1,draw=color1,mark size=\marRad, mark=square*] plot coordinates {(0.01, 0.04)};
\path[fill=color1,draw=color1,mark size=\marRad, mark=square*] plot coordinates {(-0.23, 0.21)};
\path[fill=color1,draw=color1,mark size=\marRad, mark=square*] plot coordinates {(0.12, -0.23)};
\path[fill=color1,draw=color1,mark size=\marRad, mark=square*] plot coordinates {(0.01, 0.04)};
\path[fill=color1,draw=color1,mark size=\marRad, mark=square*] plot coordinates {(-0.58, 0.11)};
\path[fill=color1,draw=color1,mark size=\marRad, mark=square*] plot coordinates {(0.01, 0.04)};
\path[fill=color1,draw=color1,mark size=\marRad, mark=square*] plot coordinates {(-0.38, 0.28)};
\path[fill=color1,draw=color1,mark size=\marRad, mark=square*] plot coordinates {(-0.02, -0.44)};
\path[fill=color1,draw=color1,mark size=\marRad, mark=square*] plot coordinates {(0.31, 0.03)};
\path[fill=color1,draw=color1,mark size=\marRad, mark=square*] plot coordinates {(0.01, 0.04)};
\path[fill=color1,draw=color1,mark size=\marRad, mark=square*] plot coordinates {(0.01, 0.04)};
\path[fill=color1,draw=color1,mark size=\marRad, mark=square*] plot coordinates {(0.01, 0.04)};
\path[fill=color1,draw=color1,mark size=\marRad, mark=square*] plot coordinates {(-0.48, 0.02)};
\path[fill=color1,draw=color1,mark size=\marRad, mark=square*] plot coordinates {(0.01, 0.04)};
\path[fill=color1,draw=color1,mark size=\marRad, mark=square*] plot coordinates {(0.18, -0.19)};
\path[fill=color1,draw=color1,mark size=\marRad, mark=square*] plot coordinates {(0.01, 0.04)};
\path[fill=color1,draw=color1,mark size=\marRad, mark=square*] plot coordinates {(0.01, 0.04)};
\path[fill=color1,draw=color1,mark size=\marRad, mark=square*] plot coordinates {(0.01, 0.04)};
\path[fill=color1,draw=color1,mark size=\marRad, mark=square*] plot coordinates {(0.01, 0.04)};
\path[fill=color1,draw=color1,mark size=\marRad, mark=square*] plot coordinates {(0.02, 0.34)};
\path[fill=color1,draw=color1,mark size=\marRad, mark=square*] plot coordinates {(0.01, 0.04)};
\path[fill=color1,draw=color1,mark size=\marRad, mark=square*] plot coordinates {(0.01, 0.04)};
\path[fill=color1,draw=color1,mark size=\marRad, mark=square*] plot coordinates {(0.39, 0.30)};
\path[fill=color1,draw=color1,mark size=\marRad, mark=square*] plot coordinates {(0.01, 0.04)};
\path[fill=color1,draw=color1,mark size=\marRad, mark=square*] plot coordinates {(0.01, 0.04)};
\path[fill=color1,draw=color1,mark size=\marRad, mark=square*] plot coordinates {(0.01, 0.04)};
\path[fill=color1,draw=color1,mark size=\marRad, mark=square*] plot coordinates {(0.01, 0.04)};
\path[fill=color1,draw=color1,mark size=\marRad, mark=square*] plot coordinates {(0.01, 0.04)};
\path[fill=color1,draw=color1,mark size=\marRad, mark=square*] plot coordinates {(-0.27, -0.05)};
\path[fill=color1,draw=color1,mark size=\marRad, mark=square*] plot coordinates {(0.01, 0.04)};
\path[fill=color1,draw=color1,mark size=\marRad, mark=square*] plot coordinates {(0.45, -0.09)};
\path[fill=color1,draw=color1,mark size=\marRad, mark=square*] plot coordinates {(0.01, 0.04)};
\path[fill=color1,draw=color1,mark size=\marRad, mark=square*] plot coordinates {(0.01, 0.04)};
\path[fill=color1,draw=color1,mark size=\marRad, mark=square*] plot coordinates {(0.01, 0.04)};
\path[fill=color1,draw=color1,mark size=\marRad, mark=square*] plot coordinates {(0.42, 0.28)};
\path[fill=color1,draw=color1,mark size=\marRad, mark=square*] plot coordinates {(-0.46, 0.14)};
\path[fill=color1,draw=color1,mark size=\marRad, mark=square*] plot coordinates {(0.47, 0.05)};
\path[fill=color1,draw=color1,mark size=\marRad, mark=square*] plot coordinates {(0.11, 0.32)};
\path[fill=color1,draw=color1,mark size=\marRad, mark=square*] plot coordinates {(0.01, 0.04)};
\path[fill=color1,draw=color1,mark size=\marRad, mark=square*] plot coordinates {(0.01, 0.04)};
\path[fill=color1,draw=color1,mark size=\marRad, mark=square*] plot coordinates {(0.01, 0.04)};
\path[fill=color1,draw=color1,mark size=\marRad, mark=square*] plot coordinates {(0.01, 0.04)};
\path[fill=color1,draw=color1,mark size=\marRad, mark=square*] plot coordinates {(0.11, -0.23)};
\path[fill=color1,draw=color1,mark size=\marRad, mark=square*] plot coordinates {(0.01, 0.04)};
\path[fill=color1,draw=color1,mark size=\marRad, mark=square*] plot coordinates {(0.01, 0.04)};
\path[fill=color1,draw=color1,mark size=\marRad, mark=square*] plot coordinates {(0.01, 0.04)};
\path[fill=color1,draw=color1,mark size=\marRad, mark=square*] plot coordinates {(0.01, 0.04)};
\path[fill=color1,draw=color1,mark size=\marRad, mark=square*] plot coordinates {(0.01, 0.04)};
\path[fill=color1,draw=color1,mark size=\marRad, mark=square*] plot coordinates {(-0.28, 0.13)};
\path[fill=color1,draw=color1,mark size=\marRad, mark=square*] plot coordinates {(0.30, 0.13)};
\path[fill=color1,draw=color1,mark size=\marRad, mark=square*] plot coordinates {(0.49, -0.04)};
\path[fill=color1,draw=color1,mark size=\marRad, mark=square*] plot coordinates {(0.01, 0.04)};
\path[fill=color1,draw=color1,mark size=\marRad, mark=square*] plot coordinates {(0.01, 0.04)};
\path[fill=color1,draw=color1,mark size=\marRad, mark=square*] plot coordinates {(0.01, 0.04)};
\path[fill=color1,draw=color1,mark size=\marRad, mark=square*] plot coordinates {(0.24, 0.46)};
\path[fill=color1,draw=color1,mark size=\marRad, mark=square*] plot coordinates {(0.01, 0.04)};
\path[fill=color1,draw=color1,mark size=\marRad, mark=square*] plot coordinates {(0.01, 0.04)};
\path[fill=color1,draw=color1,mark size=\marRad, mark=square*] plot coordinates {(0.01, 0.04)};
\path[fill=color1,draw=color1,mark size=\marRad, mark=square*] plot coordinates {(0.01, 0.04)};
\path[fill=color1,draw=color1,mark size=\marRad, mark=square*] plot coordinates {(0.01, 0.04)};
\path[fill=color1,draw=color1,mark size=\marRad, mark=square*] plot coordinates {(0.01, 0.04)};
\path[fill=color1,draw=color1,mark size=\marRad, mark=square*] plot coordinates {(0.28, -0.07)};
\path[fill=color1,draw=color1,mark size=\marRad, mark=square*] plot coordinates {(0.01, 0.04)};
\path[fill=color1,draw=color1,mark size=\marRad, mark=square*] plot coordinates {(0.01, 0.04)};
\path[fill=color1,draw=color1,mark size=\marRad, mark=square*] plot coordinates {(0.01, 0.04)};
\path[fill=color1,draw=color1,mark size=\marRad, mark=square*] plot coordinates {(0.01, 0.04)};
\path[fill=color1,draw=color1,mark size=\marRad, mark=square*] plot coordinates {(0.01, 0.04)};
\path[fill=color1,draw=color1,mark size=\marRad, mark=square*] plot coordinates {(-0.17, 0.29)};
\path[fill=color1,draw=color1,mark size=\marRad, mark=square*] plot coordinates {(0.35, 0.36)};
\path[fill=color1,draw=color1,mark size=\marRad, mark=square*] plot coordinates {(0.46, -0.11)};
\path[fill=color1,draw=color1,mark size=\marRad, mark=square*] plot coordinates {(-0.48, 0.06)};
\path[fill=color1,draw=color1,mark size=\marRad, mark=square*] plot coordinates {(0.01, 0.04)};
\path[fill=color1,draw=color1,mark size=\marRad, mark=square*] plot coordinates {(0.30, -0.04)};
\path[fill=color1,draw=color1,mark size=\marRad, mark=square*] plot coordinates {(0.01, 0.04)};
\path[fill=color1,draw=color1,mark size=\marRad, mark=square*] plot coordinates {(0.29, 0.17)};
\path[fill=color1,draw=color1,mark size=\marRad, mark=square*] plot coordinates {(0.01, 0.04)};
\path[fill=color1,draw=color1,mark size=\marRad, mark=square*] plot coordinates {(0.01, 0.04)};
\path[fill=color1,draw=color1,mark size=\marRad, mark=square*] plot coordinates {(0.01, 0.04)};
\path[fill=color1,draw=color1,mark size=\marRad, mark=square*] plot coordinates {(0.01, 0.04)};
\path[fill=color1,draw=color1,mark size=\marRad, mark=square*] plot coordinates {(0.01, 0.04)};
\path[fill=color1,draw=color1,mark size=\marRad, mark=square*] plot coordinates {(0.01, 0.04)};
\path[fill=color1,draw=color1,mark size=\marRad, mark=square*] plot coordinates {(0.01, 0.04)};
\path[fill=color1,draw=color1,mark size=\marRad, mark=square*] plot coordinates {(0.01, 0.04)};
\path[fill=color1,draw=color1,mark size=\marRad, mark=square*] plot coordinates {(-0.05, -0.53)};
\path[fill=color1,draw=color1,mark size=\marRad, mark=square*] plot coordinates {(0.01, 0.04)};
\path[fill=color1,draw=color1,mark size=\marRad, mark=square*] plot coordinates {(0.46, -0.10)};
\path[fill=color1,draw=color1,mark size=\marRad, mark=square*] plot coordinates {(0.01, 0.04)};
\path[fill=color1,draw=color1,mark size=\marRad, mark=square*] plot coordinates {(0.01, 0.04)};
\path[fill=color1,draw=color1,mark size=\marRad, mark=square*] plot coordinates {(-0.29, 0.10)};
\path[fill=color1,draw=color1,mark size=\marRad, mark=square*] plot coordinates {(0.09, -0.24)};
\path[fill=color1,draw=color1,mark size=\marRad, mark=square*] plot coordinates {(0.01, 0.04)};
\path[fill=color1,draw=color1,mark size=\marRad, mark=square*] plot coordinates {(0.01, 0.04)};
\path[fill=color1,draw=color1,mark size=\marRad, mark=square*] plot coordinates {(0.01, 0.04)};
\path[fill=color1,draw=color1,mark size=\marRad, mark=square*] plot coordinates {(0.01, 0.04)};
\path[fill=color1,draw=color1,mark size=\marRad, mark=square*] plot coordinates {(0.01, 0.04)};
\path[fill=color1,draw=color1,mark size=\marRad, mark=square*] plot coordinates {(0.00, 0.34)};
\path[fill=color1,draw=color1,mark size=\marRad, mark=square*] plot coordinates {(0.01, 0.04)};
\path[fill=color1,draw=color1,mark size=\marRad, mark=square*] plot coordinates {(0.01, 0.04)};
\path[fill=color1,draw=color1,mark size=\marRad, mark=square*] plot coordinates {(0.01, 0.04)};
\path[fill=color1,draw=color1,mark size=\marRad, mark=square*] plot coordinates {(0.01, 0.04)};
\path[fill=color1,draw=color1,mark size=\marRad, mark=square*] plot coordinates {(0.01, 0.04)};
\path[fill=color1,draw=color1,mark size=\marRad, mark=square*] plot coordinates {(0.01, 0.04)};
\path[fill=color1,draw=color1,mark size=\marRad, mark=square*] plot coordinates {(0.06, 0.34)};
\path[fill=color1,draw=color1,mark size=\marRad, mark=square*] plot coordinates {(0.01, 0.04)};
\path[fill=color1,draw=color1,mark size=\marRad, mark=square*] plot coordinates {(-0.13, 0.31)};
\path[fill=color1,draw=color1,mark size=\marRad, mark=square*] plot coordinates {(0.01, 0.04)};
\path[fill=color1,draw=color1,mark size=\marRad, mark=square*] plot coordinates {(0.01, 0.04)};
\path[fill=color1,draw=color1,mark size=\marRad, mark=square*] plot coordinates {(-0.44, 0.19)};
\path[fill=color1,draw=color1,mark size=\marRad, mark=square*] plot coordinates {(0.01, 0.04)};
\path[fill=color1,draw=color1,mark size=\marRad, mark=square*] plot coordinates {(0.01, 0.04)};
\path[fill=color1,draw=color1,mark size=\marRad, mark=square*] plot coordinates {(0.31, 0.09)};
\path[fill=color1,draw=color1,mark size=\marRad, mark=square*] plot coordinates {(0.01, 0.04)};
\path[fill=color1,draw=color1,mark size=\marRad, mark=square*] plot coordinates {(0.01, 0.04)};
\path[fill=color1,draw=color1,mark size=\marRad, mark=square*] plot coordinates {(0.01, 0.04)};
\path[fill=color1,draw=color1,mark size=\marRad, mark=square*] plot coordinates {(0.14, 0.31)};
\path[fill=color1,draw=color1,mark size=\marRad, mark=square*] plot coordinates {(0.01, 0.04)};
\path[fill=color1,draw=color1,mark size=\marRad, mark=square*] plot coordinates {(0.29, 0.16)};
\path[fill=color1,draw=color1,mark size=\marRad, mark=square*] plot coordinates {(0.18, 0.29)};
\path[fill=color1,draw=color1,mark size=\marRad, mark=square*] plot coordinates {(0.10, -0.39)};
\path[fill=color1,draw=color1,mark size=\marRad, mark=square*] plot coordinates {(0.01, 0.04)};
\path[fill=color1,draw=color1,mark size=\marRad, mark=square*] plot coordinates {(0.01, 0.04)};
\path[fill=color1,draw=color1,mark size=\marRad, mark=square*] plot coordinates {(0.20, 0.47)};
\path[fill=color1,draw=color1,mark size=\marRad, mark=square*] plot coordinates {(0.01, 0.04)};
\path[fill=color1,draw=color1,mark size=\marRad, mark=square*] plot coordinates {(0.01, 0.04)};
\path[fill=color1,draw=color1,mark size=\marRad, mark=square*] plot coordinates {(0.01, 0.04)};
\path[fill=color1,draw=color1,mark size=\marRad, mark=square*] plot coordinates {(0.01, 0.04)};
\path[fill=color1,draw=color1,mark size=\marRad, mark=square*] plot coordinates {(0.01, 0.04)};
\path[fill=color1,draw=color1,mark size=\marRad, mark=square*] plot coordinates {(0.01, 0.04)};
\path[fill=color1,draw=color1,mark size=\marRad, mark=square*] plot coordinates {(-0.27, -0.01)};
\path[fill=color1,draw=color1,mark size=\marRad, mark=square*] plot coordinates {(0.21, -0.17)};
\path[fill=color1,draw=color1,mark size=\marRad, mark=square*] plot coordinates {(0.01, 0.04)};
\path[fill=color1,draw=color1,mark size=\marRad, mark=square*] plot coordinates {(-0.04, -0.53)};
\path[fill=color1,draw=color1,mark size=\marRad, mark=square*] plot coordinates {(0.01, 0.04)};
\path[fill=color1,draw=color1,mark size=\marRad, mark=square*] plot coordinates {(0.01, 0.04)};
\path[fill=color1,draw=color1,mark size=\marRad, mark=square*] plot coordinates {(0.01, 0.04)};
\path[fill=color1,draw=color1,mark size=\marRad, mark=square*] plot coordinates {(0.31, 0.06)};
\path[fill=color1,draw=color1,mark size=\marRad, mark=square*] plot coordinates {(0.01, 0.04)};
\path[fill=color1,draw=color1,mark size=\marRad, mark=square*] plot coordinates {(0.01, 0.04)};
\path[fill=color1,draw=color1,mark size=\marRad, mark=square*] plot coordinates {(-0.13, 0.49)};
\path[fill=color1,draw=color1,mark size=\marRad, mark=square*] plot coordinates {(0.42, -0.15)};
\path[fill=color1,draw=color1,mark size=\marRad, mark=square*] plot coordinates {(0.30, -0.03)};
\path[fill=color1,draw=color1,mark size=\marRad, mark=square*] plot coordinates {(0.01, 0.04)};
\path[fill=color1,draw=color1,mark size=\marRad, mark=square*] plot coordinates {(0.01, 0.04)};
\path[fill=color1,draw=color1,mark size=\marRad, mark=square*] plot coordinates {(-0.09, 0.33)};
\path[fill=color1,draw=color1,mark size=\marRad, mark=square*] plot coordinates {(0.01, 0.04)};
\path[fill=color1,draw=color1,mark size=\marRad, mark=square*] plot coordinates {(0.01, 0.04)};
\path[fill=color1,draw=color1,mark size=\marRad, mark=square*] plot coordinates {(0.18, -0.39)};
\path[fill=color1,draw=color1,mark size=\marRad, mark=square*] plot coordinates {(-0.14, 0.30)};
\path[fill=color1,draw=color1,mark size=\marRad, mark=square*] plot coordinates {(-0.15, -0.40)};
\path[fill=color1,draw=color1,mark size=\marRad, mark=square*] plot coordinates {(0.31, -0.01)};
\path[fill=color1,draw=color1,mark size=\marRad, mark=square*] plot coordinates {(0.01, 0.04)};
\path[fill=color1,draw=color1,mark size=\marRad, mark=square*] plot coordinates {(0.01, 0.04)};
\path[fill=color1,draw=color1,mark size=\marRad, mark=square*] plot coordinates {(0.01, 0.04)};
\path[fill=color1,draw=color1,mark size=\marRad, mark=square*] plot coordinates {(0.01, 0.04)};
\path[fill=color1,draw=color1,mark size=\marRad, mark=square*] plot coordinates {(0.14, 0.31)};
\path[fill=color1,draw=color1,mark size=\marRad, mark=square*] plot coordinates {(0.20, 0.27)};
\path[fill=color1,draw=color1,mark size=\marRad, mark=square*] plot coordinates {(0.01, 0.04)};
\path[fill=color1,draw=color1,mark size=\marRad, mark=square*] plot coordinates {(0.00, 0.32)};
\path[fill=color1,draw=color1,mark size=\marRad, mark=square*] plot coordinates {(0.01, 0.04)};
\path[fill=color1,draw=color1,mark size=\marRad, mark=square*] plot coordinates {(0.25, 0.20)};
\path[fill=color1,draw=color1,mark size=\marRad, mark=square*] plot coordinates {(0.01, 0.04)};
\path[fill=color1,draw=color1,mark size=\marRad, mark=square*] plot coordinates {(0.01, 0.04)};
\path[fill=color1,draw=color1,mark size=\marRad, mark=square*] plot coordinates {(0.01, 0.04)};
\path[fill=color1,draw=color1,mark size=\marRad, mark=square*] plot coordinates {(0.04, 0.34)};
\path[fill=color1,draw=color1,mark size=\marRad, mark=square*] plot coordinates {(0.28, -0.02)};
\path[fill=color1,draw=color1,mark size=\marRad, mark=square*] plot coordinates {(0.01, 0.04)};
\path[fill=color1,draw=color1,mark size=\marRad, mark=square*] plot coordinates {(0.01, 0.04)};
\path[fill=color1,draw=color1,mark size=\marRad, mark=square*] plot coordinates {(0.01, 0.04)};
\path[fill=color1,draw=color1,mark size=\marRad, mark=square*] plot coordinates {(-0.04, -0.24)};
\path[fill=color1,draw=color1,mark size=\marRad, mark=square*] plot coordinates {(0.01, 0.04)};
\path[fill=color1,draw=color1,mark size=\marRad, mark=square*] plot coordinates {(0.01, 0.04)};
\path[fill=color1,draw=color1,mark size=\marRad, mark=square*] plot coordinates {(0.01, 0.04)};
\path[fill=color1,draw=color1,mark size=\marRad, mark=square*] plot coordinates {(-0.24, -0.07)};
\path[fill=color1,draw=color1,mark size=\marRad, mark=square*] plot coordinates {(0.24, 0.21)};
\path[fill=color1,draw=color1,mark size=\marRad, mark=square*] plot coordinates {(0.28, 0.14)};
\path[fill=color1,draw=color1,mark size=\marRad, mark=square*] plot coordinates {(-0.13, 0.28)};
\path[fill=color1,draw=color1,mark size=\marRad, mark=square*] plot coordinates {(0.01, 0.04)};
\path[fill=color1,draw=color1,mark size=\marRad, mark=square*] plot coordinates {(0.01, 0.04)};
\path[fill=color1,draw=color1,mark size=\marRad, mark=square*] plot coordinates {(0.34, -0.27)};
\path[fill=color1,draw=color1,mark size=\marRad, mark=square*] plot coordinates {(0.16, 0.30)};
\path[fill=color1,draw=color1,mark size=\marRad, mark=square*] plot coordinates {(0.01, 0.04)};
\path[fill=color1,draw=color1,mark size=\marRad, mark=square*] plot coordinates {(0.01, 0.04)};
\path[fill=color1,draw=color1,mark size=\marRad, mark=square*] plot coordinates {(0.01, 0.04)};
\path[fill=color1,draw=color1,mark size=\marRad, mark=square*] plot coordinates {(0.01, 0.04)};
\path[fill=color1,draw=color1,mark size=\marRad, mark=square*] plot coordinates {(0.01, 0.04)};
\path[fill=color1,draw=color1,mark size=\marRad, mark=square*] plot coordinates {(-0.29, 0.06)};
\path[fill=color1,draw=color1,mark size=\marRad, mark=square*] plot coordinates {(0.01, 0.04)};
\path[fill=color1,draw=color1,mark size=\marRad, mark=square*] plot coordinates {(0.20, 0.27)};
\path[fill=color1,draw=color1,mark size=\marRad, mark=square*] plot coordinates {(0.01, 0.04)};
\path[fill=color1,draw=color1,mark size=\marRad, mark=square*] plot coordinates {(0.01, 0.04)};
\path[fill=color1,draw=color1,mark size=\marRad, mark=square*] plot coordinates {(0.01, 0.04)};
\path[fill=color1,draw=color1,mark size=\marRad, mark=square*] plot coordinates {(0.01, 0.04)};
\path[fill=color1,draw=color1,mark size=\marRad, mark=square*] plot coordinates {(0.01, 0.04)};
\path[fill=color1,draw=color1,mark size=\marRad, mark=square*] plot coordinates {(0.16, -0.21)};
\path[fill=color1,draw=color1,mark size=\marRad, mark=square*] plot coordinates {(0.01, 0.04)};
\path[fill=color1,draw=color1,mark size=\marRad, mark=square*] plot coordinates {(0.26, -0.09)};
\path[fill=color1,draw=color1,mark size=\marRad, mark=square*] plot coordinates {(0.07, 0.32)};
\path[fill=color1,draw=color1,mark size=\marRad, mark=square*] plot coordinates {(0.01, 0.04)};
\path[fill=color1,draw=color1,mark size=\marRad, mark=square*] plot coordinates {(0.01, 0.04)};
\path[fill=color1,draw=color1,mark size=\marRad, mark=square*] plot coordinates {(0.01, 0.04)};
\path[fill=color1,draw=color1,mark size=\marRad, mark=square*] plot coordinates {(0.01, 0.04)};
\path[fill=color1,draw=color1,mark size=\marRad, mark=square*] plot coordinates {(0.01, 0.04)};
\path[fill=color1,draw=color1,mark size=\marRad, mark=square*] plot coordinates {(0.01, 0.04)};
\path[fill=color1,draw=color1,mark size=\marRad, mark=square*] plot coordinates {(0.01, 0.04)};
\path[fill=color1,draw=color1,mark size=\marRad, mark=square*] plot coordinates {(0.01, 0.04)};
\path[fill=color1,draw=color1,mark size=\marRad, mark=square*] plot coordinates {(0.01, 0.04)};
\path[fill=color1,draw=color1,mark size=\marRad, mark=square*] plot coordinates {(-0.24, 0.22)};
\path[fill=color1,draw=color1,mark size=\marRad, mark=square*] plot coordinates {(-0.50, 0.32)};
\path[fill=color1,draw=color1,mark size=\marRad, mark=square*] plot coordinates {(0.01, 0.04)};
\path[fill=color1,draw=color1,mark size=\marRad, mark=square*] plot coordinates {(-0.45, 0.10)};
\path[fill=color1,draw=color1,mark size=\marRad, mark=square*] plot coordinates {(-0.29, 0.06)};
\path[fill=color1,draw=color1,mark size=\marRad, mark=square*] plot coordinates {(0.31, 0.08)};
\path[fill=color1,draw=color1,mark size=\marRad, mark=square*] plot coordinates {(0.24, -0.14)};
\path[fill=color1,draw=color1,mark size=\marRad, mark=square*] plot coordinates {(0.01, 0.04)};
\path[fill=color1,draw=color1,mark size=\marRad, mark=square*] plot coordinates {(0.01, 0.04)};
\path[fill=color1,draw=color1,mark size=\marRad, mark=square*] plot coordinates {(-0.07, -0.21)};
\path[fill=color1,draw=color1,mark size=\marRad, mark=square*] plot coordinates {(0.01, 0.04)};
\path[fill=color1,draw=color1,mark size=\marRad, mark=square*] plot coordinates {(0.21, -0.15)};
\path[fill=color1,draw=color1,mark size=\marRad, mark=square*] plot coordinates {(0.01, 0.04)};
\path[fill=color1,draw=color1,mark size=\marRad, mark=square*] plot coordinates {(0.01, 0.04)};
\path[fill=color1,draw=color1,mark size=\marRad, mark=square*] plot coordinates {(0.01, 0.04)};
\path[fill=color1,draw=color1,mark size=\marRad, mark=square*] plot coordinates {(0.01, 0.04)};
\path[fill=color1,draw=color1,mark size=\marRad, mark=square*] plot coordinates {(0.01, 0.04)};
\path[fill=color1,draw=color1,mark size=\marRad, mark=square*] plot coordinates {(0.01, 0.04)};
\path[fill=color1,draw=color1,mark size=\marRad, mark=square*] plot coordinates {(0.01, 0.04)};
\path[fill=color1,draw=color1,mark size=\marRad, mark=square*] plot coordinates {(0.01, 0.04)};
\path[fill=color1,draw=color1,mark size=\marRad, mark=square*] plot coordinates {(0.01, 0.04)};
\path[fill=color1,draw=color1,mark size=\marRad, mark=square*] plot coordinates {(0.01, 0.04)};
\path[fill=color1,draw=color1,mark size=\marRad, mark=square*] plot coordinates {(0.01, 0.04)};
\path[fill=color1,draw=color1,mark size=\marRad, mark=square*] plot coordinates {(0.01, 0.04)};
\path[fill=color1,draw=color1,mark size=\marRad, mark=square*] plot coordinates {(0.22, 0.24)};
\path[fill=color1,draw=color1,mark size=\marRad, mark=square*] plot coordinates {(0.01, 0.04)};
\path[fill=color1,draw=color1,mark size=\marRad, mark=square*] plot coordinates {(0.01, 0.04)};
\path[fill=color1,draw=color1,mark size=\marRad, mark=square*] plot coordinates {(-0.10, 0.32)};
\path[fill=color1,draw=color1,mark size=\marRad, mark=square*] plot coordinates {(0.01, 0.04)};
\path[fill=color1,draw=color1,mark size=\marRad, mark=square*] plot coordinates {(0.18, 0.48)};
\path[fill=color1,draw=color1,mark size=\marRad, mark=square*] plot coordinates {(0.01, 0.04)};
\path[fill=color1,draw=color1,mark size=\marRad, mark=square*] plot coordinates {(0.01, 0.04)};
\path[fill=color1,draw=color1,mark size=\marRad, mark=square*] plot coordinates {(0.01, 0.04)};
\path[fill=color1,draw=color1,mark size=\marRad, mark=square*] plot coordinates {(0.25, 0.44)};
\path[fill=color1,draw=color1,mark size=\marRad, mark=square*] plot coordinates {(0.01, 0.04)};
\path[fill=color1,draw=color1,mark size=\marRad, mark=square*] plot coordinates {(0.01, 0.04)};
\path[fill=color1,draw=color1,mark size=\marRad, mark=square*] plot coordinates {(0.01, 0.04)};
\path[fill=color1,draw=color1,mark size=\marRad, mark=square*] plot coordinates {(-0.10, -0.21)};
\path[fill=color1,draw=color1,mark size=\marRad, mark=square*] plot coordinates {(0.01, -0.25)};
\path[fill=color1,draw=color1,mark size=\marRad, mark=square*] plot coordinates {(0.01, 0.04)};
\path[fill=color1,draw=color1,mark size=\marRad, mark=square*] plot coordinates {(0.01, 0.04)};
\path[fill=color1,draw=color1,mark size=\marRad, mark=square*] plot coordinates {(0.01, 0.04)};
\path[fill=color1,draw=color1,mark size=\marRad, mark=square*] plot coordinates {(0.01, 0.04)};
\path[fill=color1,draw=color1,mark size=\marRad, mark=square*] plot coordinates {(0.01, 0.04)};
\path[fill=color1,draw=color1,mark size=\marRad, mark=square*] plot coordinates {(0.27, 0.17)};
\path[fill=color1,draw=color1,mark size=\marRad, mark=square*] plot coordinates {(-0.27, 0.16)};
\path[fill=color1,draw=color1,mark size=\marRad, mark=square*] plot coordinates {(0.01, 0.04)};
\path[fill=color1,draw=color1,mark size=\marRad, mark=square*] plot coordinates {(0.29, -0.07)};
\path[fill=color1,draw=color1,mark size=\marRad, mark=square*] plot coordinates {(0.01, 0.04)};
\path[fill=color1,draw=color1,mark size=\marRad, mark=square*] plot coordinates {(0.05, -0.25)};
\path[fill=color1,draw=color1,mark size=\marRad, mark=square*] plot coordinates {(0.01, 0.04)};
\path[fill=color1,draw=color1,mark size=\marRad, mark=square*] plot coordinates {(0.01, 0.04)};
\path[fill=color1,draw=color1,mark size=\marRad, mark=square*] plot coordinates {(-0.34, 0.37)};
\path[fill=color1,draw=color1,mark size=\marRad, mark=square*] plot coordinates {(0.01, 0.04)};
\path[fill=color1,draw=color1,mark size=\marRad, mark=square*] plot coordinates {(0.12, 0.32)};
\path[fill=color1,draw=color1,mark size=\marRad, mark=square*] plot coordinates {(0.47, 0.04)};
\path[fill=color1,draw=color1,mark size=\marRad, mark=square*] plot coordinates {(0.01, 0.04)};
\path[fill=color1,draw=color1,mark size=\marRad, mark=square*] plot coordinates {(-0.27, -0.58)};
\path[fill=color1,draw=color1,mark size=\marRad, mark=square*] plot coordinates {(-0.25, -0.05)};
\path[fill=color1,draw=color1,mark size=\marRad, mark=square*] plot coordinates {(0.01, 0.04)};
\path[fill=color1,draw=color1,mark size=\marRad, mark=square*] plot coordinates {(0.01, 0.04)};
\path[fill=color1,draw=color1,mark size=\marRad, mark=square*] plot coordinates {(0.01, 0.04)};
\path[fill=color1,draw=color1,mark size=\marRad, mark=square*] plot coordinates {(0.01, 0.04)};
\path[fill=color1,draw=color1,mark size=\marRad, mark=square*] plot coordinates {(0.01, 0.04)};
\path[fill=color1,draw=color1,mark size=\marRad, mark=square*] plot coordinates {(0.01, 0.04)};
\path[fill=color1,draw=color1,mark size=\marRad, mark=square*] plot coordinates {(0.01, 0.04)};
\path[fill=color1,draw=color1,mark size=\marRad, mark=square*] plot coordinates {(0.01, 0.04)};
\path[fill=color1,draw=color1,mark size=\marRad, mark=square*] plot coordinates {(0.01, 0.04)};
\path[fill=color1,draw=color1,mark size=\marRad, mark=square*] plot coordinates {(-0.31, 0.51)};
\path[fill=color1,draw=color1,mark size=\marRad, mark=square*] plot coordinates {(0.01, 0.04)};
\path[fill=color1,draw=color1,mark size=\marRad, mark=square*] plot coordinates {(0.01, 0.04)};
\path[fill=color1,draw=color1,mark size=\marRad, mark=square*] plot coordinates {(0.23, 0.25)};
\path[fill=color1,draw=color1,mark size=\marRad, mark=square*] plot coordinates {(0.01, 0.04)};
\path[fill=color1,draw=color1,mark size=\marRad, mark=square*] plot coordinates {(0.01, 0.04)};
\path[fill=color1,draw=color1,mark size=\marRad, mark=square*] plot coordinates {(0.01, 0.04)};
\path[fill=color1,draw=color1,mark size=\marRad, mark=square*] plot coordinates {(0.01, 0.04)};
\path[fill=color1,draw=color1,mark size=\marRad, mark=square*] plot coordinates {(0.01, 0.04)};
\path[fill=color1,draw=color1,mark size=\marRad, mark=square*] plot coordinates {(0.01, 0.04)};
\path[fill=color1,draw=color1,mark size=\marRad, mark=square*] plot coordinates {(0.01, 0.04)};
\path[fill=color1,draw=color1,mark size=\marRad, mark=square*] plot coordinates {(-0.17, 0.28)};
\path[fill=color1,draw=color1,mark size=\marRad, mark=square*] plot coordinates {(0.01, 0.04)};
\path[fill=color1,draw=color1,mark size=\marRad, mark=square*] plot coordinates {(0.01, 0.04)};
\path[fill=color1,draw=color1,mark size=\marRad, mark=square*] plot coordinates {(0.01, 0.32)};
\path[fill=color1,draw=color1,mark size=\marRad, mark=square*] plot coordinates {(0.01, 0.04)};
\path[fill=color1,draw=color1,mark size=\marRad, mark=square*] plot coordinates {(0.01, 0.04)};
\path[fill=color1,draw=color1,mark size=\marRad, mark=square*] plot coordinates {(0.01, 0.04)};
\path[fill=color1,draw=color1,mark size=\marRad, mark=square*] plot coordinates {(0.01, 0.04)};
\path[fill=color1,draw=color1,mark size=\marRad, mark=square*] plot coordinates {(0.01, 0.04)};
\path[fill=color1,draw=color1,mark size=\marRad, mark=square*] plot coordinates {(0.01, 0.04)};
\path[fill=color1,draw=color1,mark size=\marRad, mark=square*] plot coordinates {(0.01, 0.04)};
\path[fill=color1,draw=color1,mark size=\marRad, mark=square*] plot coordinates {(0.01, 0.04)};
\path[fill=color1,draw=color1,mark size=\marRad, mark=square*] plot coordinates {(-0.40, 0.27)};
\path[fill=color1,draw=color1,mark size=\marRad, mark=square*] plot coordinates {(0.01, 0.04)};
\path[fill=color1,draw=color1,mark size=\marRad, mark=square*] plot coordinates {(0.01, 0.04)};
\path[fill=color1,draw=color1,mark size=\marRad, mark=square*] plot coordinates {(-0.22, 0.23)};
\path[fill=color1,draw=color1,mark size=\marRad, mark=square*] plot coordinates {(0.01, 0.04)};
\path[fill=color1,draw=color1,mark size=\marRad, mark=square*] plot coordinates {(0.01, 0.04)};
\path[fill=color1,draw=color1,mark size=\marRad, mark=square*] plot coordinates {(0.01, 0.04)};
\path[fill=color1,draw=color1,mark size=\marRad, mark=square*] plot coordinates {(-0.06, 0.34)};
\path[fill=color1,draw=color1,mark size=\marRad, mark=square*] plot coordinates {(0.01, 0.04)};
\path[fill=color1,draw=color1,mark size=\marRad, mark=square*] plot coordinates {(0.31, 0.01)};
\path[fill=color1,draw=color1,mark size=\marRad, mark=square*] plot coordinates {(0.33, -0.27)};
\path[fill=color1,draw=color1,mark size=\marRad, mark=square*] plot coordinates {(0.01, 0.04)};
\path[fill=color1,draw=color1,mark size=\marRad, mark=square*] plot coordinates {(0.01, 0.04)};
\path[fill=color1,draw=color1,mark size=\marRad, mark=square*] plot coordinates {(0.01, 0.04)};
\path[fill=color1,draw=color1,mark size=\marRad, mark=square*] plot coordinates {(0.01, 0.04)};
\path[fill=color1,draw=color1,mark size=\marRad, mark=square*] plot coordinates {(0.42, 0.42)};
\path[fill=color1,draw=color1,mark size=\marRad, mark=square*] plot coordinates {(0.01, 0.04)};
\path[fill=color1,draw=color1,mark size=\marRad, mark=square*] plot coordinates {(0.44, -0.15)};
\path[fill=color1,draw=color1,mark size=\marRad, mark=square*] plot coordinates {(-0.43, 0.25)};
\path[fill=color1,draw=color1,mark size=\marRad, mark=square*] plot coordinates {(0.28, 0.18)};
\path[fill=color1,draw=color1,mark size=\marRad, mark=square*] plot coordinates {(0.01, 0.04)};
\path[fill=color1,draw=color1,mark size=\marRad, mark=square*] plot coordinates {(-0.26, 0.17)};
\path[fill=color1,draw=color1,mark size=\marRad, mark=square*] plot coordinates {(0.46, 0.19)};
\path[fill=color1,draw=color1,mark size=\marRad, mark=square*] plot coordinates {(0.01, 0.04)};
\path[fill=color1,draw=color1,mark size=\marRad, mark=square*] plot coordinates {(0.49, 0.10)};
\path[fill=color1,draw=color1,mark size=\marRad, mark=square*] plot coordinates {(0.01, 0.04)};
\path[fill=color1,draw=color1,mark size=\marRad, mark=square*] plot coordinates {(0.01, 0.04)};
\path[fill=color1,draw=color1,mark size=\marRad, mark=square*] plot coordinates {(0.01, 0.04)};
\path[fill=color1,draw=color1,mark size=\marRad, mark=square*] plot coordinates {(0.06, -0.24)};
\path[fill=color1,draw=color1,mark size=\marRad, mark=square*] plot coordinates {(0.01, 0.04)};
\path[fill=color1,draw=color1,mark size=\marRad, mark=square*] plot coordinates {(0.01, 0.04)};
\path[fill=color1,draw=color1,mark size=\marRad, mark=square*] plot coordinates {(-0.02, -0.25)};
\path[fill=color1,draw=color1,mark size=\marRad, mark=square*] plot coordinates {(0.01, 0.04)};
\path[fill=color1,draw=color1,mark size=\marRad, mark=square*] plot coordinates {(0.01, 0.04)};
\path[fill=color1,draw=color1,mark size=\marRad, mark=square*] plot coordinates {(0.01, 0.04)};
\path[fill=color1,draw=color1,mark size=\marRad, mark=square*] plot coordinates {(0.01, 0.04)};
\path[fill=color1,draw=color1,mark size=\marRad, mark=square*] plot coordinates {(0.55, 0.24)};
\path[fill=color1,draw=color1,mark size=\marRad, mark=square*] plot coordinates {(-0.11, 0.32)};
\path[fill=color1,draw=color1,mark size=\marRad, mark=square*] plot coordinates {(-0.33, 0.38)};
\path[fill=color1,draw=color1,mark size=\marRad, mark=square*] plot coordinates {(0.01, 0.04)};
\path[fill=color1,draw=color1,mark size=\marRad, mark=square*] plot coordinates {(-0.05, 0.34)};
\path[fill=color1,draw=color1,mark size=\marRad, mark=square*] plot coordinates {(0.01, 0.04)};
\path[fill=color1,draw=color1,mark size=\marRad, mark=square*] plot coordinates {(0.01, 0.04)};
\path[fill=color1,draw=color1,mark size=\marRad, mark=square*] plot coordinates {(0.01, 0.04)};
\path[fill=color1,draw=color1,mark size=\marRad, mark=square*] plot coordinates {(-0.27, -0.58)};
\path[fill=color1,draw=color1,mark size=\marRad, mark=square*] plot coordinates {(0.01, 0.04)};
\path[fill=color1,draw=color1,mark size=\marRad, mark=square*] plot coordinates {(0.01, 0.04)};
\path[fill=color1,draw=color1,mark size=\marRad, mark=square*] plot coordinates {(0.01, 0.04)};
\path[fill=color1,draw=color1,mark size=\marRad, mark=square*] plot coordinates {(0.01, 0.04)};
\path[fill=color1,draw=color1,mark size=\marRad, mark=square*] plot coordinates {(0.01, 0.04)};
\path[fill=color1,draw=color1,mark size=\marRad, mark=square*] plot coordinates {(0.01, 0.04)};
\path[fill=color1,draw=color1,mark size=\marRad, mark=square*] plot coordinates {(0.01, 0.04)};
\path[fill=color1,draw=color1,mark size=\marRad, mark=square*] plot coordinates {(0.01, 0.04)};
\path[fill=color1,draw=color1,mark size=\marRad, mark=square*] plot coordinates {(0.01, 0.04)};
\path[fill=color1,draw=color1,mark size=\marRad, mark=square*] plot coordinates {(0.30, 0.05)};
\path[fill=color1,draw=color1,mark size=\marRad, mark=square*] plot coordinates {(-0.03, 0.34)};
\path[fill=color1,draw=color1,mark size=\marRad, mark=square*] plot coordinates {(0.01, 0.04)};
\path[fill=color1,draw=color1,mark size=\marRad, mark=square*] plot coordinates {(0.01, 0.04)};
\path[fill=color1,draw=color1,mark size=\marRad, mark=square*] plot coordinates {(0.01, 0.04)};
\path[fill=color1,draw=color1,mark size=\marRad, mark=square*] plot coordinates {(0.52, -0.24)};
\path[fill=color1,draw=color1,mark size=\marRad, mark=square*] plot coordinates {(-0.26, -0.00)};
\path[fill=color1,draw=color1,mark size=\marRad, mark=square*] plot coordinates {(0.01, 0.04)};
\path[fill=color1,draw=color1,mark size=\marRad, mark=square*] plot coordinates {(0.01, 0.04)};
\path[fill=color1,draw=color1,mark size=\marRad, mark=square*] plot coordinates {(0.01, 0.04)};
\path[fill=color1,draw=color1,mark size=\marRad, mark=square*] plot coordinates {(0.30, 0.15)};
\path[fill=color1,draw=color1,mark size=\marRad, mark=square*] plot coordinates {(0.01, 0.04)};
\path[fill=color1,draw=color1,mark size=\marRad, mark=square*] plot coordinates {(0.01, 0.04)};
\path[fill=color1,draw=color1,mark size=\marRad, mark=square*] plot coordinates {(0.01, 0.04)};
\path[fill=color1,draw=color1,mark size=\marRad, mark=square*] plot coordinates {(0.01, 0.04)};
\path[fill=color1,draw=color1,mark size=\marRad, mark=square*] plot coordinates {(0.01, 0.04)};
\path[fill=color1,draw=color1,mark size=\marRad, mark=square*] plot coordinates {(0.01, 0.04)};
\path[fill=color1,draw=color1,mark size=\marRad, mark=square*] plot coordinates {(-0.05, -0.43)};
\path[fill=color1,draw=color1,mark size=\marRad, mark=square*] plot coordinates {(-0.06, 0.32)};
\path[fill=color1,draw=color1,mark size=\marRad, mark=square*] plot coordinates {(0.01, 0.04)};
\path[fill=color1,draw=color1,mark size=\marRad, mark=square*] plot coordinates {(0.01, 0.04)};
\path[fill=color1,draw=color1,mark size=\marRad, mark=square*] plot coordinates {(0.30, -0.02)};
\path[fill=color1,draw=color1,mark size=\marRad, mark=square*] plot coordinates {(-0.09, 0.33)};
\path[fill=color1,draw=color1,mark size=\marRad, mark=square*] plot coordinates {(0.01, 0.04)};
\path[fill=color1,draw=color1,mark size=\marRad, mark=square*] plot coordinates {(0.01, 0.04)};
\path[fill=color1,draw=color1,mark size=\marRad, mark=square*] plot coordinates {(-0.16, 0.29)};
\path[fill=color1,draw=color1,mark size=\marRad, mark=square*] plot coordinates {(0.47, -0.08)};
\path[fill=color1,draw=color1,mark size=\marRad, mark=square*] plot coordinates {(0.57, -0.01)};
\path[fill=color1,draw=color1,mark size=\marRad, mark=square*] plot coordinates {(0.01, 0.04)};
\path[fill=color1,draw=color1,mark size=\marRad, mark=square*] plot coordinates {(0.09, 0.49)};
\path[fill=color1,draw=color1,mark size=\marRad, mark=square*] plot coordinates {(0.01, 0.04)};
\path[fill=color1,draw=color1,mark size=\marRad, mark=square*] plot coordinates {(0.01, 0.04)};
\path[fill=color1,draw=color1,mark size=\marRad, mark=square*] plot coordinates {(0.01, 0.04)};
\path[fill=color1,draw=color1,mark size=\marRad, mark=square*] plot coordinates {(0.15, 0.47)};
\path[fill=color1,draw=color1,mark size=\marRad, mark=square*] plot coordinates {(0.01, 0.04)};
\path[fill=color1,draw=color1,mark size=\marRad, mark=square*] plot coordinates {(0.08, -0.24)};
\path[fill=color1,draw=color1,mark size=\marRad, mark=square*] plot coordinates {(0.01, 0.04)};
\path[fill=color1,draw=color1,mark size=\marRad, mark=square*] plot coordinates {(0.01, 0.04)};
\path[fill=color1,draw=color1,mark size=\marRad, mark=square*] plot coordinates {(0.01, 0.04)};
\path[fill=color1,draw=color1,mark size=\marRad, mark=square*] plot coordinates {(0.01, 0.04)};
\path[fill=color1,draw=color1,mark size=\marRad, mark=square*] plot coordinates {(0.01, 0.04)};
\path[fill=color1,draw=color1,mark size=\marRad, mark=square*] plot coordinates {(-0.18, 0.25)};
\path[fill=color1,draw=color1,mark size=\marRad, mark=square*] plot coordinates {(-0.50, 0.43)};
\path[fill=color1,draw=color1,mark size=\marRad, mark=square*] plot coordinates {(0.01, 0.04)};
\path[fill=color1,draw=color1,mark size=\marRad, mark=square*] plot coordinates {(0.01, 0.04)};
\path[fill=color1,draw=color1,mark size=\marRad, mark=square*] plot coordinates {(0.01, 0.04)};
\path[fill=color1,draw=color1,mark size=\marRad, mark=square*] plot coordinates {(0.01, 0.04)};
\path[fill=color1,draw=color1,mark size=\marRad, mark=square*] plot coordinates {(-0.20, 0.25)};
\path[fill=color1,draw=color1,mark size=\marRad, mark=square*] plot coordinates {(0.01, 0.04)};
\path[fill=color1,draw=color1,mark size=\marRad, mark=square*] plot coordinates {(-0.05, 0.33)};
\path[fill=color1,draw=color1,mark size=\marRad, mark=square*] plot coordinates {(0.01, 0.04)};
\path[fill=color1,draw=color1,mark size=\marRad, mark=square*] plot coordinates {(0.01, 0.04)};
\path[fill=color1,draw=color1,mark size=\marRad, mark=square*] plot coordinates {(0.01, 0.04)};
\path[fill=color1,draw=color1,mark size=\marRad, mark=square*] plot coordinates {(0.01, 0.04)};
\path[fill=color1,draw=color1,mark size=\marRad, mark=square*] plot coordinates {(0.01, 0.04)};
\path[fill=color1,draw=color1,mark size=\marRad, mark=square*] plot coordinates {(0.36, 0.35)};
\path[fill=color1,draw=color1,mark size=\marRad, mark=square*] plot coordinates {(0.01, 0.04)};
\path[fill=color1,draw=color1,mark size=\marRad, mark=square*] plot coordinates {(0.01, 0.04)};
\path[fill=color1,draw=color1,mark size=\marRad, mark=square*] plot coordinates {(0.13, -0.23)};
\path[fill=color1,draw=color1,mark size=\marRad, mark=square*] plot coordinates {(-0.24, -0.11)};
\path[fill=color1,draw=color1,mark size=\marRad, mark=square*] plot coordinates {(-0.08, -0.41)};
\path[fill=color1,draw=color1,mark size=\marRad, mark=square*] plot coordinates {(0.01, 0.04)};
\path[fill=color1,draw=color1,mark size=\marRad, mark=square*] plot coordinates {(0.01, 0.04)};
\path[fill=color1,draw=color1,mark size=\marRad, mark=square*] plot coordinates {(0.01, 0.04)};
\path[fill=color1,draw=color1,mark size=\marRad, mark=square*] plot coordinates {(0.32, 0.40)};
\path[fill=color1,draw=color1,mark size=\marRad, mark=square*] plot coordinates {(0.01, 0.04)};
\path[fill=color1,draw=color1,mark size=\marRad, mark=square*] plot coordinates {(0.01, 0.04)};
\path[fill=color1,draw=color1,mark size=\marRad, mark=square*] plot coordinates {(0.01, 0.04)};
\path[fill=color1,draw=color1,mark size=\marRad, mark=square*] plot coordinates {(0.01, 0.04)};
\path[fill=color1,draw=color1,mark size=\marRad, mark=square*] plot coordinates {(-0.21, 0.25)};
\path[fill=color1,draw=color1,mark size=\marRad, mark=square*] plot coordinates {(0.22, -0.16)};
\path[fill=color1,draw=color1,mark size=\marRad, mark=square*] plot coordinates {(0.01, 0.04)};
\path[fill=color1,draw=color1,mark size=\marRad, mark=square*] plot coordinates {(0.01, 0.04)};
\path[fill=color1,draw=color1,mark size=\marRad, mark=square*] plot coordinates {(0.01, 0.04)};
\path[fill=color1,draw=color1,mark size=\marRad, mark=square*] plot coordinates {(0.01, 0.04)};
\path[fill=color1,draw=color1,mark size=\marRad, mark=square*] plot coordinates {(0.01, 0.04)};
\path[fill=color1,draw=color1,mark size=\marRad, mark=square*] plot coordinates {(0.01, 0.04)};
\path[fill=color1,draw=color1,mark size=\marRad, mark=square*] plot coordinates {(0.01, 0.04)};
\path[fill=color1,draw=color1,mark size=\marRad, mark=square*] plot coordinates {(0.01, 0.04)};
\path[fill=color1,draw=color1,mark size=\marRad, mark=square*] plot coordinates {(0.01, 0.04)};
\path[fill=color1,draw=color1,mark size=\marRad, mark=square*] plot coordinates {(0.01, 0.04)};
\path[fill=color1,draw=color1,mark size=\marRad, mark=square*] plot coordinates {(0.01, 0.04)};
\path[fill=color1,draw=color1,mark size=\marRad, mark=square*] plot coordinates {(0.01, 0.04)};
\path[fill=color1,draw=color1,mark size=\marRad, mark=square*] plot coordinates {(0.02, 0.34)};
\path[fill=color1,draw=color1,mark size=\marRad, mark=square*] plot coordinates {(0.01, 0.04)};
\path[fill=color1,draw=color1,mark size=\marRad, mark=square*] plot coordinates {(0.05, 0.34)};
\path[fill=color1,draw=color1,mark size=\marRad, mark=square*] plot coordinates {(0.01, 0.04)};
\path[fill=color1,draw=color1,mark size=\marRad, mark=square*] plot coordinates {(0.01, 0.04)};
\path[fill=color1,draw=color1,mark size=\marRad, mark=square*] plot coordinates {(-0.26, -0.07)};
\path[fill=color1,draw=color1,mark size=\marRad, mark=square*] plot coordinates {(0.15, 0.31)};
\path[fill=color1,draw=color1,mark size=\marRad, mark=square*] plot coordinates {(-0.02, -0.25)};
\path[fill=color1,draw=color1,mark size=\marRad, mark=square*] plot coordinates {(0.01, 0.04)};
\path[fill=color1,draw=color1,mark size=\marRad, mark=square*] plot coordinates {(0.01, 0.04)};
\path[fill=color1,draw=color1,mark size=\marRad, mark=square*] plot coordinates {(0.01, 0.04)};
\path[fill=color1,draw=color1,mark size=\marRad, mark=square*] plot coordinates {(-0.37, -0.25)};
\path[fill=color1,draw=color1,mark size=\marRad, mark=square*] plot coordinates {(0.01, 0.04)};
\path[fill=color1,draw=color1,mark size=\marRad, mark=square*] plot coordinates {(-0.43, -0.14)};
\path[fill=color1,draw=color1,mark size=\marRad, mark=square*] plot coordinates {(0.01, 0.04)};
\path[fill=color1,draw=color1,mark size=\marRad, mark=square*] plot coordinates {(0.01, 0.04)};
\path[fill=color1,draw=color1,mark size=\marRad, mark=square*] plot coordinates {(0.01, 0.04)};
\path[fill=color1,draw=color1,mark size=\marRad, mark=square*] plot coordinates {(0.01, 0.04)};
\path[fill=color1,draw=color1,mark size=\marRad, mark=square*] plot coordinates {(0.35, -0.25)};
\path[fill=color1,draw=color1,mark size=\marRad, mark=square*] plot coordinates {(0.01, 0.04)};
\path[fill=color1,draw=color1,mark size=\marRad, mark=square*] plot coordinates {(0.01, 0.04)};
\path[fill=color1,draw=color1,mark size=\marRad, mark=square*] plot coordinates {(-0.27, 0.15)};
\path[fill=color1,draw=color1,mark size=\marRad, mark=square*] plot coordinates {(0.01, 0.04)};
\path[fill=color1,draw=color1,mark size=\marRad, mark=square*] plot coordinates {(0.01, 0.04)};
\path[fill=color1,draw=color1,mark size=\marRad, mark=square*] plot coordinates {(-0.12, 0.31)};
\path[fill=color1,draw=color1,mark size=\marRad, mark=square*] plot coordinates {(0.01, 0.04)};
\path[fill=color1,draw=color1,mark size=\marRad, mark=square*] plot coordinates {(0.01, 0.04)};
\path[fill=color1,draw=color1,mark size=\marRad, mark=square*] plot coordinates {(0.01, 0.04)};
\path[fill=color1,draw=color1,mark size=\marRad, mark=square*] plot coordinates {(0.31, 0.05)};
\path[fill=color1,draw=color1,mark size=\marRad, mark=square*] plot coordinates {(0.01, 0.04)};
\path[fill=color1,draw=color1,mark size=\marRad, mark=square*] plot coordinates {(0.01, 0.04)};
\path[fill=color1,draw=color1,mark size=\marRad, mark=square*] plot coordinates {(0.01, 0.04)};
\path[fill=color1,draw=color1,mark size=\marRad, mark=square*] plot coordinates {(0.01, 0.04)};
\path[fill=color1,draw=color1,mark size=\marRad, mark=square*] plot coordinates {(-0.26, 0.12)};
\path[fill=color1,draw=color1,mark size=\marRad, mark=square*] plot coordinates {(-0.03, -0.25)};
\path[fill=color1,draw=color1,mark size=\marRad, mark=square*] plot coordinates {(0.01, 0.04)};
\path[fill=color1,draw=color1,mark size=\marRad, mark=square*] plot coordinates {(0.23, 0.24)};
\path[fill=color1,draw=color1,mark size=\marRad, mark=square*] plot coordinates {(0.04, -0.24)};
\path[fill=color1,draw=color1,mark size=\marRad, mark=square*] plot coordinates {(-0.28, 0.12)};
\path[fill=color1,draw=color1,mark size=\marRad, mark=square*] plot coordinates {(0.23, 0.24)};
\path[fill=color1,draw=color1,mark size=\marRad, mark=square*] plot coordinates {(0.01, 0.04)};
\path[fill=color1,draw=color1,mark size=\marRad, mark=square*] plot coordinates {(0.01, 0.04)};
\path[fill=color1,draw=color1,mark size=\marRad, mark=square*] plot coordinates {(0.01, 0.04)};
\path[fill=color1,draw=color1,mark size=\marRad, mark=square*] plot coordinates {(0.01, 0.04)};
\path[fill=color1,draw=color1,mark size=\marRad, mark=square*] plot coordinates {(0.01, 0.04)};
\path[fill=color1,draw=color1,mark size=\marRad, mark=square*] plot coordinates {(0.01, 0.04)};
\path[fill=color1,draw=color1,mark size=\marRad, mark=square*] plot coordinates {(0.01, 0.04)};
\path[fill=color1,draw=color1,mark size=\marRad, mark=square*] plot coordinates {(-0.06, -0.24)};
\path[fill=color1,draw=color1,mark size=\marRad, mark=square*] plot coordinates {(0.01, 0.04)};
\path[fill=color1,draw=color1,mark size=\marRad, mark=square*] plot coordinates {(0.01, 0.04)};
\path[fill=color1,draw=color1,mark size=\marRad, mark=square*] plot coordinates {(0.01, 0.04)};
\path[fill=color1,draw=color1,mark size=\marRad, mark=square*] plot coordinates {(0.01, 0.04)};
\path[fill=color1,draw=color1,mark size=\marRad, mark=square*] plot coordinates {(0.01, 0.04)};
\path[fill=color1,draw=color1,mark size=\marRad, mark=square*] plot coordinates {(0.01, 0.04)};
\path[fill=color1,draw=color1,mark size=\marRad, mark=square*] plot coordinates {(0.01, 0.04)};
\path[fill=color1,draw=color1,mark size=\marRad, mark=square*] plot coordinates {(0.01, 0.04)};
\path[fill=color1,draw=color1,mark size=\marRad, mark=square*] plot coordinates {(0.01, 0.04)};
\path[fill=color1,draw=color1,mark size=\marRad, mark=square*] plot coordinates {(0.01, 0.04)};
\path[fill=color1,draw=color1,mark size=\marRad, mark=square*] plot coordinates {(0.01, 0.04)};
\path[fill=color1,draw=color1,mark size=\marRad, mark=square*] plot coordinates {(0.01, 0.04)};
\path[fill=color1,draw=color1,mark size=\marRad, mark=square*] plot coordinates {(0.01, 0.04)};
\path[fill=color1,draw=color1,mark size=\marRad, mark=square*] plot coordinates {(0.01, 0.04)};
\path[fill=color1,draw=color1,mark size=\marRad, mark=square*] plot coordinates {(0.01, 0.04)};
\path[fill=color1,draw=color1,mark size=\marRad, mark=square*] plot coordinates {(0.01, 0.04)};
\path[fill=color1,draw=color1,mark size=\marRad, mark=square*] plot coordinates {(0.01, 0.04)};
\path[fill=color1,draw=color1,mark size=\marRad, mark=square*] plot coordinates {(0.01, 0.04)};
\path[fill=color1,draw=color1,mark size=\marRad, mark=square*] plot coordinates {(0.01, 0.04)};
\path[fill=color1,draw=color1,mark size=\marRad, mark=square*] plot coordinates {(0.01, 0.04)};
\path[fill=color1,draw=color1,mark size=\marRad, mark=square*] plot coordinates {(0.01, 0.04)};
\path[fill=color1,draw=color1,mark size=\marRad, mark=square*] plot coordinates {(0.01, 0.04)};
\path[fill=color1,draw=color1,mark size=\marRad, mark=square*] plot coordinates {(0.01, 0.04)};
\path[fill=color1,draw=color1,mark size=\marRad, mark=square*] plot coordinates {(0.01, 0.04)};
\path[fill=color1,draw=color1,mark size=\marRad, mark=square*] plot coordinates {(0.01, 0.04)};
\path[fill=color1,draw=color1,mark size=\marRad, mark=square*] plot coordinates {(0.01, 0.04)};
\path[fill=color1,draw=color1,mark size=\marRad, mark=square*] plot coordinates {(0.01, 0.04)};
\path[fill=color1,draw=color1,mark size=\marRad, mark=square*] plot coordinates {(0.01, 0.04)};
\path[fill=color1,draw=color1,mark size=\marRad, mark=square*] plot coordinates {(0.01, 0.04)};
\path[fill=color1,draw=color1,mark size=\marRad, mark=square*] plot coordinates {(0.01, 0.04)};
\path[fill=color1,draw=color1,mark size=\marRad, mark=square*] plot coordinates {(0.01, 0.04)};
\path[fill=color1,draw=color1,mark size=\marRad, mark=square*] plot coordinates {(0.01, 0.04)};
\path[fill=color1,draw=color1,mark size=\marRad, mark=square*] plot coordinates {(0.01, 0.04)};
\path[fill=color1,draw=color1,mark size=\marRad, mark=square*] plot coordinates {(0.01, 0.04)};
\path[fill=color1,draw=color1,mark size=\marRad, mark=square*] plot coordinates {(-0.15, 0.30)};
\path[fill=color1,draw=color1,mark size=\marRad, mark=square*] plot coordinates {(0.01, 0.04)};
\path[fill=color1,draw=color1,mark size=\marRad, mark=square*] plot coordinates {(0.01, 0.04)};
\path[fill=color1,draw=color1,mark size=\marRad, mark=square*] plot coordinates {(0.01, 0.04)};
\path[fill=color1,draw=color1,mark size=\marRad, mark=square*] plot coordinates {(0.01, 0.04)};
\path[fill=color1,draw=color1,mark size=\marRad, mark=square*] plot coordinates {(0.01, 0.04)};
\path[fill=color1,draw=color1,mark size=\marRad, mark=square*] plot coordinates {(0.01, 0.04)};
\path[fill=color1,draw=color1,mark size=\marRad, mark=square*] plot coordinates {(0.01, 0.04)};
\path[fill=color1,draw=color1,mark size=\marRad, mark=square*] plot coordinates {(0.01, 0.04)};
\path[fill=color1,draw=color1,mark size=\marRad, mark=square*] plot coordinates {(0.01, 0.04)};
\path[fill=color1,draw=color1,mark size=\marRad, mark=square*] plot coordinates {(0.01, 0.04)};
\path[fill=color1,draw=color1,mark size=\marRad, mark=square*] plot coordinates {(0.01, 0.04)};
\path[fill=color1,draw=color1,mark size=\marRad, mark=square*] plot coordinates {(0.01, 0.04)};
\path[fill=color1,draw=color1,mark size=\marRad, mark=square*] plot coordinates {(0.31, 0.02)};
\path[fill=color1,draw=color1,mark size=\marRad, mark=square*] plot coordinates {(0.01, 0.04)};
\path[fill=color1,draw=color1,mark size=\marRad, mark=square*] plot coordinates {(0.01, 0.04)};
\path[fill=color1,draw=color1,mark size=\marRad, mark=square*] plot coordinates {(0.01, 0.04)};
\path[fill=color1,draw=color1,mark size=\marRad, mark=square*] plot coordinates {(0.01, 0.04)};
\path[fill=color1,draw=color1,mark size=\marRad, mark=square*] plot coordinates {(0.01, 0.04)};
\path[fill=color1,draw=color1,mark size=\marRad, mark=square*] plot coordinates {(0.01, 0.04)};
\path[fill=color1,draw=color1,mark size=\marRad, mark=square*] plot coordinates {(0.01, 0.04)};
\path[fill=color1,draw=color1,mark size=\marRad, mark=square*] plot coordinates {(0.01, 0.04)};
\path[fill=color1,draw=color1,mark size=\marRad, mark=square*] plot coordinates {(0.28, 0.13)};
\path[fill=color1,draw=color1,mark size=\marRad, mark=square*] plot coordinates {(0.01, 0.04)};
\path[fill=color1,draw=color1,mark size=\marRad, mark=square*] plot coordinates {(0.01, 0.04)};
\path[fill=color1,draw=color1,mark size=\marRad, mark=square*] plot coordinates {(0.01, 0.04)};
\path[fill=color1,draw=color1,mark size=\marRad, mark=square*] plot coordinates {(0.01, 0.04)};
\path[fill=color1,draw=color1,mark size=\marRad, mark=square*] plot coordinates {(0.01, 0.04)};
\path[fill=color1,draw=color1,mark size=\marRad, mark=square*] plot coordinates {(0.01, 0.04)};
\path[fill=color1,draw=color1,mark size=\marRad, mark=square*] plot coordinates {(0.01, 0.04)};
\path[fill=color1,draw=color1,mark size=\marRad, mark=square*] plot coordinates {(0.45, -0.15)};
\path[fill=color1,draw=color1,mark size=\marRad, mark=square*] plot coordinates {(0.01, 0.04)};
\path[fill=color1,draw=color1,mark size=\marRad, mark=square*] plot coordinates {(0.01, 0.04)};
\path[fill=color1,draw=color1,mark size=\marRad, mark=square*] plot coordinates {(0.01, 0.04)};
\path[fill=color1,draw=color1,mark size=\marRad, mark=square*] plot coordinates {(0.01, 0.04)};
\path[fill=color1,draw=color1,mark size=\marRad, mark=square*] plot coordinates {(-0.26, 0.12)};
\path[fill=color1,draw=color1,mark size=\marRad, mark=square*] plot coordinates {(0.01, 0.04)};
\path[fill=color1,draw=color1,mark size=\marRad, mark=square*] plot coordinates {(0.01, 0.04)};
\path[fill=color1,draw=color1,mark size=\marRad, mark=square*] plot coordinates {(0.39, -0.24)};
\path[fill=color1,draw=color1,mark size=\marRad, mark=square*] plot coordinates {(0.01, 0.04)};
\path[fill=color1,draw=color1,mark size=\marRad, mark=square*] plot coordinates {(0.01, 0.04)};
\path[fill=color1,draw=color1,mark size=\marRad, mark=square*] plot coordinates {(0.01, 0.04)};
\path[fill=color1,draw=color1,mark size=\marRad, mark=square*] plot coordinates {(0.01, 0.04)};
\path[fill=color1,draw=color1,mark size=\marRad, mark=square*] plot coordinates {(0.09, 0.33)};
\path[fill=color1,draw=color1,mark size=\marRad, mark=square*] plot coordinates {(0.01, 0.04)};
\path[fill=color1,draw=color1,mark size=\marRad, mark=square*] plot coordinates {(0.01, 0.04)};
\path[fill=color1,draw=color1,mark size=\marRad, mark=square*] plot coordinates {(-0.03, 0.34)};
\path[fill=color1,draw=color1,mark size=\marRad, mark=square*] plot coordinates {(-0.12, -0.41)};
\path[fill=color7,draw=color7,mark size=\marRad, mark=square*] plot coordinates {(-0.16, -0.21)};
\path[fill=color7,draw=color7,mark size=\marRad, mark=square*] plot coordinates {(-0.18, -0.20)};
\path[fill=color7,draw=color7,mark size=\marRad, mark=square*] plot coordinates {(-0.17, -0.20)};
\path[fill=color7,draw=color7,mark size=\marRad, mark=square*] plot coordinates {(-0.17, -0.20)};
\path[fill=color7,draw=color7,mark size=\marRad, mark=square*] plot coordinates {(-0.40, -0.39)};
\path[fill=color7,draw=color7,mark size=\marRad, mark=square*] plot coordinates {(-0.40, -0.39)};
\path[fill=color7,draw=color7,mark size=\marRad, mark=square*] plot coordinates {(-0.25, -0.36)};
\path[fill=color7,draw=color7,mark size=\marRad, mark=square*] plot coordinates {(-0.17, -0.20)};
\path[fill=color7,draw=color7,mark size=\marRad, mark=square*] plot coordinates {(-0.17, -0.20)};
\path[fill=color7,draw=color7,mark size=\marRad, mark=square*] plot coordinates {(-0.26, -0.50)};
\path[fill=color7,draw=color7,mark size=\marRad, mark=square*] plot coordinates {(-0.27, -0.35)};
\path[fill=color7,draw=color7,mark size=\marRad, mark=square*] plot coordinates {(-0.28, -0.34)};
\path[fill=color7,draw=color7,mark size=\marRad, mark=square*] plot coordinates {(-0.23, -0.37)};
\path[fill=color7,draw=color7,mark size=\marRad, mark=square*] plot coordinates {(-0.17, -0.20)};
\path[fill=color7,draw=color7,mark size=\marRad, mark=square*] plot coordinates {(-0.32, -0.45)};
\path[fill=color7,draw=color7,mark size=\marRad, mark=square*] plot coordinates {(-0.26, -0.36)};
\path[fill=color7,draw=color7,mark size=\marRad, mark=square*] plot coordinates {(-0.17, -0.20)};
\path[fill=color7,draw=color7,mark size=\marRad, mark=square*] plot coordinates {(-0.18, -0.20)};
\path[fill=color7,draw=color7,mark size=\marRad, mark=square*] plot coordinates {(-0.17, -0.20)};
\path[fill=color7,draw=color7,mark size=\marRad, mark=square*] plot coordinates {(-0.17, -0.20)};
\path[fill=color7,draw=color7,mark size=\marRad, mark=square*] plot coordinates {(-0.18, -0.20)};
\path[fill=color7,draw=color7,mark size=\marRad, mark=square*] plot coordinates {(-0.31, -0.31)};
\path[fill=color7,draw=color7,mark size=\marRad, mark=square*] plot coordinates {(-0.17, -0.20)};
\path[fill=color7,draw=color7,mark size=\marRad, mark=square*] plot coordinates {(-0.17, -0.21)};
\path[fill=color7,draw=color7,mark size=\marRad, mark=square*] plot coordinates {(-0.18, -0.20)};
\path[fill=color7,draw=color7,mark size=\marRad, mark=square*] plot coordinates {(-0.17, -0.20)};
\path[fill=color7,draw=color7,mark size=\marRad, mark=square*] plot coordinates {(-0.47, -0.41)};
\path[fill=color7,draw=color7,mark size=\marRad, mark=square*] plot coordinates {(-0.36, -0.53)};
\path[fill=color7,draw=color7,mark size=\marRad, mark=square*] plot coordinates {(-0.33, -0.30)};
\path[fill=color7,draw=color7,mark size=\marRad, mark=square*] plot coordinates {(-0.12, -0.20)};
\path[fill=color7,draw=color7,mark size=\marRad, mark=square*] plot coordinates {(-0.41, -0.50)};
\path[fill=color7,draw=color7,mark size=\marRad, mark=square*] plot coordinates {(-0.27, -0.34)};
\path[fill=color7,draw=color7,mark size=\marRad, mark=square*] plot coordinates {(-0.17, -0.20)};
\path[fill=color7,draw=color7,mark size=\marRad, mark=square*] plot coordinates {(-0.18, -0.20)};
\path[fill=color7,draw=color7,mark size=\marRad, mark=square*] plot coordinates {(-0.48, -0.44)};
\path[fill=color7,draw=color7,mark size=\marRad, mark=square*] plot coordinates {(-0.30, -0.32)};
\path[fill=color7,draw=color7,mark size=\marRad, mark=square*] plot coordinates {(-0.24, -0.36)};
\path[fill=color7,draw=color7,mark size=\marRad, mark=square*] plot coordinates {(-0.36, -0.42)};
\path[fill=color7,draw=color7,mark size=\marRad, mark=square*] plot coordinates {(-0.35, -0.43)};
\path[fill=color7,draw=color7,mark size=\marRad, mark=square*] plot coordinates {(-0.35, -0.43)};
\path[fill=color7,draw=color7,mark size=\marRad, mark=square*] plot coordinates {(-0.32, -0.31)};
\path[fill=color7,draw=color7,mark size=\marRad, mark=square*] plot coordinates {(-0.29, -0.33)};
\path[fill=color7,draw=color7,mark size=\marRad, mark=square*] plot coordinates {(-0.22, -0.40)};
\path[fill=color7,draw=color7,mark size=\marRad, mark=square*] plot coordinates {(-0.35, -0.28)};
\path[fill=color7,draw=color7,mark size=\marRad, mark=square*] plot coordinates {(-0.33, -0.30)};
\path[fill=color7,draw=color7,mark size=\marRad, mark=square*] plot coordinates {(-0.26, -0.35)};
\path[fill=color7,draw=color7,mark size=\marRad, mark=square*] plot coordinates {(-0.30, -0.32)};
\path[fill=color5,draw=color5,mark size=\marRad, mark=heart] plot coordinates {(-0.29, -0.01)};
\path[fill=color5,draw=color5,mark size=\marRad, mark=heart] plot coordinates {(-0.29, 0.00)};
\path[fill=color5,draw=color5,mark size=\marRad, mark=heart] plot coordinates {(-0.29, -0.01)};
\path[fill=color5,draw=color5,mark size=\marRad, mark=heart] plot coordinates {(-0.29, -0.00)};
\path[fill=color5,draw=color5,mark size=\marRad, mark=heart] plot coordinates {(-0.29, 0.00)};
\path[fill=color5,draw=color5,mark size=\marRad, mark=heart] plot coordinates {(-0.29, -0.01)};
\path[fill=color5,draw=color5,mark size=\marRad, mark=heart] plot coordinates {(-0.29, -0.00)};
\path[fill=color5,draw=color5,mark size=\marRad, mark=heart] plot coordinates {(-0.55, -0.14)};
\path[fill=color5,draw=color5,mark size=\marRad, mark=heart] plot coordinates {(-0.29, -0.01)};
\path[fill=color5,draw=color5,mark size=\marRad, mark=heart] plot coordinates {(-0.29, 0.00)};
\path[fill=color5,draw=color5,mark size=\marRad, mark=heart] plot coordinates {(-0.29, -0.01)};
\path[fill=color5,draw=color5,mark size=\marRad, mark=heart] plot coordinates {(-0.46, -0.05)};
\path[fill=color5,draw=color5,mark size=\marRad, mark=heart] plot coordinates {(-0.57, -0.05)};
\path[fill=color5,draw=color5,mark size=\marRad, mark=heart] plot coordinates {(-0.56, -0.11)};
\path[fill=color5,draw=color5,mark size=\marRad, mark=heart] plot coordinates {(-0.56, -0.09)};
\path[fill=color5,draw=color5,mark size=\marRad, mark=heart] plot coordinates {(-0.29, -0.01)};
\path[fill=color5,draw=color5,mark size=\marRad, mark=heart] plot coordinates {(-0.46, -0.08)};
\path[fill=color5,draw=color5,mark size=\marRad, mark=heart] plot coordinates {(-0.70, -0.18)};
\path[fill=color5,draw=color5,mark size=\marRad, mark=heart] plot coordinates {(-0.29, -0.01)};
\path[fill=color5,draw=color5,mark size=\marRad, mark=heart] plot coordinates {(-0.29, -0.01)};
\path[fill=color5,draw=color5,mark size=\marRad, mark=heart] plot coordinates {(-0.67, -0.15)};
\path[fill=color5,draw=color5,mark size=\marRad, mark=heart] plot coordinates {(-0.29, -0.01)};
\path[fill=color2,draw=color2,mark size=\marRad, mark=triangle*] plot coordinates {(-0.57, 0.56)};
\path[fill=color2,draw=color2,mark size=\marRad, mark=triangle*] plot coordinates {(-0.52, 0.59)};
\path[fill=color3,draw=color3,mark size=\marRad, mark=diamond*] plot coordinates {(-0.73, 0.33)};
\path[fill=color3,draw=color3,mark size=\marRad, mark=diamond*] plot coordinates {(-0.76, 0.24)};
\path[fill=color3,draw=color3,mark size=\marRad, mark=diamond*] plot coordinates {(-0.72, 0.33)};
\path[fill=color4,draw=color4,mark size=\marRad, mark=pentagon*] plot coordinates {(-0.73, 0.05)};
\path[fill=color6,draw=color6,mark size=\marRad, mark=*] plot coordinates {(-0.66, -0.37)};
\path[fill=color8,draw=color8,mark size=\marRad, mark=triangle*] plot coordinates {(-0.21, -0.72)};
\path[fill=color8,draw=color8,mark size=\marRad, mark=triangle*] plot coordinates {(-0.21, -0.72)};
\path[fill=color9,draw=color9,mark size=\marRad, mark=diamond*] plot coordinates {(-0.02, -0.71)};

\draw (-0.04, 0.18) node[very thick,draw=black!50,top color=white,bottom color=black!20,anchor=west, draw=black] (A) {\#schumacher};
\draw[-] (A.south) -- (0.01, 0.04);
\draw (-0.55, 0.50) node[very thick,draw=black!50,top color=white,bottom color=black!20,anchor=west, draw=black] (B) {\#f1, \#schumacher};
\draw[-] (B.south) -- (-0.15, 0.30);
%{(-0.15, 0.30)}
\end{tikzpicture}

\end{frame}


\begin{frame}{Module - Clustering - Alternative Umsetzungen}
%Alternative Umsetzungen:
\begin{itemize}
    \item Statt Hashtags alle vorkommenden Wörter in den Tweets verwenden. 
\end{itemize}
 \def\maxW{3.61}
\begin{tikzpicture}[y=\textwidth/4.7,x=\textwidth/1.9, background rectangle/.style={draw=black, thick, fill=yellow!10,},show background rectangle]
\def\marRad{0.5mm}
\definecolor{color0}{rgb}{0.06,0.64,0.44}
\definecolor{color1}{rgb}{1.00,0.00,0.00}
\definecolor{color2}{rgb}{0.17,0.36,0.70}
\definecolor{color3}{rgb}{0.05,0.96,0.52}
\definecolor{color4}{rgb}{0.38,0.31,0.64}
\definecolor{color5}{rgb}{0.49,0.24,0.58}
\definecolor{color6}{rgb}{0.55,0.58,0.40}
\definecolor{color7}{rgb}{0.48,0.75,0.82}
\definecolor{color8}{rgb}{0.64,0.70,0.11}
\definecolor{color9}{rgb}{0.03,0.40,0.65}
\path[fill=color0,draw=color0,mark size=\marRad, mark=*] plot coordinates {(0.75, -0.24)};
\path[fill=color0,draw=color0,mark size=\marRad, mark=*] plot coordinates {(0.76, -0.22)};
\path[fill=color0,draw=color0,mark size=\marRad, mark=*] plot coordinates {(0.79, -0.25)};
\path[fill=color2,draw=color2,mark size=\marRad, mark=triangle*] plot coordinates {(0.19, 0.57)};
\path[fill=color2,draw=color2,mark size=\marRad, mark=triangle*] plot coordinates {(0.09, 0.26)};
\path[fill=color2,draw=color2,mark size=\marRad, mark=triangle*] plot coordinates {(0.50, 0.03)};
\path[fill=color2,draw=color2,mark size=\marRad, mark=triangle*] plot coordinates {(-0.21, 0.59)};
\path[fill=color2,draw=color2,mark size=\marRad, mark=triangle*] plot coordinates {(-0.02, 0.61)};
\path[fill=color2,draw=color2,mark size=\marRad, mark=triangle*] plot coordinates {(0.42, 0.45)};
\path[fill=color2,draw=color2,mark size=\marRad, mark=triangle*] plot coordinates {(0.03, 0.23)};
\path[fill=color2,draw=color2,mark size=\marRad, mark=triangle*] plot coordinates {(-0.11, 0.24)};
\path[fill=color2,draw=color2,mark size=\marRad, mark=triangle*] plot coordinates {(0.28, 0.09)};
\path[fill=color2,draw=color2,mark size=\marRad, mark=triangle*] plot coordinates {(-0.26, 0.35)};
\path[fill=color2,draw=color2,mark size=\marRad, mark=triangle*] plot coordinates {(-0.14, 0.31)};
\path[fill=color2,draw=color2,mark size=\marRad, mark=triangle*] plot coordinates {(0.42, 0.32)};
\path[fill=color2,draw=color2,mark size=\marRad, mark=triangle*] plot coordinates {(0.12, 0.35)};
\path[fill=color2,draw=color2,mark size=\marRad, mark=triangle*] plot coordinates {(-0.31, 0.54)};
\path[fill=color2,draw=color2,mark size=\marRad, mark=triangle*] plot coordinates {(0.44, 0.29)};
\path[fill=color2,draw=color2,mark size=\marRad, mark=triangle*] plot coordinates {(-0.18, 0.59)};
\path[fill=color2,draw=color2,mark size=\marRad, mark=triangle*] plot coordinates {(-0.34, 0.39)};
\path[fill=color2,draw=color2,mark size=\marRad, mark=triangle*] plot coordinates {(-0.21, 0.04)};
\path[fill=color2,draw=color2,mark size=\marRad, mark=triangle*] plot coordinates {(-0.40, 0.30)};
\path[fill=color2,draw=color2,mark size=\marRad, mark=triangle*] plot coordinates {(0.07, 0.33)};
\path[fill=color2,draw=color2,mark size=\marRad, mark=triangle*] plot coordinates {(-0.38, 0.22)};
\path[fill=color2,draw=color2,mark size=\marRad, mark=triangle*] plot coordinates {(0.14, 0.08)};
\path[fill=color2,draw=color2,mark size=\marRad, mark=triangle*] plot coordinates {(-0.02, 0.61)};
\path[fill=color2,draw=color2,mark size=\marRad, mark=triangle*] plot coordinates {(0.14, 0.72)};
\path[fill=color2,draw=color2,mark size=\marRad, mark=triangle*] plot coordinates {(0.51, -0.12)};
\path[fill=color2,draw=color2,mark size=\marRad, mark=triangle*] plot coordinates {(0.25, 0.48)};
\path[fill=color2,draw=color2,mark size=\marRad, mark=triangle*] plot coordinates {(0.51, 0.28)};
\path[fill=color2,draw=color2,mark size=\marRad, mark=triangle*] plot coordinates {(0.39, 0.14)};
\path[fill=color2,draw=color2,mark size=\marRad, mark=triangle*] plot coordinates {(0.54, 0.47)};
\path[fill=color2,draw=color2,mark size=\marRad, mark=triangle*] plot coordinates {(0.50, 0.29)};
\path[fill=color2,draw=color2,mark size=\marRad, mark=triangle*] plot coordinates {(0.27, 0.29)};
\path[fill=color2,draw=color2,mark size=\marRad, mark=triangle*] plot coordinates {(-0.04, 0.02)};
\path[fill=color2,draw=color2,mark size=\marRad, mark=triangle*] plot coordinates {(-0.52, 0.09)};
\path[fill=color2,draw=color2,mark size=\marRad, mark=triangle*] plot coordinates {(-0.11, 0.62)};
\path[fill=color2,draw=color2,mark size=\marRad, mark=triangle*] plot coordinates {(0.15, -0.16)};
\path[fill=color2,draw=color2,mark size=\marRad, mark=triangle*] plot coordinates {(0.67, -0.11)};
\path[fill=color2,draw=color2,mark size=\marRad, mark=triangle*] plot coordinates {(0.39, 0.23)};
\path[fill=color2,draw=color2,mark size=\marRad, mark=triangle*] plot coordinates {(-0.29, 0.53)};
\path[fill=color2,draw=color2,mark size=\marRad, mark=triangle*] plot coordinates {(0.46, 0.28)};
\path[fill=color2,draw=color2,mark size=\marRad, mark=triangle*] plot coordinates {(-0.28, 0.24)};
\path[fill=color2,draw=color2,mark size=\marRad, mark=triangle*] plot coordinates {(0.08, 0.49)};
\path[fill=color2,draw=color2,mark size=\marRad, mark=triangle*] plot coordinates {(-0.24, 0.53)};
\path[fill=color2,draw=color2,mark size=\marRad, mark=triangle*] plot coordinates {(-0.25, -0.34)};
\path[fill=color2,draw=color2,mark size=\marRad, mark=triangle*] plot coordinates {(0.21, -0.44)};
\path[fill=color2,draw=color2,mark size=\marRad, mark=triangle*] plot coordinates {(-0.33, 0.06)};
\path[fill=color2,draw=color2,mark size=\marRad, mark=triangle*] plot coordinates {(0.42, 0.42)};
\path[fill=color2,draw=color2,mark size=\marRad, mark=triangle*] plot coordinates {(-0.07, 0.61)};
\path[fill=color2,draw=color2,mark size=\marRad, mark=triangle*] plot coordinates {(0.10, 0.26)};
\path[fill=color2,draw=color2,mark size=\marRad, mark=triangle*] plot coordinates {(-0.11, 0.43)};
\path[fill=color2,draw=color2,mark size=\marRad, mark=triangle*] plot coordinates {(0.13, 0.01)};
\path[fill=color2,draw=color2,mark size=\marRad, mark=triangle*] plot coordinates {(-0.51, 0.56)};
\path[fill=color2,draw=color2,mark size=\marRad, mark=triangle*] plot coordinates {(0.33, 0.52)};
\path[fill=color2,draw=color2,mark size=\marRad, mark=triangle*] plot coordinates {(-0.28, 0.51)};
\path[fill=color2,draw=color2,mark size=\marRad, mark=triangle*] plot coordinates {(-0.06, 0.67)};
\path[fill=color2,draw=color2,mark size=\marRad, mark=triangle*] plot coordinates {(0.67, 0.16)};
\path[fill=color2,draw=color2,mark size=\marRad, mark=triangle*] plot coordinates {(-0.09, -0.00)};
\path[fill=color2,draw=color2,mark size=\marRad, mark=triangle*] plot coordinates {(-0.26, 0.29)};
\path[fill=color2,draw=color2,mark size=\marRad, mark=triangle*] plot coordinates {(0.65, 0.06)};
\path[fill=color2,draw=color2,mark size=\marRad, mark=triangle*] plot coordinates {(0.45, 0.02)};
\path[fill=color2,draw=color2,mark size=\marRad, mark=triangle*] plot coordinates {(0.27, 0.00)};
\path[fill=color2,draw=color2,mark size=\marRad, mark=triangle*] plot coordinates {(-0.19, 0.12)};
\path[fill=color2,draw=color2,mark size=\marRad, mark=triangle*] plot coordinates {(0.07, 0.64)};
\path[fill=color2,draw=color2,mark size=\marRad, mark=triangle*] plot coordinates {(-0.08, 0.53)};
\path[fill=color2,draw=color2,mark size=\marRad, mark=triangle*] plot coordinates {(-0.14, 0.40)};
\path[fill=color2,draw=color2,mark size=\marRad, mark=triangle*] plot coordinates {(0.62, 0.26)};
\path[fill=color2,draw=color2,mark size=\marRad, mark=triangle*] plot coordinates {(0.55, 0.36)};
\path[fill=color2,draw=color2,mark size=\marRad, mark=triangle*] plot coordinates {(0.11, 0.43)};
\path[fill=color2,draw=color2,mark size=\marRad, mark=triangle*] plot coordinates {(0.16, -0.03)};
\path[fill=color2,draw=color2,mark size=\marRad, mark=triangle*] plot coordinates {(0.41, 0.44)};
\path[fill=color2,draw=color2,mark size=\marRad, mark=triangle*] plot coordinates {(-0.07, 0.36)};
\path[fill=color2,draw=color2,mark size=\marRad, mark=triangle*] plot coordinates {(-0.25, 0.07)};
\path[fill=color2,draw=color2,mark size=\marRad, mark=triangle*] plot coordinates {(0.68, 0.14)};
\path[fill=color2,draw=color2,mark size=\marRad, mark=triangle*] plot coordinates {(0.11, 0.68)};
\path[fill=color2,draw=color2,mark size=\marRad, mark=triangle*] plot coordinates {(0.15, 0.32)};
\path[fill=color2,draw=color2,mark size=\marRad, mark=triangle*] plot coordinates {(-0.11, 0.03)};
\path[fill=color2,draw=color2,mark size=\marRad, mark=triangle*] plot coordinates {(-0.48, 0.27)};
\path[fill=color2,draw=color2,mark size=\marRad, mark=triangle*] plot coordinates {(0.51, 0.45)};
\path[fill=color2,draw=color2,mark size=\marRad, mark=triangle*] plot coordinates {(0.32, 0.37)};
\path[fill=color2,draw=color2,mark size=\marRad, mark=triangle*] plot coordinates {(-0.43, 0.52)};
\path[fill=color2,draw=color2,mark size=\marRad, mark=triangle*] plot coordinates {(0.48, 0.26)};
\path[fill=color2,draw=color2,mark size=\marRad, mark=triangle*] plot coordinates {(0.04, 0.31)};
\path[fill=color2,draw=color2,mark size=\marRad, mark=triangle*] plot coordinates {(0.58, 0.22)};
\path[fill=color2,draw=color2,mark size=\marRad, mark=triangle*] plot coordinates {(0.69, -0.18)};
\path[fill=color2,draw=color2,mark size=\marRad, mark=triangle*] plot coordinates {(-0.11, 0.42)};
\path[fill=color2,draw=color2,mark size=\marRad, mark=triangle*] plot coordinates {(0.23, 0.23)};
\path[fill=color2,draw=color2,mark size=\marRad, mark=triangle*] plot coordinates {(0.24, -0.23)};
\path[fill=color2,draw=color2,mark size=\marRad, mark=triangle*] plot coordinates {(0.28, 0.15)};
\path[fill=color2,draw=color2,mark size=\marRad, mark=triangle*] plot coordinates {(0.18, 0.76)};
\path[fill=color2,draw=color2,mark size=\marRad, mark=triangle*] plot coordinates {(0.00, 0.45)};
\path[fill=color2,draw=color2,mark size=\marRad, mark=triangle*] plot coordinates {(-0.10, 0.65)};
\path[fill=color2,draw=color2,mark size=\marRad, mark=triangle*] plot coordinates {(0.20, 0.51)};
\path[fill=color2,draw=color2,mark size=\marRad, mark=triangle*] plot coordinates {(0.46, 0.33)};
\path[fill=color2,draw=color2,mark size=\marRad, mark=triangle*] plot coordinates {(-0.30, 0.47)};
\path[fill=color2,draw=color2,mark size=\marRad, mark=triangle*] plot coordinates {(0.23, 0.21)};
\path[fill=color2,draw=color2,mark size=\marRad, mark=triangle*] plot coordinates {(0.68, 0.02)};
\path[fill=color2,draw=color2,mark size=\marRad, mark=triangle*] plot coordinates {(0.15, -0.03)};
\path[fill=color2,draw=color2,mark size=\marRad, mark=triangle*] plot coordinates {(-0.10, -0.07)};
\path[fill=color2,draw=color2,mark size=\marRad, mark=triangle*] plot coordinates {(0.40, 0.53)};
\path[fill=color2,draw=color2,mark size=\marRad, mark=triangle*] plot coordinates {(0.53, 0.10)};
\path[fill=color2,draw=color2,mark size=\marRad, mark=triangle*] plot coordinates {(-0.45, 0.24)};
\path[fill=color2,draw=color2,mark size=\marRad, mark=triangle*] plot coordinates {(0.31, 0.61)};
\path[fill=color2,draw=color2,mark size=\marRad, mark=triangle*] plot coordinates {(0.25, 0.23)};
\path[fill=color2,draw=color2,mark size=\marRad, mark=triangle*] plot coordinates {(-0.26, 0.71)};
\path[fill=color2,draw=color2,mark size=\marRad, mark=triangle*] plot coordinates {(0.62, -0.00)};
\path[fill=color2,draw=color2,mark size=\marRad, mark=triangle*] plot coordinates {(0.02, 0.18)};
\path[fill=color2,draw=color2,mark size=\marRad, mark=triangle*] plot coordinates {(0.07, -0.25)};
\path[fill=color2,draw=color2,mark size=\marRad, mark=triangle*] plot coordinates {(-0.31, 0.34)};
\path[fill=color2,draw=color2,mark size=\marRad, mark=triangle*] plot coordinates {(0.58, 0.07)};
\path[fill=color2,draw=color2,mark size=\marRad, mark=triangle*] plot coordinates {(0.02, -0.14)};
\path[fill=color2,draw=color2,mark size=\marRad, mark=triangle*] plot coordinates {(0.55, 0.15)};
\path[fill=color2,draw=color2,mark size=\marRad, mark=triangle*] plot coordinates {(0.13, -0.17)};
\path[fill=color2,draw=color2,mark size=\marRad, mark=triangle*] plot coordinates {(0.48, 0.31)};
\path[fill=color2,draw=color2,mark size=\marRad, mark=triangle*] plot coordinates {(0.48, -0.04)};
\path[fill=color2,draw=color2,mark size=\marRad, mark=triangle*] plot coordinates {(0.04, -0.04)};
\path[fill=color2,draw=color2,mark size=\marRad, mark=triangle*] plot coordinates {(-0.06, -0.09)};
\path[fill=color2,draw=color2,mark size=\marRad, mark=triangle*] plot coordinates {(-0.18, 0.21)};
\path[fill=color2,draw=color2,mark size=\marRad, mark=triangle*] plot coordinates {(-0.13, 0.73)};
\path[fill=color2,draw=color2,mark size=\marRad, mark=triangle*] plot coordinates {(0.03, 0.42)};
\path[fill=color2,draw=color2,mark size=\marRad, mark=triangle*] plot coordinates {(0.50, 0.41)};
\path[fill=color2,draw=color2,mark size=\marRad, mark=triangle*] plot coordinates {(0.28, -0.05)};
\path[fill=color2,draw=color2,mark size=\marRad, mark=triangle*] plot coordinates {(0.49, 0.43)};
\path[fill=color2,draw=color2,mark size=\marRad, mark=triangle*] plot coordinates {(0.59, 0.18)};
\path[fill=color2,draw=color2,mark size=\marRad, mark=triangle*] plot coordinates {(-0.23, 0.66)};
\path[fill=color2,draw=color2,mark size=\marRad, mark=triangle*] plot coordinates {(0.14, 0.23)};
\path[fill=color2,draw=color2,mark size=\marRad, mark=triangle*] plot coordinates {(0.21, 0.61)};
\path[fill=color2,draw=color2,mark size=\marRad, mark=triangle*] plot coordinates {(0.27, 0.51)};
\path[fill=color2,draw=color2,mark size=\marRad, mark=triangle*] plot coordinates {(-0.39, 0.49)};
\path[fill=color2,draw=color2,mark size=\marRad, mark=triangle*] plot coordinates {(0.34, 0.53)};
\path[fill=color2,draw=color2,mark size=\marRad, mark=triangle*] plot coordinates {(0.32, 0.25)};
\path[fill=color2,draw=color2,mark size=\marRad, mark=triangle*] plot coordinates {(-0.31, 0.26)};
\path[fill=color2,draw=color2,mark size=\marRad, mark=triangle*] plot coordinates {(-0.19, 0.48)};
\path[fill=color2,draw=color2,mark size=\marRad, mark=triangle*] plot coordinates {(0.19, 0.01)};
\path[fill=color2,draw=color2,mark size=\marRad, mark=triangle*] plot coordinates {(0.20, 0.30)};
\path[fill=color2,draw=color2,mark size=\marRad, mark=triangle*] plot coordinates {(0.18, 0.51)};
\path[fill=color2,draw=color2,mark size=\marRad, mark=triangle*] plot coordinates {(-0.21, 0.32)};
\path[fill=color2,draw=color2,mark size=\marRad, mark=triangle*] plot coordinates {(-0.39, 0.31)};
\path[fill=color2,draw=color2,mark size=\marRad, mark=triangle*] plot coordinates {(0.59, 0.21)};
\path[fill=color2,draw=color2,mark size=\marRad, mark=triangle*] plot coordinates {(0.48, 0.30)};
\path[fill=color2,draw=color2,mark size=\marRad, mark=triangle*] plot coordinates {(0.40, -0.10)};
\path[fill=color2,draw=color2,mark size=\marRad, mark=triangle*] plot coordinates {(-0.33, -0.02)};
\path[fill=color2,draw=color2,mark size=\marRad, mark=triangle*] plot coordinates {(-0.03, 0.31)};
\path[fill=color2,draw=color2,mark size=\marRad, mark=triangle*] plot coordinates {(-0.08, 0.21)};
\path[fill=color2,draw=color2,mark size=\marRad, mark=triangle*] plot coordinates {(0.65, 0.32)};
\path[fill=color2,draw=color2,mark size=\marRad, mark=triangle*] plot coordinates {(0.29, 0.22)};
\path[fill=color2,draw=color2,mark size=\marRad, mark=triangle*] plot coordinates {(0.13, 0.66)};
\path[fill=color2,draw=color2,mark size=\marRad, mark=triangle*] plot coordinates {(0.42, 0.54)};
\path[fill=color2,draw=color2,mark size=\marRad, mark=triangle*] plot coordinates {(-0.44, 0.40)};
\path[fill=color2,draw=color2,mark size=\marRad, mark=triangle*] plot coordinates {(-0.13, 0.66)};
\path[fill=color2,draw=color2,mark size=\marRad, mark=triangle*] plot coordinates {(0.47, 0.11)};
\path[fill=color2,draw=color2,mark size=\marRad, mark=triangle*] plot coordinates {(-0.20, 0.30)};
\path[fill=color2,draw=color2,mark size=\marRad, mark=triangle*] plot coordinates {(-0.11, 0.71)};
\path[fill=color2,draw=color2,mark size=\marRad, mark=triangle*] plot coordinates {(0.22, 0.07)};
\path[fill=color2,draw=color2,mark size=\marRad, mark=triangle*] plot coordinates {(-0.19, 0.65)};
\path[fill=color2,draw=color2,mark size=\marRad, mark=triangle*] plot coordinates {(0.33, 0.17)};
\path[fill=color2,draw=color2,mark size=\marRad, mark=triangle*] plot coordinates {(0.72, 0.22)};
\path[fill=color2,draw=color2,mark size=\marRad, mark=triangle*] plot coordinates {(-0.28, 0.59)};
\path[fill=color2,draw=color2,mark size=\marRad, mark=triangle*] plot coordinates {(-0.27, -0.05)};
\path[fill=color2,draw=color2,mark size=\marRad, mark=triangle*] plot coordinates {(0.13, 0.51)};
\path[fill=color2,draw=color2,mark size=\marRad, mark=triangle*] plot coordinates {(0.38, 0.63)};
\path[fill=color2,draw=color2,mark size=\marRad, mark=triangle*] plot coordinates {(0.34, 0.13)};
\path[fill=color2,draw=color2,mark size=\marRad, mark=triangle*] plot coordinates {(0.65, 0.35)};
\path[fill=color2,draw=color2,mark size=\marRad, mark=triangle*] plot coordinates {(0.09, 0.57)};
\path[fill=color2,draw=color2,mark size=\marRad, mark=triangle*] plot coordinates {(-0.16, 0.04)};
\path[fill=color2,draw=color2,mark size=\marRad, mark=triangle*] plot coordinates {(0.20, 0.45)};
\path[fill=color2,draw=color2,mark size=\marRad, mark=triangle*] plot coordinates {(-0.00, 0.64)};
\path[fill=color2,draw=color2,mark size=\marRad, mark=triangle*] plot coordinates {(-0.01, 0.33)};
\path[fill=color2,draw=color2,mark size=\marRad, mark=triangle*] plot coordinates {(-0.09, -0.05)};
\path[fill=color2,draw=color2,mark size=\marRad, mark=triangle*] plot coordinates {(0.36, 0.36)};
\path[fill=color2,draw=color2,mark size=\marRad, mark=triangle*] plot coordinates {(0.68, 0.09)};
\path[fill=color2,draw=color2,mark size=\marRad, mark=triangle*] plot coordinates {(0.41, 0.49)};
\path[fill=color2,draw=color2,mark size=\marRad, mark=triangle*] plot coordinates {(0.02, 0.14)};
\path[fill=color2,draw=color2,mark size=\marRad, mark=triangle*] plot coordinates {(0.47, 0.06)};
\path[fill=color2,draw=color2,mark size=\marRad, mark=triangle*] plot coordinates {(-0.31, 0.13)};
\path[fill=color2,draw=color2,mark size=\marRad, mark=triangle*] plot coordinates {(0.47, 0.52)};
\path[fill=color2,draw=color2,mark size=\marRad, mark=triangle*] plot coordinates {(0.56, 0.41)};
\path[fill=color2,draw=color2,mark size=\marRad, mark=triangle*] plot coordinates {(0.03, 0.55)};
\path[fill=color2,draw=color2,mark size=\marRad, mark=triangle*] plot coordinates {(-0.24, -0.21)};
\path[fill=color2,draw=color2,mark size=\marRad, mark=triangle*] plot coordinates {(0.75, -0.07)};
\path[fill=color2,draw=color2,mark size=\marRad, mark=triangle*] plot coordinates {(0.61, -0.03)};
\path[fill=color2,draw=color2,mark size=\marRad, mark=triangle*] plot coordinates {(0.27, 0.67)};
\path[fill=color2,draw=color2,mark size=\marRad, mark=triangle*] plot coordinates {(-0.13, 0.43)};
\path[fill=color2,draw=color2,mark size=\marRad, mark=triangle*] plot coordinates {(0.02, 0.01)};
\path[fill=color2,draw=color2,mark size=\marRad, mark=triangle*] plot coordinates {(-0.24, 0.33)};
\path[fill=color2,draw=color2,mark size=\marRad, mark=triangle*] plot coordinates {(0.25, 0.51)};
\path[fill=color2,draw=color2,mark size=\marRad, mark=triangle*] plot coordinates {(0.12, 0.04)};
\path[fill=color2,draw=color2,mark size=\marRad, mark=triangle*] plot coordinates {(0.28, 0.75)};
\path[fill=color2,draw=color2,mark size=\marRad, mark=triangle*] plot coordinates {(0.71, 0.03)};
\path[fill=color2,draw=color2,mark size=\marRad, mark=triangle*] plot coordinates {(0.06, 0.41)};
\path[fill=color2,draw=color2,mark size=\marRad, mark=triangle*] plot coordinates {(0.40, -0.09)};
\path[fill=color2,draw=color2,mark size=\marRad, mark=triangle*] plot coordinates {(0.60, 0.05)};
\path[fill=color2,draw=color2,mark size=\marRad, mark=triangle*] plot coordinates {(0.14, 0.65)};
\path[fill=color2,draw=color2,mark size=\marRad, mark=triangle*] plot coordinates {(-0.07, 0.68)};
\path[fill=color2,draw=color2,mark size=\marRad, mark=triangle*] plot coordinates {(0.69, 0.10)};
\path[fill=color2,draw=color2,mark size=\marRad, mark=triangle*] plot coordinates {(-0.32, 0.12)};
\path[fill=color2,draw=color2,mark size=\marRad, mark=triangle*] plot coordinates {(-0.17, 0.64)};
\path[fill=color2,draw=color2,mark size=\marRad, mark=triangle*] plot coordinates {(0.01, 0.62)};
\path[fill=color2,draw=color2,mark size=\marRad, mark=triangle*] plot coordinates {(0.22, 0.45)};
\path[fill=color2,draw=color2,mark size=\marRad, mark=triangle*] plot coordinates {(-0.42, 0.57)};
\path[fill=color2,draw=color2,mark size=\marRad, mark=triangle*] plot coordinates {(-0.33, 0.35)};
\path[fill=color2,draw=color2,mark size=\marRad, mark=triangle*] plot coordinates {(0.24, 0.09)};
\path[fill=color2,draw=color2,mark size=\marRad, mark=triangle*] plot coordinates {(0.65, -0.14)};
\path[fill=color2,draw=color2,mark size=\marRad, mark=triangle*] plot coordinates {(0.28, 0.42)};
\path[fill=color2,draw=color2,mark size=\marRad, mark=triangle*] plot coordinates {(-0.27, 0.34)};
\path[fill=color2,draw=color2,mark size=\marRad, mark=triangle*] plot coordinates {(-0.23, 0.49)};
\path[fill=color2,draw=color2,mark size=\marRad, mark=triangle*] plot coordinates {(-0.18, -0.11)};
\path[fill=color2,draw=color2,mark size=\marRad, mark=triangle*] plot coordinates {(0.22, 0.36)};
\path[fill=color2,draw=color2,mark size=\marRad, mark=triangle*] plot coordinates {(-0.29, 0.02)};
\path[fill=color2,draw=color2,mark size=\marRad, mark=triangle*] plot coordinates {(-0.29, 0.36)};
\path[fill=color2,draw=color2,mark size=\marRad, mark=triangle*] plot coordinates {(-0.24, 0.53)};
\path[fill=color2,draw=color2,mark size=\marRad, mark=triangle*] plot coordinates {(0.75, 0.13)};
\path[fill=color2,draw=color2,mark size=\marRad, mark=triangle*] plot coordinates {(0.52, 0.59)};
\path[fill=color2,draw=color2,mark size=\marRad, mark=triangle*] plot coordinates {(-0.23, 0.45)};
\path[fill=color2,draw=color2,mark size=\marRad, mark=triangle*] plot coordinates {(0.39, 0.06)};
\path[fill=color2,draw=color2,mark size=\marRad, mark=triangle*] plot coordinates {(0.19, 0.45)};
\path[fill=color2,draw=color2,mark size=\marRad, mark=triangle*] plot coordinates {(0.37, 0.55)};
\path[fill=color2,draw=color2,mark size=\marRad, mark=triangle*] plot coordinates {(-0.13, 0.28)};
\path[fill=color2,draw=color2,mark size=\marRad, mark=triangle*] plot coordinates {(-0.01, 0.69)};
\path[fill=color2,draw=color2,mark size=\marRad, mark=triangle*] plot coordinates {(0.30, 0.43)};
\path[fill=color2,draw=color2,mark size=\marRad, mark=triangle*] plot coordinates {(0.22, 0.64)};
\path[fill=color2,draw=color2,mark size=\marRad, mark=triangle*] plot coordinates {(0.20, -0.08)};
\path[fill=color2,draw=color2,mark size=\marRad, mark=triangle*] plot coordinates {(0.29, 0.60)};
\path[fill=color2,draw=color2,mark size=\marRad, mark=triangle*] plot coordinates {(-0.21, 0.56)};
\path[fill=color2,draw=color2,mark size=\marRad, mark=triangle*] plot coordinates {(-0.06, -0.21)};
\path[fill=color2,draw=color2,mark size=\marRad, mark=triangle*] plot coordinates {(0.04, -0.07)};
\path[fill=color2,draw=color2,mark size=\marRad, mark=triangle*] plot coordinates {(-0.08, 0.06)};
\path[fill=color2,draw=color2,mark size=\marRad, mark=triangle*] plot coordinates {(0.18, 0.18)};
\path[fill=color2,draw=color2,mark size=\marRad, mark=triangle*] plot coordinates {(-0.15, -0.24)};
\path[fill=color2,draw=color2,mark size=\marRad, mark=triangle*] plot coordinates {(0.54, 0.44)};
\path[fill=color2,draw=color2,mark size=\marRad, mark=triangle*] plot coordinates {(0.68, 0.26)};
\path[fill=color2,draw=color2,mark size=\marRad, mark=triangle*] plot coordinates {(0.57, 0.02)};
\path[fill=color2,draw=color2,mark size=\marRad, mark=triangle*] plot coordinates {(0.02, -0.04)};
\path[fill=color2,draw=color2,mark size=\marRad, mark=triangle*] plot coordinates {(-0.02, 0.56)};
\path[fill=color2,draw=color2,mark size=\marRad, mark=triangle*] plot coordinates {(-0.34, 0.61)};
\path[fill=color2,draw=color2,mark size=\marRad, mark=triangle*] plot coordinates {(0.18, 0.63)};
\path[fill=color2,draw=color2,mark size=\marRad, mark=triangle*] plot coordinates {(0.53, 0.24)};
\path[fill=color2,draw=color2,mark size=\marRad, mark=triangle*] plot coordinates {(0.07, 0.68)};
\path[fill=color2,draw=color2,mark size=\marRad, mark=triangle*] plot coordinates {(-0.37, 0.20)};
\path[fill=color2,draw=color2,mark size=\marRad, mark=triangle*] plot coordinates {(0.71, 0.03)};
\path[fill=color2,draw=color2,mark size=\marRad, mark=triangle*] plot coordinates {(0.35, 0.06)};
\path[fill=color2,draw=color2,mark size=\marRad, mark=triangle*] plot coordinates {(0.47, 0.52)};
\path[fill=color2,draw=color2,mark size=\marRad, mark=triangle*] plot coordinates {(0.38, 0.67)};
\path[fill=color2,draw=color2,mark size=\marRad, mark=triangle*] plot coordinates {(0.65, -0.10)};
\path[fill=color2,draw=color2,mark size=\marRad, mark=triangle*] plot coordinates {(0.57, 0.46)};
\path[fill=color2,draw=color2,mark size=\marRad, mark=triangle*] plot coordinates {(0.29, 0.07)};
\path[fill=color2,draw=color2,mark size=\marRad, mark=triangle*] plot coordinates {(-0.18, 0.55)};
\path[fill=color2,draw=color2,mark size=\marRad, mark=triangle*] plot coordinates {(0.13, 0.12)};
\path[fill=color2,draw=color2,mark size=\marRad, mark=triangle*] plot coordinates {(-0.10, 0.32)};
\path[fill=color2,draw=color2,mark size=\marRad, mark=triangle*] plot coordinates {(-0.42, 0.62)};
\path[fill=color2,draw=color2,mark size=\marRad, mark=triangle*] plot coordinates {(-0.15, -0.06)};
\path[fill=color2,draw=color2,mark size=\marRad, mark=triangle*] plot coordinates {(-0.19, -0.03)};
\path[fill=color2,draw=color2,mark size=\marRad, mark=triangle*] plot coordinates {(-0.34, 0.12)};
\path[fill=color2,draw=color2,mark size=\marRad, mark=triangle*] plot coordinates {(0.20, 0.09)};
\path[fill=color2,draw=color2,mark size=\marRad, mark=triangle*] plot coordinates {(0.41, 0.00)};
\path[fill=color2,draw=color2,mark size=\marRad, mark=triangle*] plot coordinates {(0.36, -0.01)};
\path[fill=color2,draw=color2,mark size=\marRad, mark=triangle*] plot coordinates {(-0.08, 0.52)};
\path[fill=color2,draw=color2,mark size=\marRad, mark=triangle*] plot coordinates {(0.09, 0.64)};
\path[fill=color2,draw=color2,mark size=\marRad, mark=triangle*] plot coordinates {(0.32, 0.10)};
\path[fill=color2,draw=color2,mark size=\marRad, mark=triangle*] plot coordinates {(0.47, 0.42)};
\path[fill=color2,draw=color2,mark size=\marRad, mark=triangle*] plot coordinates {(-0.01, 0.07)};
\path[fill=color2,draw=color2,mark size=\marRad, mark=triangle*] plot coordinates {(-0.31, 0.01)};
\path[fill=color2,draw=color2,mark size=\marRad, mark=triangle*] plot coordinates {(-0.21, 0.36)};
\path[fill=color2,draw=color2,mark size=\marRad, mark=triangle*] plot coordinates {(-0.22, 0.13)};
\path[fill=color2,draw=color2,mark size=\marRad, mark=triangle*] plot coordinates {(-0.44, 0.64)};
\path[fill=color2,draw=color2,mark size=\marRad, mark=triangle*] plot coordinates {(-0.33, 0.46)};
\path[fill=color2,draw=color2,mark size=\marRad, mark=triangle*] plot coordinates {(0.02, 0.12)};
\path[fill=color2,draw=color2,mark size=\marRad, mark=triangle*] plot coordinates {(0.06, 0.37)};
\path[fill=color2,draw=color2,mark size=\marRad, mark=triangle*] plot coordinates {(-0.03, 0.74)};
\path[fill=color2,draw=color2,mark size=\marRad, mark=triangle*] plot coordinates {(-0.35, 0.43)};
\path[fill=color2,draw=color2,mark size=\marRad, mark=triangle*] plot coordinates {(0.75, 0.21)};
\path[fill=color2,draw=color2,mark size=\marRad, mark=triangle*] plot coordinates {(0.39, 0.09)};
\path[fill=color2,draw=color2,mark size=\marRad, mark=triangle*] plot coordinates {(0.12, 0.41)};
\path[fill=color2,draw=color2,mark size=\marRad, mark=triangle*] plot coordinates {(0.40, 0.46)};
\path[fill=color2,draw=color2,mark size=\marRad, mark=triangle*] plot coordinates {(0.18, 0.05)};
\path[fill=color2,draw=color2,mark size=\marRad, mark=triangle*] plot coordinates {(-0.49, 0.61)};
\path[fill=color2,draw=color2,mark size=\marRad, mark=triangle*] plot coordinates {(-0.15, 0.33)};
\path[fill=color2,draw=color2,mark size=\marRad, mark=triangle*] plot coordinates {(0.58, 0.32)};
\path[fill=color2,draw=color2,mark size=\marRad, mark=triangle*] plot coordinates {(0.35, 0.41)};
\path[fill=color2,draw=color2,mark size=\marRad, mark=triangle*] plot coordinates {(0.16, 0.04)};
\path[fill=color2,draw=color2,mark size=\marRad, mark=triangle*] plot coordinates {(-0.11, 0.13)};
\path[fill=color2,draw=color2,mark size=\marRad, mark=triangle*] plot coordinates {(-0.01, -0.23)};
\path[fill=color2,draw=color2,mark size=\marRad, mark=triangle*] plot coordinates {(0.04, 0.22)};
\path[fill=color2,draw=color2,mark size=\marRad, mark=triangle*] plot coordinates {(0.77, 0.11)};
\path[fill=color2,draw=color2,mark size=\marRad, mark=triangle*] plot coordinates {(-0.01, 0.23)};
\path[fill=color2,draw=color2,mark size=\marRad, mark=triangle*] plot coordinates {(-0.15, -0.15)};
\path[fill=color2,draw=color2,mark size=\marRad, mark=triangle*] plot coordinates {(0.17, 0.68)};
\path[fill=color2,draw=color2,mark size=\marRad, mark=triangle*] plot coordinates {(-0.12, -0.21)};
\path[fill=color2,draw=color2,mark size=\marRad, mark=triangle*] plot coordinates {(0.24, 0.35)};
\path[fill=color2,draw=color2,mark size=\marRad, mark=triangle*] plot coordinates {(-0.33, 0.49)};
\path[fill=color2,draw=color2,mark size=\marRad, mark=triangle*] plot coordinates {(0.13, 0.21)};
\path[fill=color2,draw=color2,mark size=\marRad, mark=triangle*] plot coordinates {(0.08, -0.33)};
\path[fill=color2,draw=color2,mark size=\marRad, mark=triangle*] plot coordinates {(0.62, 0.26)};
\path[fill=color2,draw=color2,mark size=\marRad, mark=triangle*] plot coordinates {(0.28, 0.73)};
\path[fill=color2,draw=color2,mark size=\marRad, mark=triangle*] plot coordinates {(-0.15, -0.10)};
\path[fill=color2,draw=color2,mark size=\marRad, mark=triangle*] plot coordinates {(-0.12, 0.14)};
\path[fill=color2,draw=color2,mark size=\marRad, mark=triangle*] plot coordinates {(0.20, 0.70)};
\path[fill=color2,draw=color2,mark size=\marRad, mark=triangle*] plot coordinates {(0.41, 0.66)};
\path[fill=color2,draw=color2,mark size=\marRad, mark=triangle*] plot coordinates {(-0.34, 0.67)};
\path[fill=color2,draw=color2,mark size=\marRad, mark=triangle*] plot coordinates {(-0.27, -0.13)};
\path[fill=color2,draw=color2,mark size=\marRad, mark=triangle*] plot coordinates {(0.64, 0.30)};
\path[fill=color2,draw=color2,mark size=\marRad, mark=triangle*] plot coordinates {(-0.02, -0.11)};
\path[fill=color2,draw=color2,mark size=\marRad, mark=triangle*] plot coordinates {(-0.06, 0.36)};
\path[fill=color2,draw=color2,mark size=\marRad, mark=triangle*] plot coordinates {(0.20, -0.14)};
\path[fill=color2,draw=color2,mark size=\marRad, mark=triangle*] plot coordinates {(-0.09, 0.72)};
\path[fill=color2,draw=color2,mark size=\marRad, mark=triangle*] plot coordinates {(0.45, 0.59)};
\path[fill=color2,draw=color2,mark size=\marRad, mark=triangle*] plot coordinates {(0.62, -0.18)};
\path[fill=color2,draw=color2,mark size=\marRad, mark=triangle*] plot coordinates {(0.77, 0.06)};
\path[fill=color2,draw=color2,mark size=\marRad, mark=triangle*] plot coordinates {(0.31, 0.21)};
\path[fill=color2,draw=color2,mark size=\marRad, mark=triangle*] plot coordinates {(0.13, 0.58)};
\path[fill=color2,draw=color2,mark size=\marRad, mark=triangle*] plot coordinates {(0.29, 0.44)};
\path[fill=color2,draw=color2,mark size=\marRad, mark=triangle*] plot coordinates {(0.07, 0.31)};
\path[fill=color2,draw=color2,mark size=\marRad, mark=triangle*] plot coordinates {(-0.33, 0.37)};
\path[fill=color2,draw=color2,mark size=\marRad, mark=triangle*] plot coordinates {(0.18, 0.59)};
\path[fill=color2,draw=color2,mark size=\marRad, mark=triangle*] plot coordinates {(0.13, -0.30)};
\path[fill=color2,draw=color2,mark size=\marRad, mark=triangle*] plot coordinates {(-0.15, 0.09)};
\path[fill=color2,draw=color2,mark size=\marRad, mark=triangle*] plot coordinates {(0.59, 0.39)};
\path[fill=color2,draw=color2,mark size=\marRad, mark=triangle*] plot coordinates {(0.05, 0.01)};
\path[fill=color2,draw=color2,mark size=\marRad, mark=triangle*] plot coordinates {(-0.09, 0.59)};
\path[fill=color2,draw=color2,mark size=\marRad, mark=triangle*] plot coordinates {(-0.31, 0.68)};
\path[fill=color2,draw=color2,mark size=\marRad, mark=triangle*] plot coordinates {(0.62, 0.32)};
\path[fill=color2,draw=color2,mark size=\marRad, mark=triangle*] plot coordinates {(0.62, 0.12)};
\path[fill=color2,draw=color2,mark size=\marRad, mark=triangle*] plot coordinates {(0.54, 0.38)};
\path[fill=color2,draw=color2,mark size=\marRad, mark=triangle*] plot coordinates {(0.24, 0.27)};
\path[fill=color2,draw=color2,mark size=\marRad, mark=triangle*] plot coordinates {(0.01, 0.08)};
\path[fill=color2,draw=color2,mark size=\marRad, mark=triangle*] plot coordinates {(-0.30, 0.60)};
\path[fill=color2,draw=color2,mark size=\marRad, mark=triangle*] plot coordinates {(0.37, 0.40)};
\path[fill=color2,draw=color2,mark size=\marRad, mark=triangle*] plot coordinates {(0.32, 0.32)};
\path[fill=color2,draw=color2,mark size=\marRad, mark=triangle*] plot coordinates {(-0.27, 0.47)};
\path[fill=color2,draw=color2,mark size=\marRad, mark=triangle*] plot coordinates {(-0.22, 0.65)};
\path[fill=color2,draw=color2,mark size=\marRad, mark=triangle*] plot coordinates {(0.05, 0.65)};
\path[fill=color2,draw=color2,mark size=\marRad, mark=triangle*] plot coordinates {(-0.16, 0.19)};
\path[fill=color2,draw=color2,mark size=\marRad, mark=triangle*] plot coordinates {(0.24, -0.15)};
\path[fill=color2,draw=color2,mark size=\marRad, mark=triangle*] plot coordinates {(-0.42, 0.11)};
\path[fill=color2,draw=color2,mark size=\marRad, mark=triangle*] plot coordinates {(-0.11, -0.06)};
\path[fill=color2,draw=color2,mark size=\marRad, mark=triangle*] plot coordinates {(-0.37, 0.68)};
\path[fill=color2,draw=color2,mark size=\marRad, mark=triangle*] plot coordinates {(0.35, 0.20)};
\path[fill=color2,draw=color2,mark size=\marRad, mark=triangle*] plot coordinates {(-0.61, 0.33)};
\path[fill=color2,draw=color2,mark size=\marRad, mark=triangle*] plot coordinates {(0.39, -0.18)};
\path[fill=color2,draw=color2,mark size=\marRad, mark=triangle*] plot coordinates {(-0.08, 0.04)};
\path[fill=color2,draw=color2,mark size=\marRad, mark=triangle*] plot coordinates {(0.33, 0.61)};
\path[fill=color2,draw=color2,mark size=\marRad, mark=triangle*] plot coordinates {(0.69, 0.13)};
\path[fill=color2,draw=color2,mark size=\marRad, mark=triangle*] plot coordinates {(0.47, 0.18)};
\path[fill=color2,draw=color2,mark size=\marRad, mark=triangle*] plot coordinates {(-0.26, 0.26)};
\path[fill=color2,draw=color2,mark size=\marRad, mark=triangle*] plot coordinates {(0.43, 0.42)};
\path[fill=color2,draw=color2,mark size=\marRad, mark=triangle*] plot coordinates {(-0.15, 0.47)};
\path[fill=color2,draw=color2,mark size=\marRad, mark=triangle*] plot coordinates {(-0.03, 0.49)};
\path[fill=color2,draw=color2,mark size=\marRad, mark=triangle*] plot coordinates {(0.18, 0.01)};
\path[fill=color2,draw=color2,mark size=\marRad, mark=triangle*] plot coordinates {(0.66, -0.13)};
\path[fill=color2,draw=color2,mark size=\marRad, mark=triangle*] plot coordinates {(0.26, -0.06)};
\path[fill=color2,draw=color2,mark size=\marRad, mark=triangle*] plot coordinates {(-0.41, 0.40)};
\path[fill=color2,draw=color2,mark size=\marRad, mark=triangle*] plot coordinates {(0.02, 0.68)};
\path[fill=color2,draw=color2,mark size=\marRad, mark=triangle*] plot coordinates {(0.35, 0.58)};
\path[fill=color2,draw=color2,mark size=\marRad, mark=triangle*] plot coordinates {(-0.01, -0.36)};
\path[fill=color2,draw=color2,mark size=\marRad, mark=triangle*] plot coordinates {(0.29, 0.19)};
\path[fill=color2,draw=color2,mark size=\marRad, mark=triangle*] plot coordinates {(-0.52, 0.42)};
\path[fill=color2,draw=color2,mark size=\marRad, mark=triangle*] plot coordinates {(0.46, 0.54)};
\path[fill=color2,draw=color2,mark size=\marRad, mark=triangle*] plot coordinates {(0.48, 0.02)};
\path[fill=color2,draw=color2,mark size=\marRad, mark=triangle*] plot coordinates {(0.62, -0.19)};
\path[fill=color2,draw=color2,mark size=\marRad, mark=triangle*] plot coordinates {(0.04, -0.14)};
\path[fill=color2,draw=color2,mark size=\marRad, mark=triangle*] plot coordinates {(0.57, 0.10)};
\path[fill=color2,draw=color2,mark size=\marRad, mark=triangle*] plot coordinates {(0.11, 0.77)};
\path[fill=color2,draw=color2,mark size=\marRad, mark=triangle*] plot coordinates {(0.17, 0.46)};
\path[fill=color2,draw=color2,mark size=\marRad, mark=triangle*] plot coordinates {(-0.07, 0.58)};
\path[fill=color2,draw=color2,mark size=\marRad, mark=triangle*] plot coordinates {(0.58, 0.20)};
\path[fill=color2,draw=color2,mark size=\marRad, mark=triangle*] plot coordinates {(0.78, -0.02)};
\path[fill=color2,draw=color2,mark size=\marRad, mark=triangle*] plot coordinates {(-0.07, 0.35)};
\path[fill=color2,draw=color2,mark size=\marRad, mark=triangle*] plot coordinates {(-0.47, 0.15)};
\path[fill=color2,draw=color2,mark size=\marRad, mark=triangle*] plot coordinates {(-0.40, 0.29)};
\path[fill=color2,draw=color2,mark size=\marRad, mark=triangle*] plot coordinates {(0.56, 0.40)};
\path[fill=color2,draw=color2,mark size=\marRad, mark=triangle*] plot coordinates {(-0.05, 0.27)};
\path[fill=color2,draw=color2,mark size=\marRad, mark=triangle*] plot coordinates {(0.75, 0.15)};
\path[fill=color2,draw=color2,mark size=\marRad, mark=triangle*] plot coordinates {(0.31, 0.45)};
\path[fill=color2,draw=color2,mark size=\marRad, mark=triangle*] plot coordinates {(0.16, -0.31)};
\path[fill=color2,draw=color2,mark size=\marRad, mark=triangle*] plot coordinates {(-0.20, 0.67)};
\path[fill=color2,draw=color2,mark size=\marRad, mark=triangle*] plot coordinates {(-0.05, 0.35)};
\path[fill=color2,draw=color2,mark size=\marRad, mark=triangle*] plot coordinates {(-0.25, 0.75)};
\path[fill=color2,draw=color2,mark size=\marRad, mark=triangle*] plot coordinates {(-0.37, 0.52)};
\path[fill=color2,draw=color2,mark size=\marRad, mark=triangle*] plot coordinates {(0.50, 0.12)};
\path[fill=color2,draw=color2,mark size=\marRad, mark=triangle*] plot coordinates {(0.03, 0.58)};
\path[fill=color2,draw=color2,mark size=\marRad, mark=triangle*] plot coordinates {(-0.41, 0.41)};
\path[fill=color2,draw=color2,mark size=\marRad, mark=triangle*] plot coordinates {(0.15, 0.58)};
\path[fill=color2,draw=color2,mark size=\marRad, mark=triangle*] plot coordinates {(-0.04, -0.09)};
\path[fill=color2,draw=color2,mark size=\marRad, mark=triangle*] plot coordinates {(0.19, 0.36)};
\path[fill=color2,draw=color2,mark size=\marRad, mark=triangle*] plot coordinates {(0.42, -0.14)};
\path[fill=color2,draw=color2,mark size=\marRad, mark=triangle*] plot coordinates {(0.12, -0.19)};
\path[fill=color2,draw=color2,mark size=\marRad, mark=triangle*] plot coordinates {(0.31, 0.67)};
\path[fill=color2,draw=color2,mark size=\marRad, mark=triangle*] plot coordinates {(0.14, -0.14)};
\path[fill=color2,draw=color2,mark size=\marRad, mark=triangle*] plot coordinates {(-0.11, -0.19)};
\path[fill=color6,draw=color6,mark size=\marRad, mark=*] plot coordinates {(-0.34, -0.41)};
\path[fill=color6,draw=color6,mark size=\marRad, mark=*] plot coordinates {(-0.32, -0.24)};
\path[fill=color6,draw=color6,mark size=\marRad, mark=*] plot coordinates {(-0.26, -0.25)};
\path[fill=color6,draw=color6,mark size=\marRad, mark=*] plot coordinates {(-0.48, -0.51)};
\path[fill=color6,draw=color6,mark size=\marRad, mark=*] plot coordinates {(-0.51, -0.35)};
\path[fill=color6,draw=color6,mark size=\marRad, mark=*] plot coordinates {(-0.43, -0.29)};
\path[fill=color6,draw=color6,mark size=\marRad, mark=*] plot coordinates {(-0.60, -0.38)};
\path[fill=color6,draw=color6,mark size=\marRad, mark=*] plot coordinates {(-0.36, -0.30)};
\path[fill=color6,draw=color6,mark size=\marRad, mark=*] plot coordinates {(-0.55, -0.48)};
\path[fill=color6,draw=color6,mark size=\marRad, mark=*] plot coordinates {(-0.49, -0.50)};
\path[fill=color6,draw=color6,mark size=\marRad, mark=*] plot coordinates {(-0.49, -0.33)};
\path[fill=color6,draw=color6,mark size=\marRad, mark=*] plot coordinates {(-0.55, -0.48)};
\path[fill=color6,draw=color6,mark size=\marRad, mark=*] plot coordinates {(-0.49, -0.30)};
\path[fill=color6,draw=color6,mark size=\marRad, mark=*] plot coordinates {(-0.32, -0.25)};
\path[fill=color6,draw=color6,mark size=\marRad, mark=*] plot coordinates {(-0.30, -0.27)};
\path[fill=color6,draw=color6,mark size=\marRad, mark=*] plot coordinates {(-0.59, -0.36)};
\path[fill=color6,draw=color6,mark size=\marRad, mark=*] plot coordinates {(-0.52, -0.37)};
\path[fill=color6,draw=color6,mark size=\marRad, mark=*] plot coordinates {(-0.39, -0.42)};
\path[fill=color6,draw=color6,mark size=\marRad, mark=*] plot coordinates {(-0.56, -0.42)};
\path[fill=color6,draw=color6,mark size=\marRad, mark=*] plot coordinates {(-0.43, -0.35)};
\path[fill=color6,draw=color6,mark size=\marRad, mark=*] plot coordinates {(-0.51, -0.38)};
\path[fill=color6,draw=color6,mark size=\marRad, mark=*] plot coordinates {(-0.59, -0.43)};
\path[fill=color6,draw=color6,mark size=\marRad, mark=*] plot coordinates {(-0.38, -0.36)};
\path[fill=color6,draw=color6,mark size=\marRad, mark=*] plot coordinates {(-0.43, -0.34)};
\path[fill=color6,draw=color6,mark size=\marRad, mark=*] plot coordinates {(-0.58, -0.39)};
\path[fill=color6,draw=color6,mark size=\marRad, mark=*] plot coordinates {(-0.54, -0.35)};
\path[fill=color6,draw=color6,mark size=\marRad, mark=*] plot coordinates {(-0.36, -0.34)};
\path[fill=color6,draw=color6,mark size=\marRad, mark=*] plot coordinates {(-0.41, -0.35)};
\path[fill=color6,draw=color6,mark size=\marRad, mark=*] plot coordinates {(-0.39, -0.36)};
\path[fill=color6,draw=color6,mark size=\marRad, mark=*] plot coordinates {(-0.37, -0.33)};
\path[fill=color6,draw=color6,mark size=\marRad, mark=*] plot coordinates {(-0.60, -0.44)};
\path[fill=color6,draw=color6,mark size=\marRad, mark=*] plot coordinates {(-0.39, -0.36)};
\path[fill=color6,draw=color6,mark size=\marRad, mark=*] plot coordinates {(-0.46, -0.44)};
\path[fill=color6,draw=color6,mark size=\marRad, mark=*] plot coordinates {(-0.41, -0.35)};
\path[fill=color6,draw=color6,mark size=\marRad, mark=*] plot coordinates {(-0.47, -0.46)};
\path[fill=color6,draw=color6,mark size=\marRad, mark=*] plot coordinates {(-0.38, -0.36)};
\path[fill=color7,draw=color7,mark size=\marRad, mark=square*] plot coordinates {(-0.27, -0.71)};
\path[fill=color7,draw=color7,mark size=\marRad, mark=square*] plot coordinates {(-0.06, -0.70)};
\path[fill=color7,draw=color7,mark size=\marRad, mark=square*] plot coordinates {(-0.09, -0.63)};
\path[fill=color7,draw=color7,mark size=\marRad, mark=square*] plot coordinates {(-0.10, -0.46)};
\path[fill=color7,draw=color7,mark size=\marRad, mark=square*] plot coordinates {(-0.24, -0.69)};
\path[fill=color7,draw=color7,mark size=\marRad, mark=square*] plot coordinates {(-0.39, -0.62)};
\path[fill=color7,draw=color7,mark size=\marRad, mark=square*] plot coordinates {(-0.35, -0.67)};
\path[fill=color7,draw=color7,mark size=\marRad, mark=square*] plot coordinates {(-0.16, -0.62)};
\path[fill=color7,draw=color7,mark size=\marRad, mark=square*] plot coordinates {(-0.13, -0.50)};
\path[fill=color7,draw=color7,mark size=\marRad, mark=square*] plot coordinates {(-0.26, -0.49)};
\path[fill=color7,draw=color7,mark size=\marRad, mark=square*] plot coordinates {(-0.35, -0.54)};
\path[fill=color7,draw=color7,mark size=\marRad, mark=square*] plot coordinates {(-0.45, -0.60)};
\path[fill=color7,draw=color7,mark size=\marRad, mark=square*] plot coordinates {(-0.31, -0.54)};
\path[fill=color7,draw=color7,mark size=\marRad, mark=square*] plot coordinates {(-0.02, -0.67)};
\path[fill=color7,draw=color7,mark size=\marRad, mark=square*] plot coordinates {(-0.18, -0.59)};
\path[fill=color7,draw=color7,mark size=\marRad, mark=square*] plot coordinates {(-0.10, -0.43)};
\path[fill=color7,draw=color7,mark size=\marRad, mark=square*] plot coordinates {(-0.01, -0.59)};
\path[fill=color7,draw=color7,mark size=\marRad, mark=square*] plot coordinates {(-0.21, -0.57)};
\path[fill=color7,draw=color7,mark size=\marRad, mark=square*] plot coordinates {(-0.22, -0.55)};
\path[fill=color7,draw=color7,mark size=\marRad, mark=square*] plot coordinates {(-0.13, -0.75)};
\path[fill=color7,draw=color7,mark size=\marRad, mark=square*] plot coordinates {(-0.15, -0.44)};
\path[fill=color7,draw=color7,mark size=\marRad, mark=square*] plot coordinates {(-0.08, -0.46)};
\path[fill=color7,draw=color7,mark size=\marRad, mark=square*] plot coordinates {(-0.09, -0.58)};
\path[fill=color7,draw=color7,mark size=\marRad, mark=square*] plot coordinates {(-0.11, -0.51)};
\path[fill=color7,draw=color7,mark size=\marRad, mark=square*] plot coordinates {(-0.08, -0.45)};
\path[fill=color7,draw=color7,mark size=\marRad, mark=square*] plot coordinates {(-0.03, -0.43)};
\path[fill=color7,draw=color7,mark size=\marRad, mark=square*] plot coordinates {(-0.33, -0.62)};
\path[fill=color7,draw=color7,mark size=\marRad, mark=square*] plot coordinates {(-0.29, -0.65)};
\path[fill=color7,draw=color7,mark size=\marRad, mark=square*] plot coordinates {(-0.12, -0.45)};
\path[fill=color7,draw=color7,mark size=\marRad, mark=square*] plot coordinates {(-0.05, -0.42)};
\path[fill=color7,draw=color7,mark size=\marRad, mark=square*] plot coordinates {(-0.14, -0.32)};
\path[fill=color7,draw=color7,mark size=\marRad, mark=square*] plot coordinates {(-0.15, -0.41)};
\path[fill=color7,draw=color7,mark size=\marRad, mark=square*] plot coordinates {(-0.15, -0.36)};
\path[fill=color7,draw=color7,mark size=\marRad, mark=square*] plot coordinates {(-0.29, -0.60)};
\path[fill=color7,draw=color7,mark size=\marRad, mark=square*] plot coordinates {(-0.38, -0.66)};
\path[fill=color7,draw=color7,mark size=\marRad, mark=square*] plot coordinates {(-0.12, -0.50)};
\path[fill=color7,draw=color7,mark size=\marRad, mark=square*] plot coordinates {(-0.10, -0.29)};
\path[fill=color7,draw=color7,mark size=\marRad, mark=square*] plot coordinates {(-0.08, -0.45)};
\path[fill=color7,draw=color7,mark size=\marRad, mark=square*] plot coordinates {(-0.44, -0.61)};
\path[fill=color7,draw=color7,mark size=\marRad, mark=square*] plot coordinates {(-0.26, -0.47)};
\path[fill=color7,draw=color7,mark size=\marRad, mark=square*] plot coordinates {(-0.06, -0.64)};
\path[fill=color7,draw=color7,mark size=\marRad, mark=square*] plot coordinates {(-0.10, -0.40)};
\path[fill=color7,draw=color7,mark size=\marRad, mark=square*] plot coordinates {(-0.06, -0.56)};
\path[fill=color7,draw=color7,mark size=\marRad, mark=square*] plot coordinates {(-0.22, -0.53)};
\path[fill=color7,draw=color7,mark size=\marRad, mark=square*] plot coordinates {(-0.03, -0.60)};
\path[fill=color7,draw=color7,mark size=\marRad, mark=square*] plot coordinates {(-0.12, -0.28)};
\path[fill=color7,draw=color7,mark size=\marRad, mark=square*] plot coordinates {(-0.27, -0.62)};
\path[fill=color7,draw=color7,mark size=\marRad, mark=square*] plot coordinates {(-0.26, -0.43)};
\path[fill=color7,draw=color7,mark size=\marRad, mark=square*] plot coordinates {(-0.11, -0.63)};
\path[fill=color7,draw=color7,mark size=\marRad, mark=square*] plot coordinates {(-0.03, -0.72)};
\path[fill=color7,draw=color7,mark size=\marRad, mark=square*] plot coordinates {(-0.11, -0.55)};
\path[fill=color7,draw=color7,mark size=\marRad, mark=square*] plot coordinates {(-0.17, -0.72)};
\path[fill=color7,draw=color7,mark size=\marRad, mark=square*] plot coordinates {(-0.35, -0.63)};
\path[fill=color7,draw=color7,mark size=\marRad, mark=square*] plot coordinates {(-0.17, -0.49)};
\path[fill=color7,draw=color7,mark size=\marRad, mark=square*] plot coordinates {(-0.22, -0.67)};
\path[fill=color7,draw=color7,mark size=\marRad, mark=square*] plot coordinates {(-0.33, -0.70)};
\path[fill=color7,draw=color7,mark size=\marRad, mark=square*] plot coordinates {(-0.23, -0.72)};
\path[fill=color7,draw=color7,mark size=\marRad, mark=square*] plot coordinates {(-0.19, -0.72)};
\path[fill=color7,draw=color7,mark size=\marRad, mark=square*] plot coordinates {(-0.12, -0.70)};
\path[fill=color7,draw=color7,mark size=\marRad, mark=square*] plot coordinates {(-0.39, -0.58)};
\path[fill=color7,draw=color7,mark size=\marRad, mark=square*] plot coordinates {(-0.42, -0.56)};
\path[fill=color7,draw=color7,mark size=\marRad, mark=square*] plot coordinates {(-0.12, -0.68)};
\path[fill=color7,draw=color7,mark size=\marRad, mark=square*] plot coordinates {(-0.37, -0.56)};
\path[fill=color7,draw=color7,mark size=\marRad, mark=square*] plot coordinates {(-0.25, -0.67)};
\path[fill=color7,draw=color7,mark size=\marRad, mark=square*] plot coordinates {(-0.37, -0.55)};
\path[fill=color7,draw=color7,mark size=\marRad, mark=square*] plot coordinates {(-0.11, -0.72)};
\path[fill=color7,draw=color7,mark size=\marRad, mark=square*] plot coordinates {(-0.42, -0.66)};
\path[fill=color7,draw=color7,mark size=\marRad, mark=square*] plot coordinates {(-0.16, -0.67)};
\path[fill=color7,draw=color7,mark size=\marRad, mark=square*] plot coordinates {(-0.29, -0.56)};
\path[fill=color7,draw=color7,mark size=\marRad, mark=square*] plot coordinates {(-0.29, -0.49)};
\path[fill=color7,draw=color7,mark size=\marRad, mark=square*] plot coordinates {(-0.08, -0.71)};
\path[fill=color7,draw=color7,mark size=\marRad, mark=square*] plot coordinates {(-0.36, -0.59)};
\path[fill=color7,draw=color7,mark size=\marRad, mark=square*] plot coordinates {(-0.10, -0.74)};
\path[fill=color4,draw=color4,mark size=\marRad, mark=pentagon*] plot coordinates {(-0.61, 0.20)};
\path[fill=color4,draw=color4,mark size=\marRad, mark=pentagon*] plot coordinates {(-0.57, 0.21)};
\path[fill=color4,draw=color4,mark size=\marRad, mark=pentagon*] plot coordinates {(-0.53, 0.19)};
\path[fill=color4,draw=color4,mark size=\marRad, mark=pentagon*] plot coordinates {(-0.58, 0.24)};
\path[fill=color4,draw=color4,mark size=\marRad, mark=pentagon*] plot coordinates {(-0.70, 0.19)};
\path[fill=color4,draw=color4,mark size=\marRad, mark=pentagon*] plot coordinates {(-0.70, 0.20)};
\path[fill=color4,draw=color4,mark size=\marRad, mark=pentagon*] plot coordinates {(-0.71, 0.16)};
\path[fill=color4,draw=color4,mark size=\marRad, mark=pentagon*] plot coordinates {(-0.74, 0.16)};
\path[fill=color4,draw=color4,mark size=\marRad, mark=pentagon*] plot coordinates {(-0.68, 0.21)};
\path[fill=color4,draw=color4,mark size=\marRad, mark=pentagon*] plot coordinates {(-0.69, 0.17)};
\path[fill=color4,draw=color4,mark size=\marRad, mark=pentagon*] plot coordinates {(-0.70, 0.21)};
\path[fill=color4,draw=color4,mark size=\marRad, mark=pentagon*] plot coordinates {(-0.69, 0.21)};
\path[fill=color4,draw=color4,mark size=\marRad, mark=pentagon*] plot coordinates {(-0.68, 0.25)};
\path[fill=color4,draw=color4,mark size=\marRad, mark=pentagon*] plot coordinates {(-0.70, 0.12)};
\path[fill=color4,draw=color4,mark size=\marRad, mark=pentagon*] plot coordinates {(-0.70, 0.12)};
\path[fill=color4,draw=color4,mark size=\marRad, mark=pentagon*] plot coordinates {(-0.54, 0.19)};
\path[fill=color4,draw=color4,mark size=\marRad, mark=pentagon*] plot coordinates {(-0.51, 0.15)};
\path[fill=color4,draw=color4,mark size=\marRad, mark=pentagon*] plot coordinates {(-0.51, 0.17)};
\path[fill=color4,draw=color4,mark size=\marRad, mark=pentagon*] plot coordinates {(-0.47, 0.13)};
\path[fill=color4,draw=color4,mark size=\marRad, mark=pentagon*] plot coordinates {(-0.69, 0.21)};
\path[fill=color4,draw=color4,mark size=\marRad, mark=pentagon*] plot coordinates {(-0.71, 0.22)};
\path[fill=color4,draw=color4,mark size=\marRad, mark=pentagon*] plot coordinates {(-0.51, 0.22)};
\path[fill=color4,draw=color4,mark size=\marRad, mark=pentagon*] plot coordinates {(-0.69, 0.21)};
\path[fill=color4,draw=color4,mark size=\marRad, mark=pentagon*] plot coordinates {(-0.51, 0.21)};
\path[fill=color4,draw=color4,mark size=\marRad, mark=pentagon*] plot coordinates {(-0.70, 0.14)};
\path[fill=color4,draw=color4,mark size=\marRad, mark=pentagon*] plot coordinates {(-0.70, 0.14)};
\path[fill=color5,draw=color5,mark size=\marRad, mark=heart] plot coordinates {(-0.50, -0.06)};
\path[fill=color5,draw=color5,mark size=\marRad, mark=heart] plot coordinates {(-0.75, -0.20)};
\path[fill=color5,draw=color5,mark size=\marRad, mark=heart] plot coordinates {(-0.67, -0.17)};
\path[fill=color5,draw=color5,mark size=\marRad, mark=heart] plot coordinates {(-0.50, -0.06)};
\path[fill=color5,draw=color5,mark size=\marRad, mark=heart] plot coordinates {(-0.82, -0.03)};
\path[fill=color5,draw=color5,mark size=\marRad, mark=heart] plot coordinates {(-0.50, -0.07)};
\path[fill=color5,draw=color5,mark size=\marRad, mark=heart] plot coordinates {(-0.50, -0.05)};
\path[fill=color5,draw=color5,mark size=\marRad, mark=heart] plot coordinates {(-0.50, -0.05)};
\path[fill=color5,draw=color5,mark size=\marRad, mark=heart] plot coordinates {(-0.50, -0.06)};
\path[fill=color5,draw=color5,mark size=\marRad, mark=heart] plot coordinates {(-0.49, -0.07)};
\path[fill=color5,draw=color5,mark size=\marRad, mark=heart] plot coordinates {(-0.49, -0.06)};
\path[fill=color5,draw=color5,mark size=\marRad, mark=heart] plot coordinates {(-0.50, -0.06)};
\path[fill=color5,draw=color5,mark size=\marRad, mark=heart] plot coordinates {(-0.50, -0.05)};
\path[fill=color5,draw=color5,mark size=\marRad, mark=heart] plot coordinates {(-0.65, -0.36)};
\path[fill=color5,draw=color5,mark size=\marRad, mark=heart] plot coordinates {(-0.68, -0.30)};
\path[fill=color5,draw=color5,mark size=\marRad, mark=heart] plot coordinates {(-0.50, -0.05)};
\path[fill=color5,draw=color5,mark size=\marRad, mark=heart] plot coordinates {(-0.78, -0.10)};
\path[fill=color5,draw=color5,mark size=\marRad, mark=heart] plot coordinates {(-0.49, -0.07)};
\path[fill=color5,draw=color5,mark size=\marRad, mark=heart] plot coordinates {(-0.50, -0.05)};
\path[fill=color5,draw=color5,mark size=\marRad, mark=heart] plot coordinates {(-0.72, -0.21)};
\path[fill=color5,draw=color5,mark size=\marRad, mark=heart] plot coordinates {(-0.50, -0.05)};
\path[fill=color5,draw=color5,mark size=\marRad, mark=heart] plot coordinates {(-0.49, -0.06)};
\path[fill=color5,draw=color5,mark size=\marRad, mark=heart] plot coordinates {(-0.67, -0.35)};
\path[fill=color5,draw=color5,mark size=\marRad, mark=heart] plot coordinates {(-0.57, -0.20)};
\path[fill=color5,draw=color5,mark size=\marRad, mark=heart] plot coordinates {(-0.69, -0.10)};
\path[fill=color5,draw=color5,mark size=\marRad, mark=heart] plot coordinates {(-0.49, -0.06)};
\path[fill=color5,draw=color5,mark size=\marRad, mark=heart] plot coordinates {(-0.75, 0.02)};
\path[fill=color5,draw=color5,mark size=\marRad, mark=heart] plot coordinates {(-0.50, -0.06)};
\path[fill=color5,draw=color5,mark size=\marRad, mark=heart] plot coordinates {(-0.63, -0.00)};
\path[fill=color5,draw=color5,mark size=\marRad, mark=heart] plot coordinates {(-0.64, -0.13)};
\path[fill=color5,draw=color5,mark size=\marRad, mark=heart] plot coordinates {(-0.58, -0.06)};
\path[fill=color5,draw=color5,mark size=\marRad, mark=heart] plot coordinates {(-0.50, -0.06)};
\path[fill=color5,draw=color5,mark size=\marRad, mark=heart] plot coordinates {(-0.50, -0.05)};
\path[fill=color5,draw=color5,mark size=\marRad, mark=heart] plot coordinates {(-0.74, 0.02)};
\path[fill=color5,draw=color5,mark size=\marRad, mark=heart] plot coordinates {(-0.50, -0.05)};
\path[fill=color5,draw=color5,mark size=\marRad, mark=heart] plot coordinates {(-0.72, -0.18)};
\path[fill=color5,draw=color5,mark size=\marRad, mark=heart] plot coordinates {(-0.71, 0.00)};
\path[fill=color5,draw=color5,mark size=\marRad, mark=heart] plot coordinates {(-0.50, -0.06)};
\path[fill=color5,draw=color5,mark size=\marRad, mark=heart] plot coordinates {(-0.74, -0.04)};
\path[fill=color5,draw=color5,mark size=\marRad, mark=heart] plot coordinates {(-0.50, -0.05)};
\path[fill=color5,draw=color5,mark size=\marRad, mark=heart] plot coordinates {(-0.50, -0.06)};
\path[fill=color5,draw=color5,mark size=\marRad, mark=heart] plot coordinates {(-0.50, -0.06)};
\path[fill=color5,draw=color5,mark size=\marRad, mark=heart] plot coordinates {(-0.62, -0.24)};
\path[fill=color5,draw=color5,mark size=\marRad, mark=heart] plot coordinates {(-0.70, -0.21)};
\path[fill=color5,draw=color5,mark size=\marRad, mark=heart] plot coordinates {(-0.60, -0.14)};
\path[fill=color5,draw=color5,mark size=\marRad, mark=heart] plot coordinates {(-0.53, -0.18)};
\path[fill=color5,draw=color5,mark size=\marRad, mark=heart] plot coordinates {(-0.47, -0.18)};
\path[fill=color5,draw=color5,mark size=\marRad, mark=heart] plot coordinates {(-0.50, -0.05)};
\path[fill=color5,draw=color5,mark size=\marRad, mark=heart] plot coordinates {(-0.51, -0.18)};
\path[fill=color5,draw=color5,mark size=\marRad, mark=heart] plot coordinates {(-0.70, -0.05)};
\path[fill=color5,draw=color5,mark size=\marRad, mark=heart] plot coordinates {(-0.71, -0.07)};
\path[fill=color5,draw=color5,mark size=\marRad, mark=heart] plot coordinates {(-0.65, -0.31)};
\path[fill=color5,draw=color5,mark size=\marRad, mark=heart] plot coordinates {(-0.67, -0.22)};
\path[fill=color5,draw=color5,mark size=\marRad, mark=heart] plot coordinates {(-0.77, -0.14)};
\path[fill=color5,draw=color5,mark size=\marRad, mark=heart] plot coordinates {(-0.49, -0.07)};
\path[fill=color5,draw=color5,mark size=\marRad, mark=heart] plot coordinates {(-0.50, -0.06)};
\path[fill=color5,draw=color5,mark size=\marRad, mark=heart] plot coordinates {(-0.41, -0.09)};
\path[fill=color5,draw=color5,mark size=\marRad, mark=heart] plot coordinates {(-0.49, -0.07)};
\path[fill=color5,draw=color5,mark size=\marRad, mark=heart] plot coordinates {(-0.50, -0.06)};
\path[fill=color5,draw=color5,mark size=\marRad, mark=heart] plot coordinates {(-0.50, -0.05)};
\path[fill=color5,draw=color5,mark size=\marRad, mark=heart] plot coordinates {(-0.50, -0.19)};
\path[fill=color5,draw=color5,mark size=\marRad, mark=heart] plot coordinates {(-0.49, -0.07)};
\path[fill=color5,draw=color5,mark size=\marRad, mark=heart] plot coordinates {(-0.54, -0.06)};
\path[fill=color5,draw=color5,mark size=\marRad, mark=heart] plot coordinates {(-0.52, -0.19)};
\path[fill=color5,draw=color5,mark size=\marRad, mark=heart] plot coordinates {(-0.55, -0.20)};
\path[fill=color5,draw=color5,mark size=\marRad, mark=heart] plot coordinates {(-0.72, -0.12)};
\path[fill=color5,draw=color5,mark size=\marRad, mark=heart] plot coordinates {(-0.47, -0.19)};
\path[fill=color3,draw=color3,mark size=\marRad, mark=diamond*] plot coordinates {(-0.47, 0.31)};
\path[fill=color3,draw=color3,mark size=\marRad, mark=diamond*] plot coordinates {(-0.55, 0.48)};
\path[fill=color3,draw=color3,mark size=\marRad, mark=diamond*] plot coordinates {(-0.58, 0.43)};
\path[fill=color3,draw=color3,mark size=\marRad, mark=diamond*] plot coordinates {(-0.55, 0.43)};
\path[fill=color3,draw=color3,mark size=\marRad, mark=diamond*] plot coordinates {(-0.56, 0.41)};
\path[fill=color3,draw=color3,mark size=\marRad, mark=diamond*] plot coordinates {(-0.55, 0.43)};
\path[fill=color3,draw=color3,mark size=\marRad, mark=diamond*] plot coordinates {(-0.59, 0.35)};
\path[fill=color3,draw=color3,mark size=\marRad, mark=diamond*] plot coordinates {(-0.59, 0.37)};
\path[fill=color3,draw=color3,mark size=\marRad, mark=diamond*] plot coordinates {(-0.59, 0.37)};
\path[fill=color3,draw=color3,mark size=\marRad, mark=diamond*] plot coordinates {(-0.52, 0.31)};
\path[fill=color3,draw=color3,mark size=\marRad, mark=diamond*] plot coordinates {(-0.47, 0.47)};
\path[fill=color3,draw=color3,mark size=\marRad, mark=diamond*] plot coordinates {(-0.57, 0.39)};
\path[fill=color3,draw=color3,mark size=\marRad, mark=diamond*] plot coordinates {(-0.59, 0.36)};
\path[fill=color3,draw=color3,mark size=\marRad, mark=diamond*] plot coordinates {(-0.48, 0.32)};
\path[fill=color3,draw=color3,mark size=\marRad, mark=diamond*] plot coordinates {(-0.57, 0.40)};
\path[fill=color3,draw=color3,mark size=\marRad, mark=diamond*] plot coordinates {(-0.61, 0.42)};
\path[fill=color3,draw=color3,mark size=\marRad, mark=diamond*] plot coordinates {(-0.58, 0.32)};
\path[fill=color3,draw=color3,mark size=\marRad, mark=diamond*] plot coordinates {(-0.45, 0.37)};
\path[fill=color3,draw=color3,mark size=\marRad, mark=diamond*] plot coordinates {(-0.61, 0.43)};
\path[fill=color3,draw=color3,mark size=\marRad, mark=diamond*] plot coordinates {(-0.64, 0.44)};
\path[fill=color3,draw=color3,mark size=\marRad, mark=diamond*] plot coordinates {(-0.58, 0.41)};
\path[fill=color8,draw=color8,mark size=\marRad, mark=triangle*] plot coordinates {(0.09, -0.33)};
\path[fill=color8,draw=color8,mark size=\marRad, mark=triangle*] plot coordinates {(0.18, -0.74)};
\path[fill=color8,draw=color8,mark size=\marRad, mark=triangle*] plot coordinates {(0.17, -0.70)};
\path[fill=color8,draw=color8,mark size=\marRad, mark=triangle*] plot coordinates {(0.11, -0.71)};
\path[fill=color8,draw=color8,mark size=\marRad, mark=triangle*] plot coordinates {(0.09, -0.71)};
\path[fill=color8,draw=color8,mark size=\marRad, mark=triangle*] plot coordinates {(0.10, -0.37)};
\path[fill=color8,draw=color8,mark size=\marRad, mark=triangle*] plot coordinates {(0.10, -0.35)};
\path[fill=color8,draw=color8,mark size=\marRad, mark=triangle*] plot coordinates {(0.08, -0.37)};
\path[fill=color8,draw=color8,mark size=\marRad, mark=triangle*] plot coordinates {(0.07, -0.51)};
\path[fill=color8,draw=color8,mark size=\marRad, mark=triangle*] plot coordinates {(0.16, -0.43)};
\path[fill=color8,draw=color8,mark size=\marRad, mark=triangle*] plot coordinates {(0.25, -0.64)};
\path[fill=color8,draw=color8,mark size=\marRad, mark=triangle*] plot coordinates {(0.30, -0.72)};
\path[fill=color8,draw=color8,mark size=\marRad, mark=triangle*] plot coordinates {(0.11, -0.58)};
\path[fill=color8,draw=color8,mark size=\marRad, mark=triangle*] plot coordinates {(0.18, -0.65)};
\path[fill=color8,draw=color8,mark size=\marRad, mark=triangle*] plot coordinates {(0.20, -0.60)};
\path[fill=color8,draw=color8,mark size=\marRad, mark=triangle*] plot coordinates {(0.09, -0.51)};
\path[fill=color8,draw=color8,mark size=\marRad, mark=triangle*] plot coordinates {(0.09, -0.37)};
\path[fill=color8,draw=color8,mark size=\marRad, mark=triangle*] plot coordinates {(0.30, -0.70)};
\path[fill=color8,draw=color8,mark size=\marRad, mark=triangle*] plot coordinates {(0.21, -0.53)};
\path[fill=color8,draw=color8,mark size=\marRad, mark=triangle*] plot coordinates {(0.09, -0.44)};
\path[fill=color8,draw=color8,mark size=\marRad, mark=triangle*] plot coordinates {(0.09, -0.72)};
\path[fill=color8,draw=color8,mark size=\marRad, mark=triangle*] plot coordinates {(0.20, -0.47)};
\path[fill=color8,draw=color8,mark size=\marRad, mark=triangle*] plot coordinates {(0.16, -0.54)};
\path[fill=color8,draw=color8,mark size=\marRad, mark=triangle*] plot coordinates {(0.31, -0.66)};
\path[fill=color8,draw=color8,mark size=\marRad, mark=triangle*] plot coordinates {(0.20, -0.47)};
\path[fill=color8,draw=color8,mark size=\marRad, mark=triangle*] plot coordinates {(0.19, -0.75)};
\path[fill=color8,draw=color8,mark size=\marRad, mark=triangle*] plot coordinates {(0.11, -0.69)};
\path[fill=color8,draw=color8,mark size=\marRad, mark=triangle*] plot coordinates {(0.31, -0.64)};
\path[fill=color8,draw=color8,mark size=\marRad, mark=triangle*] plot coordinates {(0.26, -0.73)};
\path[fill=color8,draw=color8,mark size=\marRad, mark=triangle*] plot coordinates {(0.24, -0.75)};
\path[fill=color8,draw=color8,mark size=\marRad, mark=triangle*] plot coordinates {(0.35, -0.63)};
\path[fill=color8,draw=color8,mark size=\marRad, mark=triangle*] plot coordinates {(0.05, -0.64)};
\path[fill=color8,draw=color8,mark size=\marRad, mark=triangle*] plot coordinates {(0.30, -0.68)};
\path[fill=color8,draw=color8,mark size=\marRad, mark=triangle*] plot coordinates {(0.34, -0.66)};
\path[fill=color8,draw=color8,mark size=\marRad, mark=triangle*] plot coordinates {(0.05, -0.72)};
\path[fill=color8,draw=color8,mark size=\marRad, mark=triangle*] plot coordinates {(0.05, -0.57)};
\path[fill=color8,draw=color8,mark size=\marRad, mark=triangle*] plot coordinates {(0.11, -0.44)};
\path[fill=color8,draw=color8,mark size=\marRad, mark=triangle*] plot coordinates {(0.22, -0.71)};
\path[fill=color8,draw=color8,mark size=\marRad, mark=triangle*] plot coordinates {(0.34, -0.62)};
\path[fill=color8,draw=color8,mark size=\marRad, mark=triangle*] plot coordinates {(0.19, -0.59)};
\path[fill=color8,draw=color8,mark size=\marRad, mark=triangle*] plot coordinates {(0.20, -0.72)};
\path[fill=color8,draw=color8,mark size=\marRad, mark=triangle*] plot coordinates {(0.11, -0.47)};
\path[fill=color8,draw=color8,mark size=\marRad, mark=triangle*] plot coordinates {(0.08, -0.64)};
\path[fill=color8,draw=color8,mark size=\marRad, mark=triangle*] plot coordinates {(0.17, -0.69)};
\path[fill=color8,draw=color8,mark size=\marRad, mark=triangle*] plot coordinates {(0.07, -0.35)};
\path[fill=color8,draw=color8,mark size=\marRad, mark=triangle*] plot coordinates {(0.28, -0.65)};
\path[fill=color8,draw=color8,mark size=\marRad, mark=triangle*] plot coordinates {(0.15, -0.53)};
\path[fill=color8,draw=color8,mark size=\marRad, mark=triangle*] plot coordinates {(0.13, -0.49)};
\path[fill=color8,draw=color8,mark size=\marRad, mark=triangle*] plot coordinates {(0.15, -0.77)};
\path[fill=color8,draw=color8,mark size=\marRad, mark=triangle*] plot coordinates {(0.11, -0.69)};
\path[fill=color8,draw=color8,mark size=\marRad, mark=triangle*] plot coordinates {(0.05, -0.75)};
\path[fill=color8,draw=color8,mark size=\marRad, mark=triangle*] plot coordinates {(0.13, -0.77)};
\path[fill=color8,draw=color8,mark size=\marRad, mark=triangle*] plot coordinates {(0.06, -0.34)};
\path[fill=color8,draw=color8,mark size=\marRad, mark=triangle*] plot coordinates {(0.01, -0.30)};
\path[fill=color1,draw=color1,mark size=\marRad, mark=square*] plot coordinates {(0.70, -0.19)};
\path[fill=color1,draw=color1,mark size=\marRad, mark=square*] plot coordinates {(0.79, -0.05)};
\path[fill=color1,draw=color1,mark size=\marRad, mark=square*] plot coordinates {(0.75, 0.01)};
\path[fill=color1,draw=color1,mark size=\marRad, mark=square*] plot coordinates {(0.71, -0.10)};
\path[fill=color1,draw=color1,mark size=\marRad, mark=square*] plot coordinates {(0.72, -0.12)};
\path[fill=color1,draw=color1,mark size=\marRad, mark=square*] plot coordinates {(0.73, -0.06)};
\path[fill=color1,draw=color1,mark size=\marRad, mark=square*] plot coordinates {(0.78, -0.11)};
\path[fill=color1,draw=color1,mark size=\marRad, mark=square*] plot coordinates {(0.57, -0.06)};
\path[fill=color1,draw=color1,mark size=\marRad, mark=square*] plot coordinates {(0.65, -0.09)};
\path[fill=color1,draw=color1,mark size=\marRad, mark=square*] plot coordinates {(0.70, -0.15)};
\path[fill=color9,draw=color9,mark size=\marRad, mark=diamond*] plot coordinates {(0.63, -0.30)};
\path[fill=color9,draw=color9,mark size=\marRad, mark=diamond*] plot coordinates {(0.57, -0.30)};
\path[fill=color9,draw=color9,mark size=\marRad, mark=diamond*] plot coordinates {(0.67, -0.41)};
\path[fill=color9,draw=color9,mark size=\marRad, mark=diamond*] plot coordinates {(0.60, -0.40)};
\path[fill=color9,draw=color9,mark size=\marRad, mark=diamond*] plot coordinates {(0.61, -0.45)};
\path[fill=color9,draw=color9,mark size=\marRad, mark=diamond*] plot coordinates {(0.38, -0.44)};
\path[fill=color9,draw=color9,mark size=\marRad, mark=diamond*] plot coordinates {(0.63, -0.47)};
\path[fill=color9,draw=color9,mark size=\marRad, mark=diamond*] plot coordinates {(0.52, -0.33)};
\path[fill=color9,draw=color9,mark size=\marRad, mark=diamond*] plot coordinates {(0.40, -0.50)};
\path[fill=color9,draw=color9,mark size=\marRad, mark=diamond*] plot coordinates {(0.50, -0.58)};
\path[fill=color9,draw=color9,mark size=\marRad, mark=diamond*] plot coordinates {(0.41, -0.44)};
\path[fill=color9,draw=color9,mark size=\marRad, mark=diamond*] plot coordinates {(0.54, -0.30)};
\path[fill=color9,draw=color9,mark size=\marRad, mark=diamond*] plot coordinates {(0.55, -0.56)};
\path[fill=color9,draw=color9,mark size=\marRad, mark=diamond*] plot coordinates {(0.40, -0.42)};
\path[fill=color9,draw=color9,mark size=\marRad, mark=diamond*] plot coordinates {(0.48, -0.52)};
\path[fill=color9,draw=color9,mark size=\marRad, mark=diamond*] plot coordinates {(0.40, -0.49)};
\path[fill=color9,draw=color9,mark size=\marRad, mark=diamond*] plot coordinates {(0.51, -0.19)};
\path[fill=color9,draw=color9,mark size=\marRad, mark=diamond*] plot coordinates {(0.52, -0.24)};
\path[fill=color9,draw=color9,mark size=\marRad, mark=diamond*] plot coordinates {(0.58, -0.52)};
\path[fill=color9,draw=color9,mark size=\marRad, mark=diamond*] plot coordinates {(0.41, -0.58)};
\path[fill=color9,draw=color9,mark size=\marRad, mark=diamond*] plot coordinates {(0.45, -0.37)};
\path[fill=color9,draw=color9,mark size=\marRad, mark=diamond*] plot coordinates {(0.52, -0.47)};
\path[fill=color9,draw=color9,mark size=\marRad, mark=diamond*] plot coordinates {(0.56, -0.37)};
\path[fill=color9,draw=color9,mark size=\marRad, mark=diamond*] plot coordinates {(0.51, -0.44)};
\path[fill=color9,draw=color9,mark size=\marRad, mark=diamond*] plot coordinates {(0.44, -0.38)};
\path[fill=color9,draw=color9,mark size=\marRad, mark=diamond*] plot coordinates {(0.49, -0.49)};
\path[fill=color9,draw=color9,mark size=\marRad, mark=diamond*] plot coordinates {(0.66, -0.28)};
\path[fill=color9,draw=color9,mark size=\marRad, mark=diamond*] plot coordinates {(0.59, -0.48)};
\path[fill=color9,draw=color9,mark size=\marRad, mark=diamond*] plot coordinates {(0.20, -0.36)};
\path[fill=color9,draw=color9,mark size=\marRad, mark=diamond*] plot coordinates {(0.34, -0.38)};
\path[fill=color9,draw=color9,mark size=\marRad, mark=diamond*] plot coordinates {(0.42, -0.32)};
\path[fill=color9,draw=color9,mark size=\marRad, mark=diamond*] plot coordinates {(0.37, -0.30)};
\path[fill=color9,draw=color9,mark size=\marRad, mark=diamond*] plot coordinates {(0.37, -0.30)};
\path[fill=color9,draw=color9,mark size=\marRad, mark=diamond*] plot coordinates {(0.36, -0.30)};
\path[fill=color9,draw=color9,mark size=\marRad, mark=diamond*] plot coordinates {(0.37, -0.30)};
\path[fill=color9,draw=color9,mark size=\marRad, mark=diamond*] plot coordinates {(0.37, -0.30)};
\path[fill=color9,draw=color9,mark size=\marRad, mark=diamond*] plot coordinates {(0.37, -0.30)};
\path[fill=color9,draw=color9,mark size=\marRad, mark=diamond*] plot coordinates {(0.37, -0.30)};
\path[fill=color9,draw=color9,mark size=\marRad, mark=diamond*] plot coordinates {(0.36, -0.30)};
\path[fill=color9,draw=color9,mark size=\marRad, mark=diamond*] plot coordinates {(0.38, -0.29)};
\path[fill=color9,draw=color9,mark size=\marRad, mark=diamond*] plot coordinates {(0.37, -0.30)};
\path[fill=color9,draw=color9,mark size=\marRad, mark=diamond*] plot coordinates {(0.37, -0.30)};
\path[fill=color9,draw=color9,mark size=\marRad, mark=diamond*] plot coordinates {(0.38, -0.29)};
\path[fill=color9,draw=color9,mark size=\marRad, mark=diamond*] plot coordinates {(0.37, -0.30)};
\path[fill=color9,draw=color9,mark size=\marRad, mark=diamond*] plot coordinates {(0.37, -0.30)};
\path[fill=color9,draw=color9,mark size=\marRad, mark=diamond*] plot coordinates {(0.37, -0.29)};
\path[fill=color9,draw=color9,mark size=\marRad, mark=diamond*] plot coordinates {(0.37, -0.30)};
\path[fill=color9,draw=color9,mark size=\marRad, mark=diamond*] plot coordinates {(0.38, -0.29)};
\path[fill=color9,draw=color9,mark size=\marRad, mark=diamond*] plot coordinates {(0.37, -0.30)};
\path[fill=color9,draw=color9,mark size=\marRad, mark=diamond*] plot coordinates {(0.37, -0.30)};
\path[fill=color9,draw=color9,mark size=\marRad, mark=diamond*] plot coordinates {(0.37, -0.30)};
\path[fill=color9,draw=color9,mark size=\marRad, mark=diamond*] plot coordinates {(0.38, -0.29)};
\path[fill=color9,draw=color9,mark size=\marRad, mark=diamond*] plot coordinates {(0.36, -0.30)};
\path[fill=color9,draw=color9,mark size=\marRad, mark=diamond*] plot coordinates {(0.37, -0.30)};
\path[fill=color9,draw=color9,mark size=\marRad, mark=diamond*] plot coordinates {(0.37, -0.29)};
\path[fill=color9,draw=color9,mark size=\marRad, mark=diamond*] plot coordinates {(0.37, -0.30)};
\path[fill=color9,draw=color9,mark size=\marRad, mark=diamond*] plot coordinates {(0.37, -0.30)};
\path[fill=color9,draw=color9,mark size=\marRad, mark=diamond*] plot coordinates {(0.38, -0.29)};
\path[fill=color9,draw=color9,mark size=\marRad, mark=diamond*] plot coordinates {(0.37, -0.30)};
\path[fill=color9,draw=color9,mark size=\marRad, mark=diamond*] plot coordinates {(0.37, -0.29)};
\path[fill=color9,draw=color9,mark size=\marRad, mark=diamond*] plot coordinates {(0.36, -0.30)};
\path[fill=color9,draw=color9,mark size=\marRad, mark=diamond*] plot coordinates {(0.37, -0.30)};
\path[fill=color9,draw=color9,mark size=\marRad, mark=diamond*] plot coordinates {(0.38, -0.29)};
\path[fill=color9,draw=color9,mark size=\marRad, mark=diamond*] plot coordinates {(0.38, -0.29)};
\path[fill=color9,draw=color9,mark size=\marRad, mark=diamond*] plot coordinates {(0.37, -0.29)};
\path[fill=color9,draw=color9,mark size=\marRad, mark=diamond*] plot coordinates {(0.36, -0.30)};
\path[fill=color9,draw=color9,mark size=\marRad, mark=diamond*] plot coordinates {(0.36, -0.30)};
\path[fill=color9,draw=color9,mark size=\marRad, mark=diamond*] plot coordinates {(0.52, -0.28)};
\path[fill=color9,draw=color9,mark size=\marRad, mark=diamond*] plot coordinates {(0.49, -0.43)};
\path[fill=color9,draw=color9,mark size=\marRad, mark=diamond*] plot coordinates {(0.46, -0.57)};
\path[fill=color9,draw=color9,mark size=\marRad, mark=diamond*] plot coordinates {(0.47, -0.30)};
\path[fill=color9,draw=color9,mark size=\marRad, mark=diamond*] plot coordinates {(0.46, -0.62)};
\path[fill=color9,draw=color9,mark size=\marRad, mark=diamond*] plot coordinates {(0.37, -0.29)};
\end{tikzpicture}
 
\end{frame}


\begin{frame}{Module - Clustering - Alternative Umsetzungen}
\begin{itemize}
    \item Statt Tweets Hashtags clustern und visuell darstellen.
\end{itemize}
\def\maxW{71.06}
\begin{tikzpicture}[y=\textwidth/120,x=\textwidth/\maxW, background rectangle/.style={draw=black, thick, fill=yellow!10,},show background rectangle]
\def\marRad{0.5mm}
\definecolor{color0}{rgb}{0.78,0.89,0.11}
\definecolor{color1}{rgb}{0.10,1.00,0.52}
\definecolor{color2}{rgb}{0.09,0.32,0.64}
\definecolor{color3}{rgb}{0.70,0.69,0.42}
\definecolor{color4}{rgb}{0.36,0.70,0.25}
\definecolor{color5}{rgb}{0.91,0.34,0.56}
\definecolor{color6}{rgb}{0.76,0.33,0.65}
\definecolor{color7}{rgb}{0.06,0.22,0.62}
\definecolor{color8}{rgb}{0.45,0.68,0.87}
\definecolor{color9}{rgb}{0.03,0.89,0.67}
\path[fill=color0,draw=color0,mark size=\marRad, mark=*] plot coordinates {(6.23, 1.82)};
\path[fill=color0,draw=color0,mark size=\marRad, mark=*] plot coordinates {(8.31, 3.01)};
\path[fill=color1,draw=color1,mark size=\marRad, mark=square*] plot coordinates {(50.87, 38.70)};
\path[fill=color2,draw=color2,mark size=\marRad, mark=triangle*] plot coordinates {(11.33, 15.07)};
\path[fill=color3,draw=color3,mark size=\marRad, mark=diamond*] plot coordinates {(3.13, 9.06)};
\path[fill=color6,draw=color6,mark size=\marRad, mark=*] plot coordinates {(-6.87, 4.44)};
\path[fill=color6,draw=color6,mark size=\marRad, mark=*] plot coordinates {(-2.59, 7.06)};
\path[fill=color6,draw=color6,mark size=\marRad, mark=*] plot coordinates {(-5.28, 3.47)};
\path[fill=color6,draw=color6,mark size=\marRad, mark=*] plot coordinates {(-2.92, -7.37)};
\path[fill=color6,draw=color6,mark size=\marRad, mark=*] plot coordinates {(2.58, -0.79)};
\path[fill=color6,draw=color6,mark size=\marRad, mark=*] plot coordinates {(-4.49, -0.18)};
\path[fill=color6,draw=color6,mark size=\marRad, mark=*] plot coordinates {(-1.63, -0.91)};
\path[fill=color6,draw=color6,mark size=\marRad, mark=*] plot coordinates {(-2.86, -2.86)};
\path[fill=color6,draw=color6,mark size=\marRad, mark=*] plot coordinates {(1.02, -0.67)};
\path[fill=color6,draw=color6,mark size=\marRad, mark=*] plot coordinates {(1.60, 3.05)};
\path[fill=color6,draw=color6,mark size=\marRad, mark=*] plot coordinates {(3.17, -1.78)};
\path[fill=color6,draw=color6,mark size=\marRad, mark=*] plot coordinates {(-3.26, 1.72)};
\path[fill=color6,draw=color6,mark size=\marRad, mark=*] plot coordinates {(-2.76, 0.48)};
\path[fill=color6,draw=color6,mark size=\marRad, mark=*] plot coordinates {(0.42, -1.28)};
\path[fill=color6,draw=color6,mark size=\marRad, mark=*] plot coordinates {(-4.64, -3.05)};
\path[fill=color6,draw=color6,mark size=\marRad, mark=*] plot coordinates {(1.14, 1.32)};
\path[fill=color6,draw=color6,mark size=\marRad, mark=*] plot coordinates {(4.61, -1.55)};
\path[fill=color6,draw=color6,mark size=\marRad, mark=*] plot coordinates {(-4.02, 3.38)};
\path[fill=color6,draw=color6,mark size=\marRad, mark=*] plot coordinates {(-1.52, 1.11)};
\path[fill=color6,draw=color6,mark size=\marRad, mark=*] plot coordinates {(-6.49, -0.80)};
\path[fill=color6,draw=color6,mark size=\marRad, mark=*] plot coordinates {(-2.78, -2.91)};
\path[fill=color6,draw=color6,mark size=\marRad, mark=*] plot coordinates {(-3.76, -1.06)};
\path[fill=color6,draw=color6,mark size=\marRad, mark=*] plot coordinates {(-3.61, -0.05)};
\path[fill=color6,draw=color6,mark size=\marRad, mark=*] plot coordinates {(-1.72, -3.75)};
\path[fill=color6,draw=color6,mark size=\marRad, mark=*] plot coordinates {(0.10, 1.13)};
\path[fill=color6,draw=color6,mark size=\marRad, mark=*] plot coordinates {(-2.18, -1.59)};
\path[fill=color6,draw=color6,mark size=\marRad, mark=*] plot coordinates {(-0.79, -2.12)};
\path[fill=color6,draw=color6,mark size=\marRad, mark=*] plot coordinates {(-0.37, 6.06)};
\path[fill=color6,draw=color6,mark size=\marRad, mark=*] plot coordinates {(0.21, 2.58)};
\path[fill=color6,draw=color6,mark size=\marRad, mark=*] plot coordinates {(1.79, -5.02)};
\path[fill=color6,draw=color6,mark size=\marRad, mark=*] plot coordinates {(3.21, 4.08)};
\path[fill=color6,draw=color6,mark size=\marRad, mark=*] plot coordinates {(0.96, -0.52)};
\path[fill=color6,draw=color6,mark size=\marRad, mark=*] plot coordinates {(0.85, 3.30)};
\path[fill=color6,draw=color6,mark size=\marRad, mark=*] plot coordinates {(-0.50, -1.82)};
\path[fill=color6,draw=color6,mark size=\marRad, mark=*] plot coordinates {(-1.63, -0.52)};
\path[fill=color6,draw=color6,mark size=\marRad, mark=*] plot coordinates {(0.60, 0.74)};
\path[fill=color6,draw=color6,mark size=\marRad, mark=*] plot coordinates {(0.76, -1.83)};
\path[fill=color6,draw=color6,mark size=\marRad, mark=*] plot coordinates {(-5.70, -6.13)};
\path[fill=color6,draw=color6,mark size=\marRad, mark=*] plot coordinates {(2.84, 1.65)};
\path[fill=color6,draw=color6,mark size=\marRad, mark=*] plot coordinates {(0.68, -0.37)};
\path[fill=color6,draw=color6,mark size=\marRad, mark=*] plot coordinates {(-0.26, -0.66)};
\path[fill=color6,draw=color6,mark size=\marRad, mark=*] plot coordinates {(-1.06, -1.20)};
\path[fill=color6,draw=color6,mark size=\marRad, mark=*] plot coordinates {(-1.42, -1.17)};
\path[fill=color6,draw=color6,mark size=\marRad, mark=*] plot coordinates {(-1.37, 0.20)};
\path[fill=color6,draw=color6,mark size=\marRad, mark=*] plot coordinates {(-2.19, 5.16)};
\path[fill=color6,draw=color6,mark size=\marRad, mark=*] plot coordinates {(-0.50, -1.47)};
\path[fill=color6,draw=color6,mark size=\marRad, mark=*] plot coordinates {(-0.44, -4.58)};
\path[fill=color6,draw=color6,mark size=\marRad, mark=*] plot coordinates {(3.61, -0.29)};
\path[fill=color6,draw=color6,mark size=\marRad, mark=*] plot coordinates {(1.42, -2.55)};
\path[fill=color6,draw=color6,mark size=\marRad, mark=*] plot coordinates {(3.47, 0.76)};
\path[fill=color6,draw=color6,mark size=\marRad, mark=*] plot coordinates {(-4.37, 1.15)};
\path[fill=color6,draw=color6,mark size=\marRad, mark=*] plot coordinates {(1.88, 0.16)};
\path[fill=color6,draw=color6,mark size=\marRad, mark=*] plot coordinates {(-2.44, -1.25)};
\path[fill=color6,draw=color6,mark size=\marRad, mark=*] plot coordinates {(1.39, -1.64)};
\path[fill=color6,draw=color6,mark size=\marRad, mark=*] plot coordinates {(-0.12, 1.80)};
\path[fill=color6,draw=color6,mark size=\marRad, mark=*] plot coordinates {(-3.14, 3.91)};
\path[fill=color6,draw=color6,mark size=\marRad, mark=*] plot coordinates {(-6.09, 2.30)};
\path[fill=color6,draw=color6,mark size=\marRad, mark=*] plot coordinates {(-6.12, 0.88)};
\path[fill=color6,draw=color6,mark size=\marRad, mark=*] plot coordinates {(-2.75, -0.63)};
\path[fill=color6,draw=color6,mark size=\marRad, mark=*] plot coordinates {(-1.59, -3.20)};
\path[fill=color6,draw=color6,mark size=\marRad, mark=*] plot coordinates {(-0.11, -2.94)};
\path[fill=color6,draw=color6,mark size=\marRad, mark=*] plot coordinates {(-6.53, -4.05)};
\path[fill=color6,draw=color6,mark size=\marRad, mark=*] plot coordinates {(2.38, 2.06)};
\path[fill=color6,draw=color6,mark size=\marRad, mark=*] plot coordinates {(-2.32, 3.51)};
\path[fill=color6,draw=color6,mark size=\marRad, mark=*] plot coordinates {(-3.13, -4.08)};
\path[fill=color6,draw=color6,mark size=\marRad, mark=*] plot coordinates {(1.39, 4.29)};
\path[fill=color6,draw=color6,mark size=\marRad, mark=*] plot coordinates {(-1.16, 1.24)};
\path[fill=color6,draw=color6,mark size=\marRad, mark=*] plot coordinates {(-2.93, 1.39)};
\path[fill=color6,draw=color6,mark size=\marRad, mark=*] plot coordinates {(-1.74, -2.58)};
\path[fill=color6,draw=color6,mark size=\marRad, mark=*] plot coordinates {(-1.58, -2.11)};
\path[fill=color6,draw=color6,mark size=\marRad, mark=*] plot coordinates {(-1.33, 1.36)};
\path[fill=color6,draw=color6,mark size=\marRad, mark=*] plot coordinates {(-7.52, 2.85)};
\path[fill=color6,draw=color6,mark size=\marRad, mark=*] plot coordinates {(0.14, -4.45)};
\path[fill=color6,draw=color6,mark size=\marRad, mark=*] plot coordinates {(0.86, -8.39)};
\path[fill=color6,draw=color6,mark size=\marRad, mark=*] plot coordinates {(-8.19, -1.10)};
\path[fill=color6,draw=color6,mark size=\marRad, mark=*] plot coordinates {(6.12, 2.64)};
\path[fill=color6,draw=color6,mark size=\marRad, mark=*] plot coordinates {(-8.09, 0.97)};
\path[fill=color6,draw=color6,mark size=\marRad, mark=*] plot coordinates {(-1.39, -7.59)};
\path[fill=color6,draw=color6,mark size=\marRad, mark=*] plot coordinates {(-1.12, 2.93)};
\path[fill=color6,draw=color6,mark size=\marRad, mark=*] plot coordinates {(1.37, -0.75)};
\path[fill=color6,draw=color6,mark size=\marRad, mark=*] plot coordinates {(-0.11, 1.07)};
\path[fill=color6,draw=color6,mark size=\marRad, mark=*] plot coordinates {(0.74, 0.48)};
\path[fill=color6,draw=color6,mark size=\marRad, mark=*] plot coordinates {(-3.11, -6.33)};
\path[fill=color6,draw=color6,mark size=\marRad, mark=*] plot coordinates {(-5.57, -4.42)};
\path[fill=color6,draw=color6,mark size=\marRad, mark=*] plot coordinates {(4.26, -4.87)};
\path[fill=color6,draw=color6,mark size=\marRad, mark=*] plot coordinates {(1.61, -6.57)};
\path[fill=color6,draw=color6,mark size=\marRad, mark=*] plot coordinates {(3.35, 4.66)};
\path[fill=color6,draw=color6,mark size=\marRad, mark=*] plot coordinates {(1.07, -3.05)};
\path[fill=color6,draw=color6,mark size=\marRad, mark=*] plot coordinates {(-2.27, -3.58)};
\path[fill=color6,draw=color6,mark size=\marRad, mark=*] plot coordinates {(0.52, -0.11)};
\path[fill=color6,draw=color6,mark size=\marRad, mark=*] plot coordinates {(-0.38, 2.95)};
\path[fill=color6,draw=color6,mark size=\marRad, mark=*] plot coordinates {(-1.58, 3.27)};
\path[fill=color6,draw=color6,mark size=\marRad, mark=*] plot coordinates {(1.13, -2.21)};
\path[fill=color6,draw=color6,mark size=\marRad, mark=*] plot coordinates {(2.09, -0.26)};
\path[fill=color6,draw=color6,mark size=\marRad, mark=*] plot coordinates {(2.60, 0.79)};
\path[fill=color6,draw=color6,mark size=\marRad, mark=*] plot coordinates {(-0.16, 0.95)};
\path[fill=color6,draw=color6,mark size=\marRad, mark=*] plot coordinates {(-0.41, -3.01)};
\path[fill=color6,draw=color6,mark size=\marRad, mark=*] plot coordinates {(1.73, 1.73)};
\path[fill=color6,draw=color6,mark size=\marRad, mark=*] plot coordinates {(-0.70, -0.06)};
\path[fill=color6,draw=color6,mark size=\marRad, mark=*] plot coordinates {(0.35, 0.77)};
\path[fill=color6,draw=color6,mark size=\marRad, mark=*] plot coordinates {(0.30, -1.81)};
\path[fill=color6,draw=color6,mark size=\marRad, mark=*] plot coordinates {(-0.72, -0.14)};
\path[fill=color6,draw=color6,mark size=\marRad, mark=*] plot coordinates {(-3.59, -3.45)};
\path[fill=color6,draw=color6,mark size=\marRad, mark=*] plot coordinates {(-3.76, -3.19)};
\path[fill=color6,draw=color6,mark size=\marRad, mark=*] plot coordinates {(3.08, -2.88)};
\path[fill=color6,draw=color6,mark size=\marRad, mark=*] plot coordinates {(0.39, 3.85)};
\path[fill=color6,draw=color6,mark size=\marRad, mark=*] plot coordinates {(-0.92, -1.24)};
\path[fill=color6,draw=color6,mark size=\marRad, mark=*] plot coordinates {(-1.55, -2.25)};
\path[fill=color6,draw=color6,mark size=\marRad, mark=*] plot coordinates {(1.16, -3.61)};
\path[fill=color6,draw=color6,mark size=\marRad, mark=*] plot coordinates {(-3.85, 0.17)};
\path[fill=color6,draw=color6,mark size=\marRad, mark=*] plot coordinates {(-4.32, -1.58)};
\path[fill=color6,draw=color6,mark size=\marRad, mark=*] plot coordinates {(-1.48, 1.59)};
\path[fill=color6,draw=color6,mark size=\marRad, mark=*] plot coordinates {(1.45, 1.02)};
\path[fill=color6,draw=color6,mark size=\marRad, mark=*] plot coordinates {(-0.00, 0.03)};
\path[fill=color6,draw=color6,mark size=\marRad, mark=*] plot coordinates {(0.51, 0.30)};
\path[fill=color6,draw=color6,mark size=\marRad, mark=*] plot coordinates {(-2.39, 0.31)};
\path[fill=color6,draw=color6,mark size=\marRad, mark=*] plot coordinates {(3.27, -2.12)};
\path[fill=color6,draw=color6,mark size=\marRad, mark=*] plot coordinates {(-1.03, -0.71)};
\path[fill=color6,draw=color6,mark size=\marRad, mark=*] plot coordinates {(0.11, -1.58)};
\path[fill=color6,draw=color6,mark size=\marRad, mark=*] plot coordinates {(3.03, -6.68)};
\path[fill=color6,draw=color6,mark size=\marRad, mark=*] plot coordinates {(-4.79, 3.22)};
\path[fill=color6,draw=color6,mark size=\marRad, mark=*] plot coordinates {(0.31, -3.02)};
\path[fill=color6,draw=color6,mark size=\marRad, mark=*] plot coordinates {(3.63, -3.65)};
\path[fill=color6,draw=color6,mark size=\marRad, mark=*] plot coordinates {(0.01, -6.31)};
\path[fill=color6,draw=color6,mark size=\marRad, mark=*] plot coordinates {(1.65, -1.51)};
\path[fill=color6,draw=color6,mark size=\marRad, mark=*] plot coordinates {(1.93, -1.30)};
\path[fill=color6,draw=color6,mark size=\marRad, mark=*] plot coordinates {(0.36, 1.93)};
\path[fill=color6,draw=color6,mark size=\marRad, mark=*] plot coordinates {(-1.00, 1.70)};
\path[fill=color6,draw=color6,mark size=\marRad, mark=*] plot coordinates {(-1.02, -2.87)};
\path[fill=color6,draw=color6,mark size=\marRad, mark=*] plot coordinates {(-2.08, -0.29)};
\path[fill=color6,draw=color6,mark size=\marRad, mark=*] plot coordinates {(-2.76, -0.71)};
\path[fill=color6,draw=color6,mark size=\marRad, mark=*] plot coordinates {(7.92, 1.80)};
\path[fill=color6,draw=color6,mark size=\marRad, mark=*] plot coordinates {(-2.88, 2.51)};
\path[fill=color6,draw=color6,mark size=\marRad, mark=*] plot coordinates {(2.56, -3.69)};
\path[fill=color6,draw=color6,mark size=\marRad, mark=*] plot coordinates {(1.74, 0.32)};
\path[fill=color6,draw=color6,mark size=\marRad, mark=*] plot coordinates {(-0.38, 0.43)};
\path[fill=color6,draw=color6,mark size=\marRad, mark=*] plot coordinates {(-1.28, 3.78)};
\path[fill=color6,draw=color6,mark size=\marRad, mark=*] plot coordinates {(-6.27, -2.09)};
\path[fill=color4,draw=color4,mark size=\marRad, mark=pentagon*] plot coordinates {(-0.31, 12.05)};
\path[fill=color5,draw=color5,mark size=\marRad, mark=heart] plot coordinates {(-5.99, 11.16)};
\path[fill=color7,draw=color7,mark size=\marRad, mark=square*] plot coordinates {(12.72, -11.60)};
\path[fill=color8,draw=color8,mark size=\marRad, mark=triangle*] plot coordinates {(7.65, -4.72)};
\path[fill=color9,draw=color9,mark size=\marRad, mark=diamond*] plot coordinates {(8.53, -0.92)};
\path[fill=color9,draw=color9,mark size=\marRad, mark=diamond*] plot coordinates {(7.87, -1.66)};
\end{tikzpicture}
 
\end{frame}
