\begin{frame}{Erste Überlegungen}

bootstrap(frontend)
\end{frame}


\begin{frame}{Java}
	\begin{itemize}
	\item Alle im Team hatten Erfahrungen mit Java
	\item Es gibt Bibliotheken zur Nutzung von Twitter in Java $\Rightarrow$ Daemon nutzt Java
	\item Man kann mit Java Dynamische Webseiten erstellen wenn man Tomcat nutzt $\Rightarrow$ Rest nutzt Java
	\end{itemize}
\end{frame}


\begin{frame}{Datenbank oder Dateibasiert}
	\begin{itemize}
		\item Daten(Tweets) müssen gespeichert werden: Dateien oder Datenbank?
		\item relationale Datenbank erlaubt speichern und erstellen Individueller Abfragen
		\item MySQL die Datenbank mit der die meisten Leute Erfahrung haben
		\item Genutzt wurde MySQL mit InnoDB 
	\end{itemize}
\end{frame}

\begin{frame}{Datenbank: Anwendung}
	\begin{itemize}
	\item Ursprüngliches Schema wurde Anhand der Twitter-API erstellt(To-Do: Bild vom Schema einfügen)
	\item Schema versucht alle Relationen von Twitter abzubilden und alle für uns relevanten Informationen zu liefern
	\item Kleine Änderungen behoben Probleme mit Umlauten, zu kurzen Spalten, ergänzen Sentiment und ähnliches
	\item Problem: Performance sinkt extrem sobald Datenbestand gewisse Größe erreicht
	\item Quick'n'Dirty Lösung am Ende der Programmierzeit: teilweise de-normalisierung der DB / optimieren von Querys
	\item Besser: Überarbeiten des gesamten Schemas
	\end{itemize}
\end{frame}

\begin{frame}{Datenbank: Erfahrungen}
	\begin{itemize}
		\item Gutes Schema ist wichtig für Performance
		\item Indizes sollten zusammen mit dem Schema erarbeitet werden
		\item Gute Querys sind ebenfalls sehr wichtig, sie haben einen großen Einfluss auf die Performance
		\item Connections offen halten, Prepared Statements und Batch Inserts bringen große Performance Schübe
	\end{itemize}
\end{frame}

\begin{frame}{Datenbank: Ausblick}
	\begin{itemize}
	\item Haben nur mit MySQL gearbeitet. Niemand im Team hatte größere Erfahrung mit Datenbanken.
	\item Andere relationale oder auch NoSQL Datenbank wäre vielleicht besser geeignet gewesen.
	\item Möglichst Früh mit Vor-/Nachteilen der Datenbanken beschäftigen um passende Auszuwählen: Dazu müssen aber Anwendungen/Abfragen feststehen
	\end{itemize}
\end{frame}


